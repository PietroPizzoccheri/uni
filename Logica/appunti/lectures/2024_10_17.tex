\documentclass[../main.tex]{subfiles}

\begin{document}

\section{Anelli}

\begin{definition}[Anello]
    Sia $X$ un insieme su cui sono definite due operazioni $+$ e $\cdot $. $X$ è un \textbf{anello} con unità $1_X$ se:
    \begin{enumerate}[label=(\roman*)]
        \item $(X, +)$ è un gruppo abeliano
        \item $(X, \cdot)$ è un monoide con unità $1_X$
        \item vale la proprietà distributiva:
              \begin{itemize}
                  \item $a \cdot (b+c) = a \cdot b + a \cdot c$
                  \item $(a+b) \cdot c = a \cdot c + b \cdot c$ , $\forall a,b,c \in X$
              \end{itemize}
    \end{enumerate}
\end{definition}

\begin{definition}[Anello commutativo]
    Diciamo che un anello $X$ è \textbf{commutativo} se il monoide $(X, \cdot)$ è commutativo.
\end{definition}

Indichiamo con "$0$" l'identità del gruppo $(X, +)$.

\begin{example}
    \
    \begin{itemize}
        \item Gli insiemi $\mathbb{Z} ,\mathbb{Q} ,\mathbb{R} ,\mathbb{C} $ con le operazioni di addizione e moltiplicazione sono anelli commutativi con unità, che è il numero "1".
        \item L'insieme delle matrici $n \times n, n> 1$ a valori su $\mathbb{Z} $, su $\mathbb{Q} $, su $\mathbb{R}$ o su $\mathbb{C} $, con l'operazione di somma e il prodotto righe per colonne, è un anello \textbf{non commutativo}, con unità la matrice identità.

              In generale, se $A$ è un anello commutativo con unità, l'insieme $Mat_{n \times n}(A)$ delle matrici a valori in $\mathbb{R} $ con le operazioni di somma e prodotto righe per colonne, è un anello non commutativo con unità.
        \item$\{X\}$ è un anello, detto \textbf{anello nullo}. Le due operazioni sono la stessa e $0 = 1_{\{X\}} = x$.
    \end{itemize}
\end{example}

Considereremo sempre $0 \neq 1_A$ e studieremo solo anelli commutativi con unità. Quindi quando diremo "anello" intendiamo "anello con unità".

\begin{definition}[Zero divisore in un anello]
    Sia $A$ un anello commutativo. Un elemento $x \in A$ è detto \textbf{zero divisore} se esiste $y \in A \backslash \{0\}$
    tale che $x y = 0$.
\end{definition}

\begin{definition}[Elemento invertibile in un anello]
    Diciamo che un elemento $x \in A$ è \textbf{invertibile} se è un elemento invertibile del monoide $(A, \cdot )$.
\end{definition}

\begin{proposition}
    Sia $A$ un anello commutativo. Allora l'insieme degli elementi invertibili di $A$ è disgiunto dall'insieme degli zero-divisori di $A$.
\end{proposition}

\begin{proof}
    Siano $x,y \in A$ tali che $x y = 0$. Se $X$ è invertibile, allora $x^{-1}xy = y = 0$, quindi $x$ non è uno zero-divisore.
\end{proof}

\begin{proposition}[Legge di cancellazione]
    Sia $A$ un anello commutativo e sia $x \in A$ un elemento che non è uno zero-divisore. Allora:
    \begin{equation*}
        \cancel{x}y=\cancel{x}z \rightarrow y = z \qquad \forall y,z \in A
    \end{equation*}
\end{proposition}

\begin{proof}
    Se $xy = xz$ allora $x(y - z) = 0 $. Poiché $x$ non è uno zero-divisore, allora $y-z = 0$, ossia $y=z$.
\end{proof}

\begin{definition}[Dominio di integrità]
    Un anello commutativo privo di zero-divisori non nulli è detto \textbf{dominio di integrità}.
\end{definition}

\begin{definition}[Campo]
    Un anello commutativo i cui elementi non nulli sono tutti invertibili è detto \textbf{campo}.
\end{definition}

\begin{example}
    L'anello $\mathbb{Z}$ è un dominio di integrità, ma non è un campo.\\
    Gli anelli $\mathbb{Q} ,\mathbb{R} ,\mathbb{C} $ sono campi.
\end{example}

\section{Ideali}

\begin{definition}[Ideale]
    Sia $A$ un anello commutativo. Un sottoinsieme $I \subseteq A$ è detto \textbf{ideale} di $A$ se:
    \begin{enumerate}[label=(\roman*)]
        \item $I$ è un sottogruppo di $(A,+)$
        \item $ax \in I, \forall a \in A, x \in I$
    \end{enumerate}
\end{definition}

\begin{example}
    Abbiamo già visto che ogni sottogruppo di $(\mathbb{Z} , +)$ è del tipo $n \mathbb{Z} = \{kn : k \in \mathbb{Z} \}$, dove $n \in \mathbb{N}$.

    Inoltre, se $a \in \mathbb{Z} $ e $x \in n\mathbb{Z} $, ossia $x=kn$ per qualche $k \in \mathbb{Z} $, si ha che $ax=akn \in n \mathbb{Z} $.

    Quindi $n \mathbb{Z} $ è un ideale di $\mathbb{Z}, \forall n \in \mathbb{N}  $, e tutti gli ideali di $\mathbb{Z} $ sono di questo tipo.
\end{example}

\begin{remark}
    Siano $I,J \subseteq A$ ideali di un anello commutativo $A$. Allora:
    \begin{itemize}
        \item $I \cap J$ è un ideale di $A$
        \item $I + J := \{x+y : x \in I, y \in J\}$ è un ideale di $A$
        \item $IJ := \langle \{xy : x \in I, y \in J\} \rangle$ è un ideale di $A$
    \end{itemize}
\end{remark}

\begin{definition}[Ideale generato da un insieme]
    Sia $S \subseteq A$ un sottoinsieme di un anello commutativo. \textbf{L'ideale generato da $S$} è l'intersezione di tutti gli ideali di $A$ che contengono $S$ e lo indichiamo con $\langle S \rangle$.

    Se $S =\{x\}$, diciamo che $\langle S \rangle$ è \textbf{l'ideale principale generato da $x \in A$}.
\end{definition}

\begin{example}
    Abbiamo visto che gli ideali di $\mathbb{Z} $ sono tutti e soli i sottoinsiemi $n \mathbb{Z}  = \langle n \rangle, n \in \mathbb{N} $.
    Quindi gli ideali di $\mathbb{Z} $ sono tutti principali.
\end{example}

\begin{definition}[Anello ad ideali principali]
    un anello i cui ideali sono tutti principali si dice \textbf{anello ad ideali principali}.
\end{definition}

\begin{proposition}
    Sia $A$ un anello commutativo e $I \subseteq A$ un ideale. Allora:
    \begin{itemize}
        \item $I = A$ se e solo se $I$ contiene un elemento invertibile
        \item $A$ è un campo sse i suoi unici ideali sono $\langle 0 \rangle$ e $A = \langle 1_A \rangle$
    \end{itemize}
\end{proposition}

\begin{proof}
    \
    \begin{itemize}
        \item se $I = A$ allora $1_A \in I$ e $1_A$ è invertibile.

              Sia $u \cap I$ un elemento invertibile.

              Allora $u^{-1} \cap A$ e quindi $1_A u u^{-1} \in I$.

              Ne segue che $A = \langle 1_A \rangle \subseteq I$. e quindi $I = A$.
        \item Sia $A$ un campo e sia $I \neq \langle 0 \rangle$. se $ n \in I$ e $x \neq 0$ allora $x$ è invertibile e quindi $I = A$
              per il punto sopra.

              Viceversa, se $\langle 0 \rangle$ e $A$ sono gli unici ideali di $A$,
              e se $x \in A  \backslash \{0\}$, allora $\langle X \rangle = \langle 1_A \rangle$, ossia $ax = 1_A$ per qualche $a \in A$. Quindi $x$ è invertibile.
    \end{itemize}
\end{proof}
\end{document}