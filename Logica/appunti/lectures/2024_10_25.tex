\documentclass[../main.tex]{subfiles}

\begin{document}
Come conseguenza del corollario precedente otteniamo una formula per calcolare la funzione $\varphi$ di Eulero.

Se $p$ è un numero primo, allora ci sono $p^k$ numeri $1 \leq n \leq p^k$.

Di questi numeri $p, 2p, \ldots , p^{k-1}p$ hanno fattori comuni con $p^k$ e quindi
\begin{equation*}
    \varphi(p^k) = p^k - p^{k-1}.
\end{equation*}

se $n = p^{k_1}\ldots p^{k_s}$ per il corollario precedente $(n>1)$:
\begin{align*}
    \varphi(n) & = \varphi(p_1^{k_1}\ldots\varphi(p_s^{k_s}) =            \\
               & = (p_1^{k_1}-p_1^{k_1-1})\ldots(p_s^{k_s}-p_s^{k_s-1}) = \\
               & = p_1^{k_1} \ldots p_s^{k_s}\prod_{\substack{p|n         \\ p \text{ primo}}}\left(1-\frac{1}{p}\right) = n \prod_{\substack{p|n \\ p \text{ primo}}}\left(1-\frac{1}{p}\right)
\end{align*}

\begin{theorem}[di Eulero]
    Sia $n \in \mathbb{N} \setminus \{0\}$ ed $a \in \mathbb{N} \setminus \{0\}$ tale che $MCD\{a,n\} = 1$. Allora:
    \begin{equation*}
        a^{\overline{\varphi(n)}} = \overline{1} \quad \text{in} \quad \mathbb{Z}_n
    \end{equation*}

    (diciamo che $a^{\varphi(n)} \equiv 1 \bmod n$ ossia che $a^{\varphi(n)}$ e 1 sono equivalenti modulo $n$).
\end{theorem}

\begin{proof}
    Sappiamo che la cardinalità del gruppo degli elementi invertibili di $\mathbb{Z}_n$ è $\varphi(n)$.

    Sia $\langle\overline{a}\rangle \subseteq U(\mathbb{Z}_n)$ il sottogruppo generato da $\overline{a} \in U(\mathbb{Z}_n)$. allora $|\langle\overline{a}\rangle|$ divide $\varphi(n)$ ossia $\varphi(n) = k |\langle\overline{a}\rangle|$, per qualche $k \in \mathbb{N}$.

    Sia $c := |\langle\overline{a}\rangle|$ abbiamo che
    \begin{equation*}
        \overline{1} = \overline{a}^c = (\overline{a^c})^k = \overline{a^{ck}} = \overline{a^{\varphi(n)}}
    \end{equation*}
\end{proof}

\begin{corollary}[Piccolo Teorema di Fermat]
    Sia $p$ un numero primo e $a \in \mathbb{N}$.

    Allora in $\mathbb{Z}_p$ abbiamo che $\overline{a} = \overline{a^p}$ ($a^p \equiv a \bmod p$).
\end{corollary}
\begin{proof}
    Se $p$ è primo si ha che $\varphi(p) = p-1$. Allora dal Teo. di Eulero segue che, se $a \neq 0, p \nmid a$, $a^{\varphi(p) \equiv 1 \bmod p}\implies a ^{p-1} \equiv 1 \bmod p \implies a^p \equiv a \bmod p$.

    Se $a = 0$ o $p \mid a$ l' uguaglianza si riduce a $\overline{0} = \overline{0}$.
\end{proof}

\section{Caratteristica di un anello}
\begin{definition}[Caratteristica di un anello]
    sia $A$ un anello. Il sottogruppo $\langle1_A\rangle \subseteq (A,+)$è un gruppo ciclico.

    Quindi esiste un $n \in \mathbb{N}$ tale che $\langle1_A\rangle \simeq \mathbb{Z}_n$. $n$ è detto la caratteristica dell'anello $A$.
\end{definition}
\begin{example}
    La caratteristica di $\mathbb{Z}$ è 0, infatti $\langle1\rangle = \mathbb{Z} \simeq \mathbb{Z}_0$.

    La caratteristica degli anelli $\mathbb{Q},\mathbb{R},\mathbb{C}$ è sempre 0 poiché $\langle1\rangle = \mathbb{Z} \simeq \mathbb{Z}_0$ in $\mathbb{Q},\mathbb{R},\mathbb{C}$
\end{example}

\begin{example}
    Sia $n \in \mathbb{N}$ allora la caratteristica dell'anello $\mathbb{Z}_n$ è $n$.

    Infatti $\langle\overline{1}\rangle = \mathbb{Z}_n$, rispetto all'operazione +
\end{example}

Indichiamo con $CHAR(A)$ la caratteristica di un anello $A$.

\begin{definition}[Sottoanello Fondamentale]
    Sia A un anello e sia $\langle1_A\rangle$ il sottogruppo di $(A,+)$ generato da $1_a$.

    L'intersezione di tutti i sottoanelli di $A$ contenenti $\langle1_a\rangle$ si chiama \textbf{sottoanello fondamentale di A}.
\end{definition}

\begin{example}
    Il sottoanello fondamentale di $\mathbb{Z},\mathbb{Q},\mathbb{R},\mathbb{C}$ è $\mathbb{Z}$
\end{example}

\begin{definition}[Sottocampo Fondamentale]
    sia $K$ un campo, l'intersezione di tutti i sottocampi di $K$ contenenti il gruppo $\langle1_k\rangle \subseteq (K,+)$ si chiama \textbf{sottocampo fondamentale di K}.
\end{definition}

\begin{example}
    Il sottocampo fondamentale di $\mathbb{Q},\mathbb{R},\mathbb{C}$ è $\mathbb{Q}$.

    Se $p \in \mathbb{N}$ è primo, il sottocampo fondamentale di $\mathbb{F}_p$ è $\mathbb{F}_p$ perché $\langle\overline{1}\rangle = \mathbb{F}_p$.
\end{example}

\section{Anello dei polinomi in una indeterminata a coefficienti in un campo}
\begin{definition}[Successione a valori di un campo]
    Sia $K$ un campo. una funzione $f: \mathbb{N} \rightarrow K$ si chiama \textbf{successione a valori in $K$}.
\end{definition}
ad una successione a valori in $K$ corrisponde una serie formale nella variabile $x$ su $K$:
\begin{equation*}
    \sum_{n=0}^{\infty} f(n) x^n
\end{equation*}
Se l'insieme $\{ m \in \mathbb{N} : f(n) \neq 0\}$ è finito diciamo che la serie formale è un polinomio in $x$ di grado $deg(P) := MAX \{ n \in \mathbb{N}: f(n) \neq 0\}$.

Il grado del poliniomio 0 non è definito.

L'insieme dei polinomi in $x$ a coefficienti in $K$ si indica con $K[x]$ ed è un anello commutativo con le operazioni:
\begin{itemize}
    \item somma: $(\sum_{n=0}^{\infty} a_n X^n) + (\sum_{n=0}^{\infty} b_n X^n) = \sum_{n=0}^{\infty} (a_n + b_n) X^n$
    \item prodotto: $(\sum_{n=0}^{\infty} a_n X^n) \cdot (\sum_{n=0}^{\infty} b_n X^n) = \sum_{n=0}^{\infty} (\sum_{k=0}^{n} a_k b_{n-k}) X^n$
\end{itemize}
L'unità di $K[x]$ è il polinomio $1_k$.

\begin{example}
    in $\mathbb{F}_2[x]$ siano $P := 1 + X^2 + X^3$ e $Q:= X + X^2$.

    Allora $P+Q = 1 + X + X^2 + X^3$ e $P \cdot Q = X + X^2 + X^3 + X^5$
\end{example}

\begin{proposition}
    Siano $P,Q \in K[x]$ polinomi non nulli. Allora:
    \begin{equation*}
        deg(P \cdot Q) = deg(P) + deg(Q).
    \end{equation*}
    In particolare $K[x]$ è un dominio di integrità.
\end{proposition}

\begin{definition}[Polinomio monico]
    Un polinomio si dice \textbf{monico} se il coefficiente del termine di grado massimo è 1.
\end{definition}

\begin{definition}
    sia $K$ un campo. un polinomio $P \in K[x]$ si dice \textbf{irriducibile} se i suoi unici divisori sono del tipo $a, aP$ con $a \in K \setminus \{0\}$.

    Altrimenti si dice \textbf{riducibile}.
\end{definition}

\begin{example}
    In $\mathbb{F}_2[X]$ il polinomio $X^2 + 1$ è irriducibile, infatti $X^2 + 1 = (X + 1)^2$.

    Quindi $X + 1$ divide $X^2 + 1$ e $X + 1 \notin K \setminus \{0\}$.
\end{example}

\begin{example}
    In $K[X]$ ogni polinomio di grado 1 è irriducibile, infatti se $deg(P) = 1$ allora $P = aX + b$ con $a,b \in K, a \neq 0$.

    I suoi divisori sono $c$ e $c^{-1} (aX + b), c \in K \setminus \{0\}$.
\end{example}

\begin{definition}[Radice di un polinomio]
    Sia $\alpha \in K$. L'elemento $\alpha$ è detto \textbf{radice} del polinomio $P = \sum_{n=0}^{\infty} a_n X^n \in K[X]$ se $P(\alpha) = \sum_{n=0}^{\infty} a_n \alpha^n = 0$.
\end{definition}

Anche nell'anello $K[X]$ come in $\mathbb{Z}$ abiamo un algoritmo di divisione Euclidea.

Se $f(X), g(X) \in K[X]$ sono polinomi non nulli allora esistono unici polinomi $q(X), r(X) \in K[X]$ tali che:

$f(X) = q(X) \cdot g(X) + r(X)$ e $r(X) = 0$ oppure $deg(r) < deg(g)$.

$q(X)$ si chiama \textbf{quoziente} e $r(X)$ si chiama \textbf{resto} della divisione.

Ne segue il seguente teorema, dimostrato come in $\mathbb{Z}$:
\begin{theorem}
    L'anello $K[X]$ è a ideali principali. Se $I = \langle p(X)\rangle$ allora esiste un unico generatore monico di $I$.
\end{theorem}

\begin{definition}[Massimo Comune Divisore]
    Definiamo il \textbf{massimo comune divisore} di due polinomi $f(X), g(X) \in K[X]$ come l'unico massimo comune divisore monico.
\end{definition}

Come in $\mathbb{Z}$ possiamo trovarlo con l'algoritmo delle divisioni successive che dà anche un \underline{identità di Bézout}.

\begin{example}
    $f(X) = X^4 - X^3 -4X^2 + 4X + 1$ e $g(X) = X^2 - 1$ in $\mathbb{Q}[X]$, allora:
    \begin{align*}
        f(X) & = g(X)(X^2 - 3) + (X - 2)                 \\
        g(X) & = (X - 2)(X + 1) +1 \implies MCD(f,g) = 1
    \end{align*}
    inoltre
    \begin{align*}
        1 & = g(X) - (X - 2)(X + 1) =                \\
          & = g(X) - [f(X) - g(X)(X^2 - 3)](X + 1) = \\
          & = -(X - 1)f(X) + (X^3 + X^2 - 3X -2)g(X)
    \end{align*}
\end{example}

\begin{proposition} sia $K$ un campo e $P(X) \in K[X]$ un poliniomio irriducibile. Allora l'anello quoziente $\nicefrac{K[X]}{\langle P(X)\rangle}$ è un campo.
\end{proposition}
\begin{proof}
    Sia $[f] \in \nicefrac{K[X]}{\langle P(X)\rangle}$ tale che $[p] \neq [0]$ ossia $p(X)$ non divide $f(X)$.

    Dunque $MCD\{f(X),p(X)\} = 1$ perchè $p(X)$ è irriducibile.

    Quindi abbiamo un'identità di Bézout $a(X)f(X) + b(X)p(X) = 1$.

    Ossia $[a(X)] = [f(X)]^{-1}$ in $\nicefrac{K[X]}{\langle P(X)\rangle}$.
\end{proof}

\begin{example}
    In $\mathbb{F}_2[X]$ il polinomio $P(X) = 1 + X + X^2$ è irriducibile.

    Infatti non ha radici in $\mathbb{F}_2$.

    Quindi l'anello $\nicefrac{\mathbb{F}_2[X]}{\langle1 + X + X^2\rangle}$ è un campo, che chiamiamo $\mathbb{F}_4$.

    Un elemento di $\mathbb{F}_4$ è della forma $a_0 + a_1X$ con $a_0,a_1 \in \mathbb{F}_2$.

    La tavola moltiplicativa è la seguente:
    \begin{center}
        \begin{tabular}{c|c|c|c|c}
            $\cdot$ & 0 & 1     & X     & 1 + X \\
            \hline
            0       & 0 & 0     & 0     & 0     \\
            \hline
            1       & 0 & 1     & X     & 1 + X \\
            \hline
            X       & 0 & X     & 1 + X & 1     \\
            \hline
            1 + X   & 0 & 1 + X & 1     & X     \\
        \end{tabular}
    \end{center}
    L'inverso di $X$ è $1 + X$.
\end{example}

\begin{example}
    In $\mathbb{F}_3[X]$ il polinomio $P(X) = 1 + X^2$ è irriducibile.

    Indichiamo con $\mathbb{F}_9$ il campo $\nicefrac{\mathbb{F}_3[X]}{\langle1 + X^2\rangle}$.

    Un elemento di $\mathbb{F}_9$ è della forma $a_0 + a_1X$ con $a_0,a_1 \in \mathbb{F}_3$ quindi sono 9.

    La tavola moltiplicativa è la seguente:
    \begin{center}
        \begin{tabular}{c|c|c|c|c|c|c|c|c|c}
            $\cdot$ & 0 & 1      & 2      & X      & 1 + X  & 2 + X  & 2X     & 1 + 2X & 2 + 2X \\
            \hline
            0       & 0 & 0      & 0      & 0      & 0      & 0      & 0      & 0      & 0      \\
            \hline
            1       & 0 & 1      & 2      & X      & 1 + X  & 2 + X  & 2X     & 1 + 2X & 2 + 2X \\
            \hline
            2       & 0 & 2      & 1      & 2X     & 2 + 2X & 1 + 2X & X      & 2 + X  & 1 + X  \\
            \hline
            X       & 0 & X      & 2X     & 2      & 2 + X  & 2 + 2X & 1      & 1 + X  & 1 + 2X \\
            \hline
            1 + X   & 0 & 1 + X  & 2 + 2X & 2 + X  & 2X     & 1      & 1 + 2X & 2      & X      \\
            \hline
            2 + X   & 0 & 2 + X  & 1 + 2X & 2 + 2X & 1      & X      & 1 + X  & 2X     & 2      \\
            \hline
            2X      & 0 & 2X     & X      & 1      & 1 + 2X & 1 + X  & 2      & 2 + 2X & 2 + X  \\
            \hline
            1 + 2X  & 0 & 1 + 2X & 2 + X  & 1 + X  & 2      & 2X     & 2 + 2X & X      & 1      \\
            \hline
            2 + 2X  & 0 & 2 + 2X & 1 + X  & 1 + 2X & X      & 2      & 2 + X  & 1      & 2X     \\
        \end{tabular}
    \end{center}
\end{example}

\begin{theorem}[di Ruffini]
    Sia $f(X) \in K[X]$ un polinomio non nullo.

    Se $\alpha \in K$, il resto della divisione di $f(X)$ per $X - \alpha$ è $f(\alpha)$, in particolare $\alpha$ è una radice di $f(X)$ s.s.e. $X - \alpha$ divide $f(X)$ in $K[X]$.
\end{theorem}

\begin{proof}
    $f(X) = (X - \alpha) q(X) + r(X)$ con $r(X) = 0$ oppure $deg(r(X)) < 1$.

    Quindi $r(X)$ è un polinomio costante, $r(X) = x \in K$.

    Calcolando in $\alpha$ otteniamo $f(\alpha) = c$.
\end{proof}

\begin{example}
    Il polinomio $X^2 + 1 \in \mathbb{R}[X]$ non ha radici in $\mathbb{R}$ quindi è irriducibile e $\nicefrac{\mathbb{R}[X]}{\langle X^2 + 1\rangle}$ è un campo isomorfo a $\mathbb{C}$, dove l'isomorfismo è dato dall'assegnazione $1 \mapsto 1$ e $x \mapsto i$
\end{example}
\end{document}