\documentclass[../main.tex]{subfiles}

\begin{document}
\chapter{Logica Proposizionale}
\section{Sintassi}
Introduciamo il linguaggio della logica proposizionale. L'alfabeto è costituito da:
\begin{enumerate}
    \item Insime numerabile di variabili
    \item Connettivi logici: $\neg, \land, \lor$
\end{enumerate}
le parole del linguaggio possono essere:
\begin{enumerate}
    \item \textbf{Un Letterale}: variabile $x$ o la sua negazione $\neg x$
    \item \textbf{Una Clausola}: disgiunzione finita di letterali $l_1 \lor \ldots \lor l_n$
    \item \textbf{Una Formula CNF (forma normale congiuntiva)}: congiunzione finita di clausole $C_1 \land \ldots \land C_n$
\end{enumerate}
Per ogni letterale $L$ definiamo
\begin{equation*}
    \overline{L} = \begin{cases}
        \neg x & \text{se } L \text{ è } x      \\
        x      & \text{se } L \text{ è } \neg x
    \end{cases}
\end{equation*}
Indichiamo con $VAR(F)$ l'insieme delle variabili che appaiono in una formula CNF $F$.

Useremo anche le parentesi $(,)$ come simboli ausiliari per rendere chiara la lettura delle formule CNF.
\begin{example}
    $(\neg x_1 \lor x_2) \land x_1 \land (x_2 \lor x_3)$ è una formula CNF. Chiamiamola F. Allora $VAR(F)=\{x_1,x_2,x_3\}$.

    Le clausole di $F$ sono $\{\neg x_1 \lor x_2, x_1, x_2 \lor x_3\}$ e i letterali di $F$ sono $\{x_1, x_2, x_3, \neg x_1 \}$
\end{example}

\section{Semantica}
\begin{definition}[Assegnazione appropriata]
    Sia $F$ una formula CNF. Una \textbf{assegnazione appropriata a F} è una funzione
    \begin{equation*}
        V: X \rightarrow \{0,1\}
    \end{equation*}
    dove $X \supseteq VAR(F)$.
\end{definition}

L'insieme $\{0,1\}$ è l'insieme dei valori di verità di F e può essere interpretato come \{falso, vero\}.

\begin{definition}[Soddisfacibilità]
    Sia $F$ una formula e $V$ un'assegnazione appropriata a $F$.

    Diciamo che \textbf{V soddisfa F}, scritto $V \vDash F$, se
    \begin{enumerate}
        \item \underline{F è una variabile X}: $V \vDash X$, significa che $V(X) = 1$
        \item \underline{F è il letterale $\neg X$}: $V \vDash \neg X$, significa che $V(X) = 0$
        \item \underline{F è una clausola $L_1 \lor \ldots \lor L_n$}: $V \vDash F$, significa che $V$ soddisfa almeno uno dei letterali $L_i$
        \item \underline{F è una congiunzione di clausole} $C_1 \land \ldots \land C_n$: $V \vDash F$, significa che $V$ soddisfa tutte le clausole $C_i$
    \end{enumerate}
    In questo modo abbiamo dato un significato al linguaggio definito precedentemente.

    Se $V$ non soddisfa $F$, scriveremo $V \nvDash F$.
\end{definition}
\begin{example}
    Sia $F$ la formula CNF $(\neg x_1 \lor x_2) \land x_1 \land (x_2 \lor x_3)$ e sia $U: \{x_1,x_2,x_3\} \rightarrow \{0,1\}$ tale che $U(x_1) = U(x_2) = 1, U(x_3) = 0$.

    Dunque
    \begin{align*}
        U(\neg x_1)          & = 0 \\
        U(\neg x_1 \lor x_2) & = 1 \\
        U(x_2 \lor x_3)      & = 1 \\
        U(F)                 & = 1
    \end{align*}
    Quindi $U \vDash F$.

    Sia $V: \{x_1,x_2,x_3\} \rightarrow \{0,1\}$ tale che $V(x_1) = 1, V(x_2) = V(x_3) = 0$.

    Allora
    \begin{align*}
        V(\neg x_1)          & = 0 \\
        V(\neg x_1 \lor x_2) & = 0 \\
        V(x_2 \lor x_3)      & = 0 \\
        V(F)                 & = 0
    \end{align*}
    quindi $V \nvDash F$.
\end{example}

\begin{definition}[Formula soddisfacibile]
    Diciamo che una formula è \textbf{soddisfacibile} se esiste un'assegnazione $V: VAR(F) \rightarrow \{0,1\}$ che la soddisfa ($V \vdash F$).

    Altrimenti è \textbf{insoddisfacibile}.
\end{definition}
La formula dell' esempio precedente è soddisfacibile, $x \land \neg x$ è insoddisfacibile.

\begin{definition}[Tautologia]
    Una formula $F$ è una \textbf{tautologia} se per ogni assegnazione $V$ si ha $V \vDash F$.
\end{definition}
La formula $x \lor \neg x$ è una tautologia, la formula dell' esempio precedente no

\begin{definition}[Conseguenza logica]
    Date due formule $F,G$ diciamo che $G$ è \textbf{conseguenza logica} di $F$, se per ogni assegnazione $V$ appropriata ad entrambe si ha che $V \vDash F \implies V \vDash G$.
\end{definition}
\begin{example}
    La formula $y$ è conseguenza logica della formula $F := (\neg x \lor y) \land x$.

    Infatti l'unica assegnazione che soffisfa $F$ è $x \rightarrow 1, y \rightarrow 1$.

    Tale assegnazione soddisfa anche $y$.
\end{example}
\begin{definition}[Implicazione logica]
    Definiamo l'\textbf{implicazione logica} tra due formule $F,G$ come
    \begin{gather*}
        x \implies y := \neg x \lor y \\
        \begin{tabular}{c|c|c}
            \hline
            $x$ & $y$ & $x \implies y$ \\
            \hline
            0   & 0   & 1              \\
            0   & 1   & 1              \\
            1   & 0   & 0              \\
            1   & 1   & 1              \\
        \end{tabular}
    \end{gather*}
\end{definition}
\begin{definition}[Equivalenza logica]
    Due formule $F,G$ sono logicamente equivalenti se $F$ è conseguenza logica di $G$ e $G$ è conseguenza logica di $F$. In tal caso scriviamo $F \equiv G$.
\end{definition}
\begin{example}
    Nell'esempio precedente abbiamo visto che $y$ è conseguenza logica di $F:= (\neg x \lor y) \land x $, che possiamo ricrivere come $(x \implies y) \land x$.

    Poiché $x \mapsto 0, y \mapsto 1$ soddisfa $y$ ma non $F$, abbiamo che $F$ non è conseguenza logica di $y$.
\end{example}
\begin{example}\
    \begin{itemize}
        \item $l_1 \lor l_2 \equiv l_2 \lor l_1$ (la disgiunzione è commutativa)
        \item $c_1 \land c_2 \equiv c_2 \land c_1$ (la congiunzione è commutativa)
        \item $c \land c \equiv c$ e $l \lor l \equiv l$ (congiunzione e disgiunzione sono idempotenti)
    \end{itemize}
\end{example}
Diamo altre definizioni:
\begin{definition}[Doppia implicazione]
    \begin{equation*}
        x \iff y := (x \implies y)\land(y \implies x)
    \end{equation*}
\end{definition}

\begin{definition}[Negazione di formule CNF]
    \begin{enumerate}
        \item $l$ letterale: $\neg l = \overline{l}$
        \item $\neg (l_1 \lor \ldots \lor l_n) := \neg l_1 \land \ldots \land \neg l_n$
        \item $\neg (c_1 \land \ldots \land c_n) := \neg c_1 \lor \ldots \lor \neg c_n$
    \end{enumerate}
\end{definition}
L'interpretazione di questa definizione di negazione è quella corretta grazie alle leggi di De Morgan e alla proprietà distributiva di $\lor$ e $\land$.
\begin{example}
    Consideriamo il problema di colorare i vertici di un quadrato con due colori in modo che i vertici su uno stesso lato abbiano colori diversi.

    Tale problema ha ovviamente una soluzione:
    \begin{center}
        \begin{tikzpicture}
            \draw (0,0) -- (2,0) -- (2,2) -- (0,2) -- (0,0);
            \draw (0,0) node[below left] {2};
            \draw (2,0) node[below right] {1};
            \draw (2,2) node[above right] {2};
            \draw (0,2) node[above left] {1};
        \end{tikzpicture}
    \end{center}
    Formuliamo il problema come una formula CNF.

    Potremo così dire che il problema è soddisfacibile se e solo se tale formula CNF è soddisfacibile. $x_{ij}$ indica "il vertice $i$ ha colore $j$" $\forall 1 \leq i \leq 4, 1 \leq j \leq 2$
    \begin{align*}
        F & := (x_{11} \lor x_{12}) \land (x_{21} \lor x_{22}) \land (x_{31} \lor x_{32}) \land (x_{41} \lor x_{42}) \land                                                                              \\
          & \land (\neg x_{11} \lor \neg x_{12}) \land ( \neg x_{21} \lor \neg x_{22}) \land ( \neg x_{31} \lor \neg x_{32}) \land ( \neg x_{41} \lor \neg x_{42}) \land                                \\
          & \land (\neg x_{11} \lor \neg x_{21}) \land ( \neg x_{12} \lor \neg x_{22}) \land ( \neg x_{21} \lor \neg x_{31}) \land ( \neg x_{22} \lor \neg x_{32}) \land (\neg x_{31} \lor \neg x_{41})
    \end{align*}
    l'assegnazione $x_{11} \mapsto 1, x_{12} \mapsto 0, x_{21} \mapsto 0, x_{22} \mapsto 1, x_{31} \mapsto 1, x_{32} \mapsto 0, x_{41} \mapsto 0, x_{42} \mapsto 1$ soddisfa $F$
\end{example}

\chapter{Logica Modale}
\section{Sintassi}
La logica modale è una estensione della logica proposizionale.

L'alfabeto è quello della logica proposizionale a cui si aggiungono i \textbf{connettivi modali}:
\begin{enumerate}
    \item Un insieme numerabili di variabili (o formule atomiche)
    \item I connettivi logici $\neg, \land, \lor, \implies, \iff$
    \item I simboli ausiliari $(,)$
    \item I connettivi modali $\Box$ (\textbf{Scatola o Box}) e $\Diamond$ (\textbf{Diamante o Diamond})
\end{enumerate}
Le parole del lingiaggio sono le formule ben formate (FBF) definite in modo ricorsivo:
\begin{enumerate}
    \item Ogni variabile è una FBF
    \item Se $A$ è una FBF, allora $\neg A, \Box A, \Diamond A$ sono FBF
    \item Se $A,B$ sono FBF, allora $(A \land B), (A \lor B), (A \implies B), (A \iff B)$ sono FBF
\end{enumerate}
Alcune letture dei simboli $\Box$ e $\Diamond$:
\begin{itemize}
    \item La lettura piu comune è: $\Box A$:"è necessario che A", $\Diamond A$: "è possibile che A".

          Secondo questa lettura i connettivi modali possono essere definiti uno in termini dell'altro:
          \begin{gather*}
              \Box A \equiv \neg \Diamond \neg A\\
              \Diamond A \equiv \neg \Box \neg A
          \end{gather*}
    \item Logiche modali epistemiche: $\Box A$: "si sa che A"
    \item Logiche modali deonetiche: $\Box A$: "è obbligatorio che A"
    \item Logiche modali doxastiche: $\Box A$: "si crede che A"
    \item Logica modale dimostrativa: $\Box A$: "è dimostrabile che A"
\end{itemize}

Come abbiamo visto, la logica proposizionale è una logica vero-funzionale: assegnando valori "0" e "1" alle variabili possiamo assegnare un valore "0" o "1" ad una formula in modo univoco, che corrisponde alla nostra inutizione di negazione, disgiunzione, congiunzione.

Per la logica modale la situazione è più complicata.

Interpretando il simbolo "$\Box$" come operatore di necessità, ossia "$\Diamond$" come operatore di possibilità, possiamo essere, ad esempio, d'accordo che le formule
\begin{equation*}
    \Box A \implies \Diamond A, \quad A \implies \Diamond A
\end{equation*}
siano vere, ma è vera la formula
\begin{equation*}
    A \implies \Box \Diamond A \quad \text{?}
\end{equation*}
Non è chiaro se sia vera o falsa. Nel caso della logica epistemica, l'operatore "$\Box$" si indica di solito con "K" (da "knowledge").

In questo contesto, la formula $KA \implies A$ (se si sa che A allora A vale) sembra dover essere vera.

Invece la formula $A \implies KA$ (se vale A allora si sa che A) sembra essere falsa perchè non si è onniscenti.

\section{Semantica dei mondi possibili (semantica di Kripke)}
\begin{definition}[Frame]
    Un \textbf{Frame} è una coppia $(S,R)$, dove $S$ è un insieme non vuoto detto \textbf{insieme dei mondi} e $R \subseteq S \times S$ è una relazione su $S$, detta \textbf{relazione di accessibilità} (se $(x,y) \in R$ si dice che $y$ è accessibile da $x$).
\end{definition}
Un Frame può essere rappresentato con un grafo diretto con cappi (loop) i cui vertici sono gli elementi dell'insieme $S$ e ho una freccia da $x$ a $y$ se $(x,y) \in R$.
\end{document}