\documentclass[../main.tex]{subfiles}

\begin{document}
\chapter{Teoria degli anelli commutativi e dei campi}
\section{Insiemi}
Un insieme è una collezione di oggetti, detti elementi dell'insieme.
\begin{align*}
    \mathbb{N} & = \left\{ 0,1,2,3,\ldots \right\}                                        & \text{(Naturali)}  \\[1ex]
    \mathbb{Z} & = \left\{ \ldots,-2,-1,0,1,2,\ldots \right\}                             & \text{(Interi)}    \\[1ex]
    \mathbb{Q} & = \left\{ \frac{a}{b} \;\middle|\; a,b \in \mathbb{Z}, b \neq 0 \right\} & \text{(Razionali)} \\[1ex]
    \mathbb{R} & = \left\{ x \mid x \text{ is a real number} \right\}                     & \text{(Reali)}     \\[1ex]
    \mathbb{C} & = \left\{ a + bi \mid a,b \in \mathbb{R}, i^2=-1 \right\}                & \text{(Complessi)}
\end{align*}

\subsection{Operazioni tra insiemi}
\begin{itemize}
    \item $\subseteq$  (Inclusione tra insiemi)
    \item $\subsetneq$ (Inclusione propria tra insiemi)
\end{itemize}

$X \subseteq Y$ si legge $X$ è sottoinsieme di $Y$ o $X$ è incluso in $Y$.

Se $X$ è un insieme finito, indico con $|X|$ il numero di elementi di $X$, detto anche la \textbf{cardinalità di $X$}.

\begin{definition}[Insieme vuoto]
    L'insieme $\emptyset$ è l'insieme che non contiene alcun elemento. $\emptyset = \{ \}$ e $\abs{\emptyset} = 0$
\end{definition}

\begin{definition}[Prodotto Cartesiano]
    Siano $X$ e $Y$ due insiemi. L'insieme $X \times Y := \{ (x,y) : x \in X , y \in Y \}$ lo chiamiamo \textbf{prodotto cartesiano} di $X$ e $Y$.
\end{definition}

\begin{definition}[Insieme delle parti]
    Sia $A \in \mathcal{P} (x)$, dove $\mathcal{P} (X) := \{ A : A \subseteq X \}$ è detto \textbf{Insieme delle parti di $X$}
\end{definition}

\begin{definition}[Complementare]
    L'insieme $A^\mathrm{C} := X \setminus A$ è detto \textbf{complementare} di $A$ % TODO: Why is \complementary not working?
\end{definition}

\section{Funzioni}
Siano $X$ e $Y$ due insiemi. \textbf{Una funzione $f$ da $X$ a $Y$} è un sottoinsieme $F \subseteq X \times Y$
tale che:
\begin{itemize}
    \item $(x, y_1) \in F$, $(x,y_2) \in F$ $\implies y_1 = y_2$, $\forall x \in X$, $y_1,y_2 \in Y$.
    \item $x \in X \implies \exists y \in Y \text{ tale che } (x,y) \in F$
\end{itemize}
Una funzione $F \subseteq X \times Y$ la indichiamo con $f : X \to Y$. E scriviamo $f(x) = y$ se $(x,y) \in F$.

\begin{definition}[Funzione identità]
    La funzione $Id_x : X \rightarrow X $ tale che $Id_x (x) = x, \forall x \in X$ la chiamiamo \textbf{funzione
        identità su $X$}
\end{definition}

\begin{definition}[Funzione iniettiva]
    Una funzione $f: X \rightarrow Y$ è \textbf{iniettiva} se $\forall x_1, x_2 \in X, f(x_1) = f(x_2)
        \implies x_1 = x_2$
\end{definition}

\begin{definition}[Funzione suriettiva]
    Una funzione $f: X \rightarrow Y$ è \textbf{suriettiva} se $Im(f) = Y$, dove $Im(f) = \{ y \in Y :
        \exists x \in X \text{ tale che } f(x) = y \}$ è detta \textbf{immagine di $f$}
\end{definition}

\begin{definition}[Funzione biunivoca]
    Una funzione $f: X \rightarrow Y$ è \textbf{biunivoca} se è sia iniettiva che suriettiva.
\end{definition}

\subsection{Composizione di funzioni}
Siano $f: X \rightarrow Y$ e $g: Y \rightarrow Z$ due funzioni. La \textbf{composizione di $f$ e $g$}
è la funzione $g \circ f : X \rightarrow Z$ tale che $(g \circ f)(x) = g(f(x))$, $\forall x \in X$.

\begin{definition}[Funzione invertibile]
    una funzione $f: X \rightarrow Y$ è detta \textbf{invertibile} se esiste una funzione $g: Y \rightarrow X$ tale che
    \begin{itemize}
        \item $g \circ f = Id_X$
        \item $f \circ g = Id_Y$
    \end{itemize}
    la funzione $g$ è detta \textbf{funzione inversa di $f$} e la indichiamo con $f^{-1}$.
\end{definition}
\begin{remark}
    Una funzione $f: X \rightarrow Y$ è invertibile se e solo se è biunivoca.
\end{remark}

\subsection{Operazioni su insiemi}

\begin{definition}[Operazione]
    Una funzione $f: X \times X \rightarrow X$ è detta \textbf{operazione su $X$}. Invece di $f(x,y)$
    scriveremo $x \cdot y$.
\end{definition}

\begin{definition}[Operazione associativa]
    Un'operazione $\cdot$ su $X$ è detta \textbf{associativa} se $(x \cdot y) \cdot z = x \cdot (y \cdot z)$,
    $\forall x,y,z \in X$.
\end{definition}

\begin{definition}[Operazione commutativa]
    Un'operazione $\cdot$ su $X$ è detta \textbf{commutativa} se $x \cdot y = y \cdot x$, $\forall x,y \in X$.
\end{definition}

\begin{example}
    \
    \begin{itemize}
        \item $\mathcal{P}(X) $con l'operazione di unione $\cup$ è associativa e commutativa, così come lo è con
              l'intersezione $ \cap $.
        \item $A \backslash B  := A \cup B^C$ \textbf{(differenza insiemistica)} è un'operazione su
              $\mathcal{P}(X)$.\newline non è associativa: sia $A \neq \emptyset.$ Allora $A \backslash (A \backslash A)
                  = A \neq (A\backslash A) \backslash A = \emptyset$\newline non è commutativa: $A \backslash \emptyset
                  = A \neq \emptyset \backslash A = \emptyset$, se $A \neq \emptyset$
        \item $A \Delta B := (A \backslash B) \cup (B \backslash A)$ \textbf{(differenza simmetrica)}
              è un'operazione su $\mathcal{P}(X)$. \newline è commutativa e anche associativa, facilmente verificabile
              coi diagrammi di Venn.
        \item Sia $F(X) := \{f : X\rightarrow X\}$.\newline La composizione" $\circ$" è un'operazione su $F(X)$.
              \newline è associativa, ma non è commutativa.
        \item $a \circ b = \frac{a + b}{2} $ è un'operazione commutativa su $\mathbb{Q}$, ma non associativa.
    \end{itemize}
\end{example}

\end{document}