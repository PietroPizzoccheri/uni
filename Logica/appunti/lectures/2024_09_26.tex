\documentclass[../main.tex]{subfiles}

\begin{document}
\begin{remark}
    Il nucleo di un morfismo di gruppi $f: G_1 \rightarrow G_2$ è il sottogruppo di $G_1$ definito come: $Ker(f):= \{x \in G_1 : f(x) = e_2\}$.
    \textbf{Il nucleo è un sottogruppo di $G_1$}. e \textbf{$Im(f)$ è un sottogruppo di $G_2$}.
\end{remark}

\begin{definition}[Isomorfismo]
    Un \textbf{isomorfismo di monoidi (e di gruppi)} è un morfismo biunivoco, tale che la funzione inversa sia un morfismo.
\end{definition}

\begin{proposition}
    Sia $f: M_1 \rightarrow M_2$ un morfismo di monoidi. Se $f$ è biunivoco, allora è un isomorfismo. Questo vale anche per i gruppi.
\end{proposition}

\begin{proof}
    Dobbiamo far vedere che la funzione inversa $f^{-1} : M_2 \rightarrow M_2$ è un morfismo di monoidi.

    Poiché $f(e_1) = e_2$, allora $f^{-1}(e_2) = e_1$.
    Siano $x_2,y_2 \in M_2$, allora esistono $x_1,y_1 \in M_1$ tali che $f(x_1)= x_2 , f(y_1)=y_2$.
    Quindi $f^{-1} (f(x_1)f(y_1)) = f^{-1}(f(x_1 y_1)) = x_1 y_1 = f^{-1}(x_2) f^{-1}(y_2)$
\end{proof}

\begin{example}
    \
    \begin{itemize}
        \item Siano $M_1 = (\mathcal{P}(X), \cup) \text{ e } M_2 = (\mathcal{P}(X), \cup), \text{ dove } X$
              è un insieme. Sia $f : M_1 \rightarrow M_2$ definita ponendo $f(A) = A^C , \forall A \subseteq X$.
              la funzione $f$ è biunivoca. Inotre, dalle formule di De Morgan segue che $f(A \cap B) =
                  (A \cap B)^C = A^C \cup B^C = f(A) \cup f(B)$. Quindi $f$ è un isomorfismo di monoidi, poiché
              $f(X) = X^C = \varnothing$, essendo $X$ l'identità di $M_1$ e $\varnothing$ l'identità di $M_2$.
        \item Sia $\mathbb{Z}_2 := \{0,1\}$ con l'operazione definita come: $0+0=0, 0+1=1+0=1, 1+1=0$.
              Sia $X := \{1,2,\ldots,n\}, n \in \mathbb{N}$. La funzione $f: \mathcal{P}(X) \rightarrow
                  \mathbb{Z}_2 \times \ldots \times \mathbb{Z}_2$ (n volte) definita da: $f(A) = (a_1,a_2,\ldots,a_n)$,
              dove $a_i = 1$ se $i \in A$ e $a_i = 0$ se $i \notin A$. \newline è un isomorfismo del gruppo
              $(\mathcal{P}(X), \Delta)$ con il gruppo $\mathcal{P}(X) \rightarrow \mathbb{Z}_2 \times \ldots
                  \times \mathbb{Z}_2 = (\mathbb{Z}_2)^n$
    \end{itemize}
\end{example}

\textbf{Vediamo ora come ogni monoide finito è isomorfo a un monoide di matrici quadrate, dove l'operazione
    è il prodotto righe per colonne.}

Sia $M=\{x_1,\ldots,x_n\}$ un monoide, $\abs{M} = n \in \mathbb{N}$, con identità $e = x_1$.

Per ogni $x \in M$ definiamo una matrice $A(x) \in Mat_{n \times n}(\mathbb{Z})$ nel seguente modo:
\begin{itemize}
    \item $A(x)_{ij} = 1$ se $x_i \cdot x = x_j$
    \item $A(x)_{ij} = 0$ altrimenti.
\end{itemize}
La funzione $F : M \rightarrow Mat_{n \times n} (\mathbb{Z})$ ($x \mapsto A(x)$) è iniettiva.

Infatti, se $A(x) = a(y)$, allora $A(x)_{i1} = A(y)_{i1} \; \forall i \in \{1,\ldots,n\}$.

Quindi se $A(x)_{i1} = A(y)_{i1} = 1$, allora $xx_1 = xe = x = yx_1 = y$.

Risulta inoltre facile vedere che $A(xy) = A(x)A(y)$ (prodotto righe per colonne), ossia che $F$ è un morfismo di monoidi ($Mat_{n \times n}(\mathbb{Z})$ è un monoide con l'operazione di prodotto righe per colonne, la cui identità è la matrice $I_n$).

Quindi $F: M \rightarrow Im(F)$ è un isomorfismo di monoidi.

\begin{example}
    Sia $M = (\mathbb{Z}_2, \cdot)$ il monoide definito da:
    \vspace*{0.5em}
    \begin{center}
        \begin{tabular}{c|c|c}
            $\cdot$ & $0$ & $1$ \\
            \hline
            $0$     & $0$ & $0$ \\
            \hline
            $1$     & $0$ & $1$ \\
        \end{tabular}
    \end{center}
    \vspace*{0.5em}
    Costruiamo un sottomonoide di $\text{Mat}_{4 \times 4}(\mathbb{Z})$ isomorfo a $M \times M = \{(0,0), (0,1), (1,0), (1,1)\}$.
    \vspace*{1em}
    \begin{center}
        $(0,0) \mapsto
            \begin{bmatrix}
                1 & 1 & 1 & 1 \\
                0 & 0 & 0 & 0 \\
                0 & 0 & 0 & 0 \\
                0 & 0 & 0 & 0
            \end{bmatrix}$
        $(0,1) \mapsto \begin{bmatrix}
                1 & 0 & 1 & 0 \\
                0 & 1 & 0 & 1 \\
                0 & 0 & 0 & 0 \\
                0 & 0 & 0 & 0
            \end{bmatrix}$
        $(1,0) \mapsto \begin{bmatrix}
                1 & 1 & 0 & 0 \\
                0 & 0 & 0 & 0 \\
                0 & 0 & 1 & 1 \\
                0 & 0 & 0 & 0
            \end{bmatrix}$
        $(1,1) \mapsto \begin{bmatrix}
                1 & 0 & 0 & 0 \\
                0 & 1 & 0 & 0 \\
                0 & 0 & 1 & 0 \\
                0 & 0 & 0 & 1
            \end{bmatrix}$
    \end{center}
    \vspace*{1em}
    \begin{center}
        \begin{tabular}{c|c|c|c|c}
            $\cdot$ & $(0,0)$ & $(0,1)$ & $(1,0)$ & $(1,1)$ \\ \hline
            $(0,0)$ & $(0,0)$ & $(0,0)$ & $(0,0)$ & $(0,0)$ \\ \hline
            $(0,1)$ & $(0,0)$ & $(0,1)$ & $(0,0)$ & $(0,1)$ \\ \hline
            $(1,0)$ & $(0,0)$ & $(0,0)$ & $(1,0)$ & $(1,0)$ \\ \hline
            $(1,1)$ & $(0,0)$ & $(0,1)$ & $(1,0)$ & $(1,1)$ \\
        \end{tabular}
    \end{center}
    \vspace*{1em}
    Si può verificare direttamente che le matrici hanno la stessa tabella moltiplicativa.
\end{example}

Abbiamo quindi visto che un monoide finito di cardinalità $n$ è isomorfo a un monoide di matrici $n \times n$ le cui colonne hanno un unico "$1$" e altrove sono "$0$".

Ognuna di queste matrici può essere vista come una funzione da $X = \{1,\ldots,n\} \in X$:
\begin{gather*}
    A_{ij} = 1 \iff f(j) = i\\
    A_{ij} = 0 \iff f(j) \neq i
\end{gather*}
Il prodotto righe per colonne corrisponde alla composizione di funzioni.

Quindi un monoide finito di cardinalità $n$ è isomorfo a un sottomonide del monoide delle funzioni $f$ da $\{1,\ldots,n\}$ in $\{1,\ldots,n\}$ con l'operazione di composizione.

Notiamo che un elemento $x \in M$ di un monoide finito M è invertibile se e solo se la matrice associata è invertibile (una matrice $A \in Mat_{n \times n} (\mathbb{Z})$ è invertibile se e solo se il suo determinante è invertibile su $\mathbb{Z}$, ossia se e solo se $det(a) \in \{-1,1\}$).

Da ciò segue che un gruppo finito $G$ di cardinalità $|G|=n$, è isomorfo a un gruppo di matrici le cui componenti sono "$0$" e "$1$" e che hanno un unico "$1$" in ogni riga e ogni colonna (matrici di permutazioni).

Il gruppo $G$ è inoltre isomorfo a un sottogruppo del gruppo delle funzioni biunivoche da $\{1,\ldots,n\}$ in $\{1,\ldots,n\}$, che abbiamo chiamato \textbf{gruppo simmetrico $S_n$}.

Gli elementi di $S_n$ in notazione a una linea sono indicati nel modo seguente: sia $\sigma \in S_n$ una funzione biunivoca da $\{1,\ldots,n\}$ in $\{1,\ldots,n\}$, allora $\sigma$ è indicata come $\sigma(1)\sigma(2)\ldots\sigma(n)$.

\begin{theorem}[Teorema di Cayley]
    Ogni sottogruppo finito di cardinalità $n \in \mathbb{N} \backslash \{0\}$ è isomorfo a un sottogruppo di $S_n$
\end{theorem}

\begin{example}
    \
    \begin{itemize}
        \item $S_2 = \{12,21\}, S_3 = \{123,132,213,231,312,321\}$
        \item vediamo il gruppo $(\mathbb{Z}_2,+)$ come gruppo di matrici e come gruppo di permutazioni.

              \begin{equation*}
                  (\mathbb{Z}_2, +) \simeq \left\{\begin{bmatrix}
                      1 & 0 \\
                      0 & 1
                  \end{bmatrix}, \begin{bmatrix}
                      0 & 1 \\
                      1 & 0
                  \end{bmatrix}\right\} \underbrace{\simeq}_{\text{Isomorfismo di gruppi}} \left\{12,21\right\} = S_2
              \end{equation*}
    \end{itemize}
\end{example}
\end{document}