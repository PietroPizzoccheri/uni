\documentclass[a4paper,12pt]{article}
\usepackage[utf8]{inputenc}
\usepackage[T1]{fontenc} % Use T1 encoding
\usepackage{amsthm}
\usepackage{xfrac}
\usepackage{amsmath, amssymb, amsthm, mdframed}
\usepackage[a4paper, margin=1in]{geometry}
\usepackage{tikz-cd}

% NOTA:
% Questo file .tex deve essere compilato in locale utilizzando LaTeX in locale, oppure utilizzando il Overleaf.
% Modificando il contenuto di questo file e ricompilandolo, è possibile generare un nuovo file PDF.

\title{Appunti di Logica e Algebra 2}
\author{Pietro Pizzoccheri \\ Lorenzo Bardelli \\ https://github.com/PietroPizzoccheri/uni}
\date{2024}


\sloppy % Allow LaTeX to adjust spacing to avoid overfull/underfull boxes


\newtheoremstyle{def}
    {10pt} % Space above
    {5pt} % Space below
    {\slshape} % Body font
    {} % Indent amount
    {\color[rgb]{0,0,0.8}\bfseries \slshape} % head font
    {:} % Punctuation after theorem head
    {.5em} % Space after theorem head
    {} % head spec 

\newtheoremstyle{teo}
    {10pt} % Space above
    {5pt} % Space below
    {\slshape} % Body font
    {} % Indent amount
    {\color{red}\bfseries \slshape} % head font
    {:} % Punctuation after theorem head
    {.5em} % Space after theorem head
    {} % head spec 

\newtheoremstyle{prop}
    {10pt} % Space above
    {5pt} % Space below
    {\slshape} % Body font
    {} % Indent amount
    {\color[rgb]{1,0.5,0}\bfseries} % head font
    {:} % Punctuation after theorem head
    {.5em} % Space after theorem head
    {} % head spec 


\newtheoremstyle{esempio}
    {10pt} % Space above
    {4pt} % Space below
    {} % Body font
    {} % Indent amount
    {\color[rgb]{0,0.7,0}\bfseries} % head font
    {:} % Punctuation after theorem head
    {.5em} % Space after theorem head
    {} % head spec 


\newtheoremstyle{dimostrazione}
    {10pt} % Space above
    {5pt} % Space below
    {\itshape} % Body font
    {} % Indent amount
    {\color[rgb]{0.7,0,0.7}\itshape\bfseries} % head font
    {:} % Punctuation after theorem head
    {.5em} % Space after theorem head
    {} % head spec 

    \newtheoremstyle{osservazione}
    {10pt} % Space above
    {5pt} % Space below
    {\itshape} % Body font
    {} % Indent amount
    {\color[rgb]{0.5,0.5,0}\itshape\bfseries} % head font
    {:} % Punctuation after theorem head
    {.5em} % Space after theorem head
    {} % head spec


\theoremstyle{def}
\newtheorem*{definition}{Definizione}


\theoremstyle{prop}
\newtheorem*{proposition}{Proposizione}


\theoremstyle{esempio}
\newtheorem*{example}{Esempio}


\theoremstyle{dimostrazione}
\newtheorem*{dimostrazione}{Dimostrazione}

\theoremstyle{teo}
\newtheorem*{teorema}{Teorema}

\theoremstyle{osservazione}
\newtheorem*{osservazione}{Osservazione}

\begin{document}

\maketitle

\tableofcontents

\newpage
\section{Teoria degli anelli commutativi e dei campi}


\subsection{Insiemi}

Un insieme è una collezione di oggetti, detti elementi dell'insieme.
\[
    \mathbb{N} := \{ 0,1,2,3,\cdots\} \quad \text{insieme dei numeri naturali}
\]

\[
    \mathbb{Z} := \{ \cdots,-2,-1,0,1,2,\cdots\} \quad \text{insieme degli interi}
\]

\[
    \mathbb{Q} := \left\{ \frac{a}{b} \mid a,b \in \mathbb{Z}, b \neq 0
    \right\} \quad \text{insieme dei numeri razionali}
\]

\[
    \mathbb{R} := \text{insieme dei numeri reali}
\]

\[
    \mathbb{C} := \text{insieme dei numeri complessi}
\]

\subsubsection{Operazioni tra insiemi}

\[
    \subseteq \quad \text{inclusione tra insiemi}
\]
\[
    \subsetneq \quad \text{inclusione propria tra insiemi}
\]
\[
    X \subseteq Y \quad \text{si legge "} X \text{ è sottoinsieme di }Y \text{"
        o "} X \text{ è incluso in } Y\text{"}
\]

Se \(X\) è un insieme finito, indico con \(|X|\) il numero di elementi di
\(X\), detto anche la \textbf{cardinalità di \(X\)}.

\[\varnothing : \text{Insieme vuoto e } |\varnothing| = 0\]

Siano \(X\) e \(Y\) due insiemi. L'insieme \begin{math}
    X \times Y := \{ (x,y) : x \in X , y \in Y \}
\end{math} lo chiamiamo \textbf{prodotto cartesiano} di \(X\) e \(Y\).
\vspace{\baselineskip}

Sia \begin{math}
    A \in \mathcal{P} (x)
\end{math}, dove \begin{math}
    \mathcal{P} (X) := \{ A : A \subseteq X \}
\end{math} è detto \textbf{Insieme delle parti di \(X\)}. L'insieme \begin{math}
    A^c := X \setminus A
\end{math} è detto \textbf{complementare} di \(A\)


\subsection{Funzioni}

Siano \(X\) e \(Y\) due insiemi. \textbf{Una funzione \(f\) da \(X\) a \(Y\)} è un sottoinsieme \(F \subseteq X \times Y\)
tale che:
\begin{itemize}
    \item \((x, y_1) \in F\), \((x,y_2) \in F\) \(\implies y_1 = y_2\), \(\forall x \in X\), \(y_1,y_2 \in Y\).
    \item \(x \in X \implies \exists y \in Y \text{ tale che } (x,y) \in F\)
\end{itemize}


Una funzione \(F \subseteq X \times Y\) la indichiamo con \(f : X \to Y\). E scriviamo \(f(x) = y\) se \((x,y) \in F\).


\begin{definition}
    La funzione \(Id_x : X \rightarrow X \) tale che \(Id_x (x) = x, \forall x \in X\) la chiamiamo \textbf{funzione
        identità su \(X\)}
\end{definition}

\begin{definition}
    Una funzione \(f: X \rightarrow Y\) è \textbf{iniettiva} se \(\forall x_1, x_2 \in X, f(x_1) = f(x_2)
    \implies x_1 = x_2\)
\end{definition}

\begin{definition}
    Una funzione \(f: X \rightarrow Y\) è \textbf{suriettiva} se \(Im(f) = Y\), dove \(Im(f) = \{ y \in Y :
    \exists x \in X \text{ tale che } f(x) = y \}\) è detta \textbf{immagine di \(f\)}
\end{definition}

\begin{definition}
    Una funzione \(f: X \rightarrow Y\) è \textbf{biunivoca} se è sia iniettiva che suriettiva.
\end{definition}

\subsubsection{Composizione di funzioni}
Siano \(f: X \rightarrow Y\) e \(g: Y \rightarrow Z\) due funzioni. La \textbf{composizione di \(f\) e \(g\)}
è la funzione \(g \circ f : X \rightarrow Z\) tale che \((g \circ f)(x) = g(f(x))\), \(\forall x \in X\).

\begin{definition}
    una funzione \(f: X \rightarrow Y\) è detta \textbf{invertibile} se esiste una funzione \(g: Y \rightarrow X\) tale che
    \begin{itemize}
        \item \(g \circ f = Id_X\)
        \item \(f \circ g = Id_Y\)
    \end{itemize}
    la funzione \(g\) è detta \textbf{funzione inversa di \(f\)} e la indichiamo con \(f^{-1}\).
\end{definition}

Una funzione \(f: X \rightarrow Y\) è invertibile se e solo se è biunivoca.

\subsubsection{Operazioni su insiemi}

\begin{definition}
    Una funzione \(f: X \times X \rightarrow X\) è detta \textbf{operazione su \(X\)}. Invece di \(f(x,y)\)
    scriveremo \(x \cdot y\).
\end{definition}

\begin{definition}
    Un'operazione \(\cdot\) su \(X\) è detta \textbf{associativa} se \((x \cdot y) \cdot z = x \cdot (y \cdot z)\),
    \(\forall x,y,z \in X\).
\end{definition}

\begin{definition}
    Un'operazione \(\cdot\) su \(X\) è detta \textbf{commutativa} se \(x \cdot y = y \cdot x\), \(\forall x,y \in X\).
\end{definition}

\begin{example}
    \
    \begin{itemize}
        \item \(\mathcal{P}(X) \)con l'operazione di unione \(\cup\) è associativa e commutativa, così come lo è con
              l'intersezione \( \cap \).
        \item \(A \backslash B  := A \cup B^C\) \textbf{(differenza insiemistica)} è un'operazione su
              \(\mathcal{P}(X)\).\newline non è associativa: sia \(A \neq \varnothing.\) Allora \(A \backslash (A \backslash A)
              = A \neq (A\backslash A) \backslash A = \varnothing\)\newline non è commutativa: \(A \backslash \varnothing
              = A \neq \varnothing \backslash A = \varnothing\), se \(A \neq \varnothing\)
        \item \(A \Delta B := (A \backslash B) \cup (B \backslash A)\) \textbf{(differenza simmetrica)}
              è un'operazione su \(\mathcal{P}(X)\). \newline è commutativa e anche associativa, facilmente verificabile
              coi diagrammi di Venn.
        \item Sia \(F(X) := \{f : X\rightarrow X\}\).\newline La composizione" \(\circ\)" è un'operazione su \(F(X)\).
              \newline è associativa, ma non è commutativa.
        \item \(a \circ b = \frac{a + b}{2} \) è un'operazione commutativa su \(\mathbb{Q}\), ma non associativa.
    \end{itemize}
\end{example}

\begin{definition}
    Sia \(\cdot\) un'operazione su \(X\). Un elemento \(e \in X\) tale che \(e \cdot x = x \cdot e = x\), \(\forall x \in X\) è detto \textbf{elemento neutro }o \textbf{identità}.
\end{definition}

L'identità è unica; se \(e,e' \in X\) sono due identità, allora \(e = e \cdot e' = e'\).

\subsection{Monoidi e Gruppi}

\begin{definition}
    Un insieme \(X\) con un'operazione associativa e un'identità è detto \newline \textbf{monoide}.
\end{definition}

\begin{example}
    \
    \begin{itemize}
        \item \(\mathbb{N,Z,Q,R,C}\) con l'addizione e identità 0 sono monoidi.
        \item \(\mathbb{N,Z,Q,R,C}\) con la moltiplicazione e identità 1 sono monoidi.
        \item \(\mathcal{P} (X)\) con \(\cup\)  e come identità l'insieme X è un monoide.
        \item \(\mathcal{P} (X)\) con \(\cap\)  e come identità l'insieme vuoto è un monoide.
        \item \(F(X):= \{f: X \rightarrow X\}\) con la composizione" \(\circ\) "e come identità
              la funzione identità (\(Id_X\)) è un monoide.
    \end{itemize}
\end{example}

\begin{definition}
    Sia \(X \)un monoide. Un elemento \(x \in X\) è detto \textbf{invertibile} se esiste \(y \in X\) tale che
    \(x \cdot y = y \cdot x = e\), dove \(e\) è l'identità di \(X\). L'elemento \(y\) è detto \textbf{inverso} di \(x\).
\end{definition}

Se \(x \in X\) è invertibile, il suo inverso è unico e lo indichiamo con \(x^{-1}\).\newline
L'identità del monoide è invertibile e il suo inverso è l'identità stessa.

\begin{example}
    \
    \begin{itemize}
        \item L'insieme degli elementi invertibili di \((\mathbb{N},+)\) è \(\{0\}\).
        \item Linsieme degli elementi invertibili di \((\mathbb{Z},+)\) è \(\mathbb{Z}\), di \((\mathbb{Q},+)\)
              è \(\mathbb{Q}\), di \((\mathbb{R},+)\) è \(\mathbb{R}\), di \((\mathbb{C},+)\) è \(\mathbb{C}\).
        \item L'insieme degli elementi invertibili di \((\mathbb{N},\cdot)\) è \(\{1\}\), di \((\mathbb{Z},\cdot)\)
              è \(\{1,-1\}\), di \((\mathbb{Q},\cdot)\) è \(\mathbb{Q} \setminus \{0\}\), di \((\mathbb{R},\cdot)\) è
              \(\mathbb{R} \setminus \{0\}\), di \((\mathbb{C},\cdot)\) è \(\mathbb{C} \setminus \{0\}\).
        \item L'insieme degli elementi invertibili di \(F(X) = \{f: X \rightarrow X\}\) è l'insieme delle funzioni
              invertibili.
    \end{itemize}
\end{example}

\begin{definition}
    Un monoide \(X\) è detto \textbf{gruppo} se ogni suo elemento è invertibile. Se l'operazione è
    commutativa, il gruppo è detto \textbf{gruppo abeliano}.
\end{definition}

\begin{example}
    \
    \begin{itemize}
        \item (\(\mathcal{P}(x) ,\Delta \)) è un gruppo abeliano. L'identità è l'insieme vuoto e l'inverso
              di \(A \in \mathcal{P}(x)\) è \(A\) stesso. (\(A^2 = \varnothing,  \forall A \subseteq X \))
        \item (\(\mathbb{Z},+\)), (\(\mathbb{Q},+\)), (\(\mathbb{R},+\)), (\(\mathbb{C},+\)) sono gruppi abeliani
        \item (\(\mathbb{Q}\backslash \{0\}, \centerdot\)), (\(\mathbb{R}\backslash \{0\}, \centerdot\)),
              (\(\mathbb{C}\backslash \{0\}, \centerdot\)) sono gruppi abeliani
        \item sia \(X= \{1,2,\cdots,n\}\) l'insieme delle funzioni invertibili \(f:X \rightarrow X\) è il
              \textbf{Gruppo delle permutazioni di n elementi (o gruppo simmetrico)}.Lo indiciamo con \(S?n\).
              \(|S_n| = m!\). Non è abeliano se \(n \geq 3\).
    \end{itemize}
\end{example}

\begin{definition}
    Sia \(X\) un monoide con identità \(e\). Un sottoinsieme \(Y \subseteq X\) tale che \(e \in Y\) e \(Y\)
    è chiuso rispetto all'operazione di \(X\) è detto \textbf{sottomonide di \(X\)}. Analogamente definiamo
    la nozione di \textbf{sottogruppo di \(X\)}. il gruppo \{\(e\)\} è detto \textbf{sottogruppo banale di \(X\)}.
\end{definition}

\begin{example}
    \
    \begin{itemize}
        \item Con l'addizione, \(\{0\}\) èun sottomonoide di \(\mathbb{N}\). \(\{0\}\) è anche sottogruppo banale.
        \item Con la moltiplicazione abbiamo la catena di sottomonoidi \(\{1\} \subseteq \mathbb{N} \subseteq
              \mathbb{Z} \subseteq \mathbb{Q} \subseteq insieme R \subseteq  \mathbb{C}\) e di sottogruppi \(\{1\}
              \subseteq \mathbb{Q}\backslash \{0\} \subseteq \mathbb{R} \backslash \{0\} \subseteq \mathbb{C}
              \backslash \{0\}\)
        \item con l'addizione abbiamo la caten di sottogruppi \(\{0\} \subseteq \mathbb{Z} \subseteq \mathbb{Q}
              \subseteq \mathbb{R} \subseteq \mathbb{C}\)
    \end{itemize}
\end{example}

\begin{definition}
    Sia \(X\) un monoide e \(S \subseteq X\) un sottoinsieme. L'insieme \(\langle S \rangle := \{ x_1
    \cdot x_2 \cdot \cdots \cdot x_n : n \in \mathbb{N}, x_1,x_2,\cdots,x_n \in S \}\) è detto
    \textbf{sottomonoide generato da \(S\)} (intersezione di utti i sottomonoidi di \(X\) che contengono
    \(S\)). Se \(X\) è un gruppo, \(\langle S \rangle\) è detto \textbf{sottogruppo generato da \(S\)}.
\end{definition}

\begin{example}
    \
    \begin{itemize}
        \item \(S = \{1\} \subseteq  (\mathbb{N}, +)\). Allora \(\langle S \rangle = \{0,1,2,\cdots\} = \mathbb{N}\)
        \item sia \(S:= \{p \in \mathbb{N} : p \text{ è primo}\} \cup \{0\} \subseteq (\mathbb{N}, \cdot)\).
              allora \(\langle S \rangle = \mathbb{N}\)
        \item \(S = \{0,1\} \subseteq (\mathbb{N} , \centerdot)\). Allora \(\langle S \rangle = \{0,1\}\)
        \item sia \(S = \{1\} \subseteq  (\mathbb{Z}, +)\). il sottogruppo generato da \(S\) è \(\langle S \rangle
              = \mathbb{Z}\)
        \item uno spazio ettoriale \(V\) è un gruppo abeliano se consideriamo l'operazione di addizione fra vettori.
              Prendiamo \(V=\mathbb{R}^2 = \mathbb{R} \times \mathbb{R}\). Sia \(v = (1,1) \in \mathbb{R}^2\).
              Il sottogruppo \(\langle \{v\} \rangle = \{(n,n): n \in \mathbb{Z}\}\) è un sottogruppo proprio del
              sottospazio generato da \(\{v\}\). Sia \(v_1 = (1,0)\) ed \(v_2 = (0,1)\), allora il sottogruppo
              \(\langle \{v_1,v_2\} \rangle\) è \(\mathbb{Z} \times \mathbb{Z} \subseteq  \mathbb{R} \times \mathbb{R}\)
    \end{itemize}
\end{example}


\begin{definition}
    Siano \(M_1 , M_2\) con identità \(e_1 , e_2\) rispettivamente. Si definisce prodotto diretto di \(M_1\)
    e \(M_2\) l'insieme \(M_1 \times M_2\) con l'operazione \((m_1,m_2) \cdot (m_1',m_2') = (m_1 \cdot m_1',
    m_2 \cdot m_2')\) e identità \((e_1,e_2)\). Analogamente si definisce prodotto diretto di
    gruppi \(G_1 \text{e} G_2\).
\end{definition}

L'inverso di una coppia \((a,b) \in G_1 \times G_2\) è \((a^{-1},b^{-1})\).

\subsection{Morfismi}

\begin{definition}
    Siano \(M_1 e M_2\) monoidi con identità \(e_1 e e_2\). Una funzione \(f : M_1 \rightarrow M_2\) è un
    \textbf{morfismo di monoidi se:}
    \
    \begin{itemize}
        \item \(f(e_1) = e_2\)
        \item \(f(xy) = f(x)f(y)\)
    \end{itemize}
\end{definition}

\begin{definition}
    Siano \(G_1 e G_2\) gruppi con identità \(e_1 e e_2\). Una funzione \(f : G_1 \rightarrow G_2\) è un
    \textbf{morfismo di gruppi se:}
    \
    \begin{itemize}
        \item \(f(e_1) = e_2\)
        \item \(f(xy) = f(x)f(y)\)
    \end{itemize}
\end{definition}


\begin{definition}
    Il \textbf{nucleo} di un morfismo di monoidi \(f: M_1 \rightarrow M_2\) è il sottomonoide di \(M_1\)
    definito come: \(Ker(f):= \{x \in M_1 : f(x) = e_2\}\)
\end{definition}

\begin{definition}
    Il nucleo di un morfismo di gruppi \(f: G_1 \rightarrow G_2\) è il sottogruppo di \(G_1\) definito
    come: \(Ker(f):= \{x \in G_1 : f(x) = e_2\}\). \textbf{Il nucleo è un sottogruppo di \(G_1\)}. e
    \textbf{\(Im(f)\) è un sottogruppo di \(G_2\)}.
\end{definition}

\begin{definition}
    Un \textbf{isomorfismo di monoidi (e di gruppi)} èun morfismo biunivoco, tale che la funzione inversa sia un morfismo.
\end{definition}


\begin{proposition}
    Sia \(f: M_1 \rightarrow M_2\) un morfismo di monoidi. Se \(f\) è biunivoco, allora è un isomorfismo.
    Questo vale anche per i gruppi.
\end{proposition}


\begin{dimostrazione}
    \
    Dobbiamo far vedere che la funzione inversa \(f^{-1} : M_2 \rightarrow M_2\) è un morfismo di monoidi.
    Poiché \(f(e_1) = e_2\), allora \(f^{-1}(e_2) = e_1\). Siano \(x_2,y_2 \in M_2\), allora esistono
    \(x_1,y_1 \in M_1\) tali che \(f(x_1)=x_2 , f(y_1)=y_2\). Quindi \(f^{-1} (f(x_1)f(y_1)) = f^{-1}(f(x_1 y_1))
    = x_1 y_1 = f^{-1}(x_2) f^{-1}(y_2)\)
\end{dimostrazione}

\begin{example}
    \
    \begin{itemize}
        \item Siano \(M_1 = (\mathcal{P}(X), \cup) \text{ e } M_2 = (\mathcal{P}(X), \cup), \text{ dove } X\)
              è un insieme. Sia \(f : M_1 \rightarrow M_2\) definita ponendo \(f(A) = A^C , \forall A \subseteq X\).
              la funzione \(f\) è biunivoca. Inotre, dalle formule di De Morgan segue che \(f(A \cap B) =
              (A \cap B)^C = A^C \cup B^C = f(A) \cup f(B)\). Quindi \(f\) è un isomorfismo di monoidi, poiché
              \(f(X) = X^C = \varnothing\), essendo \(X\) l'identità di \(M_1\) e \(\varnothing\) l'identità di \(M_2\).
        \item Sia \(\mathbb{Z}_2 := \{0,1\}\) con l'operazione definita come: \(0+0=0, 0+1=1+0=1, 1+1=0\).
              Sia \(X := \{1,2,\cdots,n\}, n \in \mathbb{N}\). La funzione \(f: \mathcal{P}(X) \rightarrow
              \mathbb{Z}_2 \times \cdots \times \mathbb{Z}_2\) (n volte) definita da: \(f(A) = (a_1,a_2,\cdots,a_n)\),
              dove \(a_i = 1\) se \(i \in A\) e \(a_i = 0\) se \(i \notin A\). \newline è un isomorfismo del gruppo
              \((\mathcal{P}(X), \Delta)\) con il gruppo \(\mathcal{P}(X) \rightarrow \mathbb{Z}_2 \times \cdots
              \times \mathbb{Z}_2 = (\mathbb{Z}_2)^n\)
    \end{itemize}
\end{example}

\textbf{Vediamo ora come ogni monoide finito è isomorfo a un monoide di matrici quadrate, dove l'operazione
    è il prodotto righe per colonne.}\newline
Sia \(M=\{x_1,\cdots,x_n\}\) un monoide, \(|M|=n \in \mathbb{N}\), con identità \(e = x_1\). Per ogni \(x \in
M\) definiamo una matrice \(A(x) \in Mat_{n \times n}(\mathbb{Z})\) nel seguente modo: \(A(x)_{ij} = 1\)
se \(x_i \cdot x = x_j\) e \(A(x)_{ij} = 0\) altrimenti. La funzione \(F : M \rightarrow Mat_{n \times n}
(\mathbb{Z})\) (\(x \mapsto A(x)\)) è iniettiva.\newline
Infatti, se \(A(x) = a(y)\), allora \(A(x)_{i1} = A(y)_{i1} \text{ , } \forall i \in \{1,\cdots,n\}\).\newline
Quindi se \(A(x)_{i1} = A(y)_{i1} = 1\), allora \(xx_1 = xe = x = yx_1 = y\).\newline
Risulta inoltre facile vedere che \(A(xy) = A(x)A(y)\) (prodotto righe per colonne), ossia che \(F\) è
un morfismo di monoidi (\(Mat_{n \times n}(\mathbb{Z})\) è un monoide con l'operazione di
prodotto righe per colonne, la cui identità è la matrice \(I_n\)).\newline
Quindi \(F: M \rightarrow Im(F)\) è un isomorfismo di monoidi.

\newpage
\begin{example}
    Sia \(M = (\mathbb{Z}_2, \cdot)\) il monoide definito da:
    \begin{table}[htbp]
        \centering
        \begin{tabular}{|c|c|c|}
            \hline
            \(\cdot\) & 0 & 1 \\ \hline
            0         & 0 & 0 \\ \hline
            1         & 0 & 1 \\ \hline
        \end{tabular}
    \end{table}
    \newline costruiamo un sottomonoide di \(Mat_{4 \times 4}(\mathbb{Z})\) isomorfo a \(M \times M = \{(0,0), (0,1),
    (1,0),(1,1)\}\).\newline
    \newline\((0,0) \mapsto
    \begin{bmatrix}
        1 & 1 & 1 & 1 \\
        0 & 0 & 0 & 0 \\
        0 & 0 & 0 & 0 \\
        0 & 0 & 0 & 0
    \end{bmatrix}\),
    \((0,1) \mapsto \begin{bmatrix}
        1 & 0 & 1 & 0 \\
        0 & 1 & 0 & 1 \\
        0 & 0 & 0 & 0 \\
        0 & 0 & 0 & 0
    \end{bmatrix} \),
    \((1,0) \mapsto \begin{bmatrix}
        1 & 1 & 0 & 0 \\
        0 & 0 & 0 & 0 \\
        0 & 0 & 1 & 1 \\
        0 & 0 & 0 & 0
    \end{bmatrix} \),
    \((1,1) \mapsto \begin{bmatrix}
        1 & 0 & 0 & 0 \\
        0 & 1 & 0 & 0 \\
        0 & 0 & 1 & 0 \\
        0 & 0 & 0 & 1
    \end{bmatrix} \).\newline

    \begin{table}[htbp]
        \centering
        \begin{tabular}{|c|c|c|c|c|}
            \hline
            \(\cdot\) & \((0,0)\) & \((0,1)\) & \((1,0)\) & \((1,1)\) \\ \hline
            \((0,0)\) & \((0,0)\) & \((0,0)\) & \((0,0)\) & \((0,0)\) \\ \hline
            \((0,1)\) & \((0,0)\) & \((0,1)\) & \((0,0)\) & \((0,1)\) \\ \hline
            \((1,0)\) & \((0,0)\) & \((0,0)\) & \((1,0)\) & \((1,0)\) \\ \hline
            \((1,1)\) & \((0,0)\) & \((0,1)\) & \((1,0)\) & \((1,1)\) \\ \hline
        \end{tabular}
    \end{table}
    Si può verificare direttamnete che le matrici hanno la stessa tabella moltiplicativa. (fine esempio)
\end{example}

Abbiamo quindi visto che un monoide finito di cardinalità \(n\) è isomorfo a un monoide di matrici \(n \times n\)
le cui colonne hanno un unico \("1"\) e altrove sono \("0"\).\newline
Ognuna di queste matrici può essere vista come una funzione da \(X = \{1,\cdots,n\} in X\) :\newline
\begin{center}
    \(A_{ij} = 1 \Leftrightarrow f(j) = i\)
\end{center}
\begin{center}
    \(A_{ij} = 0 \Leftrightarrow f(j) \neq i\)
\end{center}
Il prodotto righe per colonne corrisponde alla composizione di funzioni.\newline
Quindi un monoide finito di cardinalità \(n\) è isomorfo a un sottomonide del monoide delle funzioni
\(f\) da \(\{1,\cdots,n\}\) in \(\{1,\cdots,n\}\) con l'operazione di composizione.

Notiamo che un elemento \(x \in M\) di un monoide finito M è invertibile se e solo se la matrice associata è
invertibile (una matrice \(A \in Mat_{n \times n} (\mathbb{Z})\) è invertibile se e solo se il suo determinante è
invertibile su \(\mathbb{Z}\), ossia se e solo se \(det(a) \in \{-1,1\}\)).

Da ciò segue che un gruppo finito \(G\) di cardinalità \(|G|=n\), è isomorfo a un gruppo di matrici le cui componenti
sono\("0"\) e \("1"\) e che hanno un unico \("1"\) in ogni riga e ogni colonna (matrici di permutazioni). \newline
Il gruppo \(G\) è inoltre isomorfo a un sottogruppo del gruppo delle funzioni biunivoche da \(\{1,\cdots,n\}\) in \(\{1,\cdots,n\}\),
che abbiamo chiamato \textbf{gruppo simmetrico \(S_n\)}. \newline
Gli elementi di \(S_n\) in notazione a una linea sono indicati nel modo seguente: sia
\(\sigma \in S_n\) una funzione biunivoca da \(\{1,\cdots,n\}\) in \(\{1,\cdots,n\}\), allora \(\sigma\) è indicata come
\(\sigma(1)\sigma(2)\cdots\sigma(n)\).


\begin{teorema}[Teorema di Cayley]
    Ogni sottogruppo finito di cardinalità \(n \in \mathbb{N} \backslash \{0\}\) è isomorfo a un sottogruppo di \(S_n\)
\end{teorema}

\begin{example}
    \
    \begin{itemize}
        \item \(S_2 = \{12,21\}\)\newline
              \(S_3 = \{123,132,213,231,312,321\}\)\newline

        \item vediamo il gruppo \((\mathbb{Z}_2,+)\) come gruppo di matrici e come gruppo di permutazioni. \((\mathbb{Z}_2, +) \simeq \{\begin{bmatrix}
                  1 & 0 \\
                  0 & 1
              \end{bmatrix}, \begin{bmatrix}
                  0 & 1 \\
                  1 & 0
              \end{bmatrix}\} \simeq \{12,21\} = S_2\) (\(\simeq\) : isomorfismo di gruppi)
    \end{itemize}
\end{example}

\subsection{Relazioni}

\begin{definition}
    Sia \(X\) un insieme. Un sottoinsieme \(R \subseteq X \times X\) è detto \textbf{relazione su \(X\)}.
\end{definition}

\begin{definition}
    Una relazione \(R \subseteq X \times X\) è detta \textbf{relazione di equivalenza} se soddisfa le seguenti proprietà: \
    \begin{itemize}
        \item \textbf{riflessità}: \((x,x) \in R\), \(\forall x \in X\)
        \item \textbf{simmetria}: \((x,y) \in R \implies (y,x) \in R\), \(\forall x,y \in X\)
        \item \textbf{transitività}: \((x,y) \in R \text{ e } (y,z) \in R \implies (x,z) \in R\), \(\forall x,y,z \in X\)
    \end{itemize}
\end{definition}

Se \(R\) è una relazione di equivalenza su \(X\) e \((x,y) \in R\), scriviamo \(x \sim y\),
che si legge "\(x\) è equivalente a \(y\)".

\begin{definition}
    Sia \(X\) un insieme e \(R \subseteq X \times X\) una relazione di equivalenza su \(X\). L'insieme
    \begin{math}
        [x]_R := \{y \in X : x \sim y\}
    \end{math} è detto \textbf{classe di equivalenza di \(x\) rispetto a \(R\)}.
\end{definition}

\begin{definition}
    L'insieme \(\sfrac{X}{\sim} := \{[x] : x \in X\}\) è detto \textbf{insieme quoziente}.
\end{definition}

\begin{definition}
    La funzione \begin{math}
        \pi : X \rightarrow \sfrac{X}{\sim} \text{ , } x \mapsto [x] \end{math} è detta \textbf{proiezione canonica}.
\end{definition}

\begin{definition}
    Siano \(x,y \in X\). Allora se \(x \sim y\) abbiamo che \([x] = [y]\).
    Se \(x \nsim y\) abbiamo che \([x] \cap [y] = \varnothing\).
    Quindi \(X = \underset{[x] \in \sfrac{X}{\sim}}{\uplus} [x]\), ossia \(\sfrac{X}{\sim}\) è una partizione di X.
\end{definition}

\begin{example}
    \
    \begin{itemize}
        \item L'uguaglianza \("="\) è una relazione di equivalenza
              su ogni insieme \(X\).
        \item Sia \(X = \{1,2,\cdots,n\}\). Definiamo si \(\mathcal{P}(X)\)
              la seguente relazione:
              \(A \sim B \Leftrightarrow |A| = |B|, \forall A,B \subseteq X\).
              Questa è una relazione di equivalenza e \(\sfrac{\mathcal{P}(X)}{\sim}
              \equiv \{0,1,\cdots,n\}\). Se \(A \subseteq X\) è tale che
              \(|A| = k \leq n\) allora \(|[A]| = \binom{n}{k} := \frac{n!}{k!(n-k)!}\)
        \item Sia \(G\) un gruppo e \(H \subseteq G\) un sottogruppo. La relazione
              \(\sim \) su \(G\) definita da \(g_1 \sim g_2 \Leftrightarrow g_1=g_2 h\)
              per qualche \(h \in H\) è una relazione di equivalenza.\
              \begin{itemize}
                  \item \(g \sim g : g \cdot e \text{ , } \forall g \in G , e \in H\)
                  \item \(g_1 \sim g_2 \rightarrow g_2 \sim g_1 : g_1 = g_2 h \rightarrow
                        g_1 h^{-1} = g_2\) (\(h^{-1} \in H\))
                  \item \(g_1 \sim g_2 , g_2 \sim g_3 \rightarrow g_1 \sim g_3 :
                        g_1 = g_2 h, g_2 = g_3 h' \rightarrow g_1 = g_3 h h' = g_3 h''
                        , \forall g_1,g_2,g_3 \in G\)
              \end{itemize}
              In questo caso l'insieme quoziente lo indichiamo con \sfrac{G}{H}.

    \end{itemize}
\end{example}

\begin{definition}
    Il numero \(\binom{n}{k}\) è chiamato \textbf{coefficiente binomiale},
    questo perché \((x+y)^n = \sum_{k=0}^{n} \binom{n}{k} x^n y^{n-k},
    \forall x,y \in \mathbb{C} \)
\end{definition}


\subsection{Insieme quoziente per gruppi abeliani}

Se \(G\) è un gruppo abeliano, possiamo definire la seguente operazione "\(+\)"
su \sfrac{G}{H}: \([g_1] + [g_2] := [g_1 + g_2]\), vediamo che è ben definita:
se \(g_{1}' = g_1 + h_1 \text{ e } g_{2}' = g_2 + h_2\), allora \([g_{1}'] = [g_1]
\), \([g_{2}'] = [g_2]\) e \(g_{1}' + g_{2}' = g_1 + h_1 + g_2 + h_2 = g_1 + g_2 + h\),
dove \(h = h_1 + h_2 \in H\). Quindi \([g_{1}' + g_{2}'] = [g_1 + g_2]\).
L'operazione è ovviamente associativa e commutativa, perché lo è quella su \(G\).
Inoltre \([g] + [0] = [g] \text{, } \forall [g] \in \sfrac{G}{H}\) dove con \("0"\)
abbiamo indicato l'identità di \(G\). Quindi la classe \([0]\) dell'identità di \((\sfrac{G}{H} , +)
\).
Infine \([g] + [-g] = [g-g] = [0]\), dove con \(-g\) abbiamo indicato l'inverso di \(g\) in \(G\).
Quindi \(-[g] = [-g], \forall [g] \in \sfrac{G}{H}\), ossia \((\sfrac{G}{H},+)\) è un gruppo abeliano.

\begin{example}
    \
    \begin{itemize}
        \item Se \(H = \{0\} \subseteq G\), allora \(\sfrac{G}{H}\) è isomorfo a \(G\). (\(\{0\}\) gruppo banale e \(G\) gruppo abeliano)
        \item Sia \(G = (\mathbb{Z} , +) \text{ e } n \in \mathbb{N} \). Il sottoinsieme \(n\mathbb{Z} = \{nz : z \in \mathbb{Z}\}\)
              è un sottogruppo di \(\mathbb{Z}\). \
              \begin{itemize}
                  \item \(0 \mathbb{Z} = \{0\}\)
                  \item \(1 \mathbb{Z} = \{\mathbb{Z} \}\)
                  \item \(2 \mathbb{Z} = \{\cdots,-4,-2,0,2,4,\cdots\}\)
                  \item \(3 \mathbb{Z} = \{\cdots,-6,-3,0,3,6,\cdots\}\)
              \end{itemize}
              Definiamo il gruppo abeliano \(\mathbb{Z}_n := \sfrac{\mathbb{Z}}{n\mathbb{Z}}\),
              per \(\mathbb{Z}_0 = \sfrac{\mathbb{Z}}{0 \mathbb{Z}}
              = \sfrac{\mathbb{Z}}{\{0\}} = \mathbb{Z} \). \newline
              Sia \(n \geq 0 \text{ e siano } x,y \in \mathbb{Z} \).
              \
              \begin{itemize}
                  \item Allora \(x \sim y \Leftrightarrow
                        x = y+h \text{  } (h \in n \mathbb{Z})
                        \Leftrightarrow x-y = kn \text{ (per \(k \in \mathbb{Z}\)) }  \Leftrightarrow
                        \text{ il resto della divisione di \(x\) per \(n\) è uguale
                            al resto  della divisione di \(y\) per \(n\)}\).
              \end{itemize}

              I possibili resti della divisione
              per \(n\) sono \(0,1,\cdots,n-1\).\\
              Quindi \(\mathbb{Z}_n = \{[0],[1],\cdots,[n-1]\} = \{\overline{0}, \overline{1},\cdots,\overline{n - 1}\}\).
              (\(\{[0],[1],\cdots,[n-1]\}\) sono le classi di resto)
              \
              \begin{itemize}
                  \item \(\mathbb{Z}_2 = \{\overline{0},\overline{1}\}\),
                        \begin{table}[htbp]
                            \centering
                            \begin{tabular}{|c|c|c|}
                                \hline
                                +                & \(\overline{0}\) & \(\overline{1}\) \\ \hline
                                \(\overline{0}\) & \(\overline{0}\) & \(\overline{1}\) \\ \hline
                                \(\overline{1}\) & \(\overline{1}\) & \(\overline{0}\) \\ \hline
                            \end{tabular}
                        \end{table}
                        \(\overline{1} + \overline{1} = [1 + 1] = [2] = [0]\)
                  \item \(\mathbb{Z}_3 = \{\overline{0},\overline{1},\overline{2}\}\),
                        \begin{table}[htbp]
                            \centering
                            \begin{tabular}{|c|c|c|c|}
                                \hline
                                +                & \(\overline{0}\) & \(\overline{1}\) & \(\overline{2}\) \\ \hline
                                \(\overline{0}\) & \(\overline{0}\) & \(\overline{1}\) & \(\overline{2}\) \\ \hline
                                \(\overline{1}\) & \(\overline{1}\) & \(\overline{2}\) & \(\overline{0}\) \\ \hline
                                \(\overline{2}\) & \(\overline{2}\) & \(\overline{0}\) & \(\overline{1}\) \\ \hline
                            \end{tabular}
                        \end{table}
              \end{itemize}
    \end{itemize}
\end{example}


\begin{definition}
    Sia \(G\) un gruppo abeliano e \(H \subseteq G\) un sottogruppo. La proiezione canonica \(\pi : G \rightarrow \sfrac{G}{H}\)
    è un \textbf{morfismo suriettivo di gruppi}
\end{definition}

Se \(G\) è un gruppo finito e \(H \subseteq G\) è un sottogruppo, allora \([g] \in \sfrac{G}{H} \rightarrow |[g]| = |H|\).\\
Infatti \([g] = \{gh : h \in  H\}\) e \(gh_1 = gh_2 \rightarrow h_1 = h_2\).\\
Poiché le classi di quivalenza sono una partizione di G , abbiamo \(|G| = |\sfrac{G}{H}| \cdot |H|\).\\
In particolare la cardinalità o (\textbf{ordine}) di un sottogruppo di un gruppo finito divide la cardinalità del gruppo.\\

\begin{teorema}
    Sia \(f : G_1 \rightarrow G_2\) un morfismo di gruppi. Allora \(f\) è iniettivo se e solo se \(Ker(f) = \{e_1\}\).\\
    (Questo non vale per i morfismi di monoidi.)
\end{teorema}

\begin{dimostrazione}
    \
    Sia \(f\) iniettivo. Sia \(x \in Ker(f)\). Allora \(f(x) = e_2\) e quindi, poiché anche \(f(e_1) = e_2\), si ha che \(x = e_1\)
    per l'ipotesi di iniettività.\\\
    Sia \(Ker(f) = \{e_1\}\). Siano \(x,y \in G_1\) tali che \(f(x) = f(y)\).\\
    Allora \(f(x)f(y^{-1}) = e_2 \rightarrow  f(xy^{-1}) = e_2 \rightarrow  xy^{-1} \in Ker(f) \rightarrow xy^{-1} = e_1 \rightarrow
    x=y\),
\end{dimostrazione}

\begin{example}
    \
    \begin{itemize}
        \item \(G = \mathbb{Z}_4 = \{\overline{0}, \overline{1}, \overline{2}, \overline{3}\}\), \
              \begin{itemize}
                  \item \(\langle \overline{0} \rangle = {\overline{0}}\) sottogruppo banale \(\simeq \mathbb{Z}_1\)
                  \item \(\langle \overline{1} \rangle = \mathbb{Z}_4\)
                  \item \(\langle \overline{2} \rangle = \{\overline{0}, \overline{2}\} \simeq \mathbb{Z}_2\) (\(2+2=0\))
                  \item \(\langle \overline{3} \rangle = \mathbb{Z}_4\) (\(3 , 3+3=6=2, 3+2=5=1, 3+1=4=0\))
              \end{itemize}
              I sottogruppi di \(\mathbb{Z}_4\) possono averer cardinalità \(1,2,4\). L'insieme
              dei sottogruppo di \(\mathbb{Z}_4\) è \(\{\{\overline{0}\}, \{\overline{0}, \overline{2}\},
              \{\overline{0}, \overline{1}, \overline{2}, \overline{3}\}= \mathbb{Z}_4\}\)
        \item \(G = \mathbb{Z}_6 = \{\overline{0}, \overline{1}, \overline{2}, \overline{3}, \overline{4}, \overline{5}\}\), \
              \begin{itemize}
                  \item \(\langle \overline{0} \rangle = {\overline{0}}\) sottogruppo banale \(\simeq \mathbb{Z}_1\)
                  \item \(\langle \overline{1} \rangle = \mathbb{Z}_6\)
                  \item \(\langle \overline{2} \rangle = \{\overline{0}, \overline{2}, \overline{4}\} \simeq \mathbb{Z}_3\)
                  \item \(\langle \overline{3} \rangle = \{\overline{0}, \overline{3}\} \simeq \mathbb{Z}_2\)
                  \item \(\langle \overline{4} \rangle = \{\overline{0}, \overline{2}, \overline{4}\} \simeq \mathbb{Z}_3\)
                  \item \(\langle \overline{5} \rangle = \mathbb{Z}_6\)
              \end{itemize}
              I sottogruppi di \(\mathbb{Z}_6\) possono averer cardinalità \(1,2,3,6\). L'insieme
              dei sottogruppo di \(\mathbb{Z}_6\) è \(\{\{\overline{0}\}, \{\overline{0}, \overline{2}, \overline{4}\},
              \{\overline{0}, \overline{3}\}, \{\overline{0}, \overline{1}, \overline{2}, \overline{3}, \overline{4}, \overline{5}\}= \mathbb{Z}_6\}\)
    \end{itemize}
\end{example}

\textbf{Caso generale:} consideriamo il gruppo \(\mathbb{Z}_n = (\{\overline{0},\overline{1},\cdots,\overline{n-1}\}
,+) \)  sia \(m \in \mathbb{N}, m < n\).\\
Se \(m=0\), \(\langle \overline{0} \rangle = \{\overline{0}\}\).\\
Sia \(m > 0 \) e \(z := \frac{mcm \{m,n\}}{m}\). (mcm = minimo comune multiplo)

\(\overline{m}+\overline{m}+\cdots=\overline{m} = \overline{zm} = \overline{mcm \{m,n\}} = \overline{0}\)

Se \(i \leq i \leq z\): \(im < zm = mcm \{m,n\} \rightarrow n \text{ non divide } im\).

\(\overline{m}+\overline{m}+\cdots=\overline{m} = \overline{im} \neq \overline{0}\) perché \(im\) è multiplo di \(m\)
e \(im < mcm \{m,n\}\), quindi \(im\) non è multiplo di \(n\). Dunque \(|\langle \overline{m} \rangle| =
z= \frac{mcmc \{m,n\}}{m}\).

In particolare, \(\langle \overline{m} \rangle = \mathbb{Z}_n \Leftrightarrow z=n \Leftrightarrow MCD \{m,n\}=1\). Ossia
\textbf{l'insieme \(\{\overline{m}\}\) genera il gruppo \(\mathbb{Z}_n\) sse \(m \text{ e } n\) sono coprimi.}

\begin{definition}
    La funzione definita da \(\varphi : \mathbb{N} \backslash \{0\} \rightarrow \mathbb{N} \backslash \{0\}\) ,

    \(\varphi(n) := |\{m \in \mathbb{N} \backslash \{0\} : m < n \text{ e } MCD \{m,n\} = 1\}|\)
    è detta \textbf{funzione di Eulero}.

    Quindi ci sono \(\varphi(n)\) elementi \(\overline{m}\) tali che \(\langle \overline{m} \rangle = \mathbb{Z}_n\).
\end{definition}

\begin{proposition}
    L'insieme dei sottogruppi di \((\mathbb{Z} ,+)\) è \(\{n \mathbb{Z} : n \in \mathbb{N} \} \).
\end{proposition}


\begin{dimostrazione}
    \
    Sia \(H \subseteq \mathbb{Z} \) un sottogruppo non banale.\\
    Sia \(k := min (H_{>0})\) dove \(H_{>0} := \{h \in H : h > 0\}\).\\
    Sia \(h \in H_{>0}, h \neq k\).\\
    Allora \(h > k\) e \(h = nk + r\), \(n \in \mathbb{N} , 0 \leq r <k\).\\
    Dunque \(r = h - nk \in H \rightarrow r =0\) per la minimalità di \(k\).\\
\end{dimostrazione}

\begin{definition}
    Un gruppo \(G\) è detto \textbf{ciclico} se esiste \(g \in G\) tale che \(\langle g \rangle = G\).\\
    Un gruppo ciclico è anche abeliano
\end{definition}

\begin{example}
    \
    \begin{itemize}
        \item \(\mathbb{Z} = \langle 1 \rangle\) è ciclico
        \item \(\mathbb{Z}_n = \langle \overline{1} \rangle\) è ciclico
        \item \(\mathbb{Z} \times \mathbb{Z} = \langle (1,0), (0,1) \rangle\) non è ciclico,
              infatti in \(\mathbb{Z}  \times \mathbb{Z} \), se \((a,b) \in \mathbb{Z} \times \mathbb{Z} \),
              \(\langle (a,b) \rangle = \{(ka,kb) : k \in \mathbb{Z} \} = \{(x,y) : a \text{ divide } x,b \text{ divide } y\}
              \subsetneq \mathbb{Z} \times \mathbb{Z} \).
        \item \(\mathbb{Z}_2 \times \mathbb{Z}_2\) non è ciclico. Infatti, in \(\mathbb{Z}_2 \times \mathbb{Z}_2\) si ha:
              \
              \begin{itemize}
                  \item \(\langle (\overline{0},\overline{0}) \rangle = \{(\overline{0},\overline{0})\}\)
                  \item \(\langle (\overline{0}, \overline{1}) \rangle = \{\overline{0}\} \times \mathbb{Z}_2\)
                  \item  \(\langle (\overline{1}, \overline{0}) \rangle = \mathbb{Z}_2 \times \{\overline{0}\}\)
                  \item \(\langle (\overline{1}, \overline{1}) \rangle = \{(\overline{0},\overline{0}),(\overline{1},\overline{1})\}\)
              \end{itemize}
              Quindi nessun elemento di \(\mathbb{Z}_2 \times \mathbb{Z}_2\) genera \(\mathbb{Z}_2 \times \mathbb{Z}_2\).
    \end{itemize}
\end{example}

\begin{teorema}[di isomorfismo per gruppi abeliani]
    Sia \(f: G_1 \rightarrow G_2\) un morfismo di gruppi abeliani. Allora esiste un morfismo iniettivo
    \(\varphi : \sfrac{G_1}{Ker \varphi} \rightarrow G_2\) tale che il seguente diagramma è commutativo:
    \[
        \begin{tikzcd}
            G_1 \arrow[r, "f"] \arrow[d, "\pi"'] & G_2 \\
            \sfrac{G_1}{Ker(f)} \arrow[ru, "\varphi"']
        \end{tikzcd}
    \]
    In particolare, \(\sfrac{G_1}{Ker(f)} \simeq \Im(f)\).
\end{teorema}

\begin{dimostrazione}
    \
    L'assegnazione \([g] \mapsto f(g), \forall g \in  G\), definisce una funzione \(\varphi : \sfrac{G_1}{Ker(f)}
    \rightarrow G_2\).\\
    Infatti, se \(g' \sim g\), ossia \([g]  = [g']\), allora \(g = g' + h , h \in Ker(f)\).\\
    Dunque \(f(g) = f(g' + h) = f(g') + f(h) = f(g')\). Poiché \(f\) è morfismo di gruppi, anche \(\varphi\) lo è.\\
    Inoltre \(Ker(f) = \{[g] \in \sfrac{G}{Ker(f)} : \varphi([g]) = O_2\}=\{[g] \in  \sfrac{G}{Ker(f)} : f(g) = O_2\}
    = {[O_1]}\). Quindi \(\varphi\) è iniettiva.\\
    Infine, \( \varphi : \sfrac{G_1}{Ker(f)} \rightarrow Im(f)\) è un morfismo di gruppi, iniettivo e suriettivo, quindi un isomorfismo.
\end{dimostrazione}


\begin{teorema}
    Sia \(G\) un gruppo ciclico. Allora ogni sottogruppo di \(G\) è ciclico.
\end{teorema}

\begin{dimostrazione}
    \
    Sia \(g \in G\) tale che \(g = \langle g \rangle\). La funzione \(\varphi: (\mathbb{Z} , +) \rightarrow G\) definita
    da \(\varphi(g) = g^n, \forall  n \in \mathbb{Z}, \) è un morfismo suriettivo di gruppi.\\
    \
    \begin{itemize}
        \item G è infinito: allora \(Ker(f) = \{0\}\) e quindi \(\varphi\) è iniettivo. Dunque \(\varphi\) è un
              isomorfismo di gruppi. Tutti i sottogruppi di \(\mathbb{Z} \) sono ciclici.
        \item G è finito: sia \(H \subseteq G\) un sottogruppo. Allora \(\varphi^{-1}(H) := \{n \in \mathbb{Z} :
              \varphi(n) \in H\} \subseteq \mathbb{Z} \) è un sottogruppo di \(\mathbb{Z} \), quindi esiste \(
              \varphi^{-1}(H)= \langle k \rangle\) con \(k \in \mathbb{N} \).\\
              La restrizione \(\varphi: k \mathbb{Z} \rightarrow H\) è un morfismo suriettivo di gruppi e
              \(\varphi(hk) = \varphi(\underbrace{k+k+\cdots+k}_{h \text{ volte}}) = \varphi(k) \varphi(k) \cdots
              \varphi(k) = [\varphi(k)]^h , \forall h \in \mathbb{Z} \). Quindi \(H = \langle \varphi(k) \rangle\).
    \end{itemize}
\end{dimostrazione}

\textbf{Corollario:} L'insieme dei sottogruppi di \(\mathbb{Z}_n , n \in \mathbb{N} \) è \(\{\langle \overline{m} \rangle
: \overline{m} \in \mathbb{Z}_n \}\).

\begin{proposition}
    Sia \(n \in \mathbb{N} \text{ e sia } d/n\) (d divide n). Allora esiste al più un unico sottogruppo di \(\mathbb{Z}_n\)
    di cardinalità \(d\).
\end{proposition}

\begin{dimostrazione}
    \
    Sia \(H \subseteq \mathbb{Z}_n\) sottogruppo tale che \(|H| = d\). Si considerino le proiezioni canoniche
    \(\mathbb{Z} \rightarrow^{\pi_1} \mathbb{Z}_n \rightarrow^{\pi_2} \sfrac{\mathbb{Z}_n }{H}\).\\
    Poiché \(\pi^{-1}_1 (H) = \{m \in \mathbb{Z} : \pi_1 (m) \in H\}\) è un sottogruppo di \(\mathbb{Z} \), allora esiste
    \(k \in \mathbb{N} \) tale che \(\pi^{-1}_1(H) = k \mathbb{Z}\). Inoltre \(Ker(\pi_1 \cdot \pi_2) = \pi^{-1}_1
    (H)\) e quindi, essendo \(\pi_1 \cdot \pi_2\) un morfismo suriettivo di gruppi, \(\sfrac{\mathbb{Z}_n }{H}
    \simeq \sfrac{\mathbb{Z} }{\pi^{-1}(H)} = \sfrac{\mathbb{Z} }{k \mathbb{Z} }= \mathbb{Z}_k\).\\
    Quindi \(|\mathbb{Z}_k | = k =|\sfrac{\mathbb{Z}_n }{H}| = \sfrac{|\mathbb{Z}_n |}{|H|} = \frac{n}{d}\), ossia \(k\)
    è univocamente determinato, e allora \(H = \pi_1 (k \mathbb{Z} )\) è univocamente determinato.
\end{dimostrazione}

\begin{example}
    I sottogruppi di \(\mathbb{Z}_{899}\) sono quattro, perché \(899 = 31 \cdot 29\), quindi c'è un sottogruppo di
    cardinalità 1 (il sottogruppo banale), uno di cardinalità 31, uno di cardinalità 29 e \(\mathbb{Z}_{899}\).\\
    Sono: \(\{\{0\} , \langle \overline{29} \rangle , \langle \overline{31} \rangle , \mathbb{Z}_{899}\}\).

\end{example}

\subsection{Anelli}

\begin{definition}
    Sia \(X\) un insieme su cui sono definite due operazioni \(+\) e \(\cdot \). \\
    \(X\) è un \textbf{anello} con unità \(1_X\) se:
    \
    \begin{itemize}
        \item \((X, +)\) è un gruppo abeliano
        \item \((X, \cdot)\) è un monoide con unità \(1_X\)
        \item vale la proprietà distributiva: \
              \begin{itemize}
                  \item \(a \cdot (b+c) = a \cdot b + a \cdot c\)
                  \item \((a+b) \cdot c = a \cdot c + b \cdot c\) , \(\forall a,b,c \in X\)
              \end{itemize}
    \end{itemize}
\end{definition}

\begin{definition}
    Diaciamo che un anello \(X\) è \textbf{commutativo} se il monoide \((X, \cdot)\) è commutativo.
\end{definition}


Indichiamo con \("0"\) l'identità del gruppo \((X, +)\).

\begin{example}
    \
    \begin{itemize}
        \item Gli insiemi \(\mathbb{Z} ,\mathbb{Q} ,\mathbb{R} ,\mathbb{C} \) con le operazioni di addizione e moltiplicazione
              sono anelli commutativi con unità, che è il numero "1".
        \item L'insieme delle matrici \(n \times n, n> 1\) a valori su \(\mathbb{Z} \), su \(\mathbb{Q} \), su \(\mathbb{R} \)
              o su \(\mathbb{C} \), con l'operazione di somma e il prodotto righe per colonne, è un anello \textbf{non commutativo},
              con unità la matrice identità.\\
              In generale, se \(A\) è un anello commutativo con unità, l'insieme \(Mat_{n \times n}(A)\) delle matrici a valori
              in \(\mathbb{R} \) con le operazioni di somma e prodotto righe per colonne, è un anello non commutativo con unità.
        \item\(\{X\}\) è un anello, detto \textbf{anello nullo}. Le due operazioni sono la stessa e \(0 = 1_{\{X\}} = x\).
    \end{itemize}
\end{example}

Considereremo sempre \(0 \neq 1_A\) e studieremo solo anelli commutativi con unità. Quindi quando diremo "anello" intendiamo "anello con unità".

\begin{definition}
    Sia \(A\) un anello commutativo. Un elemento \(x \in A\) è detto \textbf{zero divisore} se esiste \(y \in A \backslash \{0\}\)
    tale che \(x y = 0\).
\end{definition}

\begin{definition}
    Diciamo che un elemento \(x \in A\) è \textbf{invertibile} se è un elemento invertibile del
    monoide \((A, \cdot )\).
\end{definition}

\begin{proposition}
    Sia \(A\) un anello commutativo. Allora l'insieme degli elementi invertibili di \(A\) è disgiunto
    dall'insieme degli zero-divisori di \(A\).
\end{proposition}

\begin{dimostrazione}
    \
    Siano \(x,y \in A\) tali che \(x y = 0\). Se \(X\) è invertibile, allora \(x^{-1}xy = y = 0\),
    quindi \(x\) non è uno zero-divisore.
\end{dimostrazione}

\begin{proposition}[legge di cancellazione]
    Sia \(A\) un anello commutativo e sia \(x \in A\) un elemento che non è uno zero-divisore. Allora
    \(xy=xz \rightarrow y =z , \forall y,z \in A\).
\end{proposition}

\begin{dimostrazione}
    \
    Se \(xy = xz\) allora \(x(y - z) = 0 \). Poiché \(x\) non è uno zero-divisore, allora \(y-z = 0\), ossia \(y=z\).
\end{dimostrazione}

\begin{definition}
    Un anello commutativo privo di zero-divisori non nulli è detto \textbf{dominio di integrità}.
\end{definition}

\begin{definition}
    Un anello commutativo i cui elementi non nulli sono tutti invertibili è detto \textbf{campo}.
\end{definition}

\begin{example}
    L'anello \(\mathbb{Z}\) è un dominio di integrità, ma non è un campo.\\
    Gli anelli \(\mathbb{Q} ,\mathbb{R} ,\mathbb{C} \) sono campi.
\end{example}

\subsection{Ideali}

\begin{definition}
    Sia \(A\) un anello commutativo. Un sottoinsieme \(I \subseteq A\) è detto \textbf{ideale} di \(A\) se:
    \
    \begin{itemize}
        \item \(I\) è un sottogruppo di \((A,+)\)
        \item \(ax \in I, \forall a \in A, x \in I\)
    \end{itemize}
\end{definition}

\begin{example}
    Abbiamo già visto che ogni sottogruppo di \((\mathbb{Z} , +)\) è del tipo \(n \mathbb{Z} = \{kn : k \in \mathbb{Z} \}\),
    dove \(n \in \mathbb{N} \). Inoltre, se \(a \in \mathbb{Z} \) e \(x \in n\mathbb{Z} \), ossia
    \(x=kn\) per qualche \(k \in \mathbb{Z} \), si ha che \(ax=akn \in n \mathbb{Z} \).
    Quindi \(n \mathbb{Z} \) è un ideale di \(\mathbb{Z}, \forall n \in \mathbb{N}  \), e tutti gli ideali di
    \(\mathbb{Z} \) sono di questo tipo.
\end{example}

\textbf{Osservazioni:} Siano \(I,J \subseteq A\) ideali di un anello commutativo \(A\).\\
Allora : \
\begin{itemize}
    \item \(I \cap J\) è un ideale di \(A\)
    \item \(I + J := \{x+y : x \in I, y \in J\}\) è un ideale di \(A\)
    \item \(IJ := \langle \{xy : x \in I, y \in J\} \rangle\) è un ideale di \(A\)
\end{itemize}

\begin{definition}
    Sia \(S \subseteq A\) un sottoinsieme di un anello commutativo. \textbf{L'ideale generato da \(S\)} è
    l'intersezione di tutti gli ideali di \(A\) che contengono \(S\) e lo indichiamo con \(\langle S \rangle\).\\
    Se \(S =\{x\}\), diciamo che \(\langle S \rangle\) è \textbf{l'ideale principale generato da \(x \in A\)}.
\end{definition}

\begin{example}
    Abbiamo visto che gli ideali di \(\mathbb{Z} \) sono tutti e soli i sottoinsiemi \(n \mathbb{Z}  = \langle n \rangle, n \in \mathbb{N} \).
    Quindi gli ideali di \(\mathbb{Z} \) sono tutti principali.
\end{example}


\begin{definition}
    un anello i cui ideali sono tutti principali si dice \textbf{anello ad ideali principali}.
\end{definition}

\begin{proposition}
    Sia \(A\) un anello commutativo e \(I \subseteq A\) un ideale.\\
    Allora: \
    \begin{itemize}
        \item \(I = A\) se e solo se \(I\) contiene un elemento invertibile
        \item \(A\) è un campo sse i suoi unici ideali sono
              \(\langle 0 \rangle\) e \(A = \langle 1_A \rangle\)
    \end{itemize}
\end{proposition}

\begin{dimostrazione}
    \
    \
    \begin{itemize}
        \item se \(I = A\) allora \(1_A \in I\) e \(1_A\) è invertibile.\\
              Sia \(u \cap I\) un elemento invertibile. Allora \(u^{-1} \cap A\) e quindi
              \(1_A u u^{-1} \in I\). Ne segue che \(A = \langle 1_A \rangle \subseteq I\).
              e quindi \(I = A\).
        \item Sia \(A\) un campo e sia \(I \neq \langle 0 \rangle\).\\
              se \( n \in I\) e \(x \neq 0\) allora \(x\) è invertibile e quindi \(I = A\)
              per il punto sopra.\\
              Vicerversa, se \(\langle 0 \rangle\) e \(A\) sono gli unici ideali di \(A\),
              e se \(x \in A  \backslash \{0\}\), allora \(\langle X \rangle = \langle 1_A \rangle\),
              ossia \(ax = 1_A\) per qualche \(a \in A\). Quindi \(x\) è invertibile.

    \end{itemize}

\end{dimostrazione}

\subsection{Anelli quoziente}

Sia \(A\) un anello commutativo e \(I \subseteq A\) un ideale.\\
In particolare, \(A\) con l'operazione \("+"\) è un gruppo abeliano e \(I\) è un sottogruppo di \(A\).\\
Allora possiamo definire il gruppo quoziente \(\sfrac{A}{I}\).\\
Con l'operazione \([x] \cdot [y] := [xy]\), per ogni \([x] , [y] \in \sfrac{A}{I}\), abbiamo che
\(\sfrac{A}{I}\) è un anello commutativo con unità \([1_A]\).\\
Infatti, mostriamo che l'operazione è ben definita. Siano \(x' \in [x]\) e \(y' \in [y]\). Allora
esistono \(i_x \in I\) e \(i_y \in I\) tali che \(x' = x + i_x\) e \(y' = y + i_y\).\\
Quindi \(x'y' = (x + i_x) (y + i_y) = xy + \underbrace{x i_y + y i_x + i_x i_y}_{\in I \text{ perchè \(I\) è
    un ideale di \(A\) }}  \)\\
Quindi \([x'y'] = [xy]\).\\
Inoltre \([1_A] [x] = [1_A x] = [x] \), per ogni \([x] \in \sfrac{A}{I}\), quindi \([1_A]\)
è l'unità di \(\sfrac{A}{I}\).

\begin{example}
    Abbiamo visto che \(n \mathbb{Z} = \{kn : k \in \mathbb{Z} \}\) è un ideale
    dell'anello \(\mathbb{Z} \). Quindi il quoziente \(\mathbb{Z}_n = \sfrac{\mathbb{Z}
    }{n \mathbb{Z} }\) ha la struttura di anello.
    \
    \begin{itemize}
        \item \(\mathbb{Z}_0 \simeq \mathbb{Z} \)
        \item \(\mathbb{Z}_1 \simeq \{0\}\) anello nullo.
        \item \(\mathbb{Z}_2 \simeq \{\overline{0}, \overline{1}\}\) \begin{table}[htbp]
                  \centering
                  \begin{tabular}{ c | c | c}

                      \(\cdot\)        & \(\overline{0}\) & \(\overline{1}\) \\ \hline
                      \(\overline{0}\) & \(\overline{0}\) & \(\overline{0}\) \\ \hline
                      \(\overline{1}\) & \(\overline{0}\) & \(\overline{1}\) \\
                  \end{tabular}
              \end{table}
        \item \(\mathbb{Z}_3 \simeq \{\overline{0}, \overline{1}, \overline{2}\}\) è un campo perchè
              \(\overline{1}\) è invertibile e \(\overline{2} \cdot \overline{2} = \overline{1}\), quindi
              anche \(\overline{2}\) è invertibile. \begin{table}[htbp]
                  \centering
                  \begin{tabular}{ c | c | c | c}

                      \(\cdot\)        & \(\overline{0}\) & \(\overline{1}\) & \(\overline{2}\) \\ \hline
                      \(\overline{0}\) & \(\overline{0}\) & \(\overline{0}\) & \(\overline{0}\) \\ \hline
                      \(\overline{1}\) & \(\overline{0}\) & \(\overline{1}\) & \(\overline{2}\) \\ \hline
                      \(\overline{2}\) & \(\overline{0}\) & \(\overline{2}\) & \(\overline{1}\) \\
                  \end{tabular}
              \end{table}
        \item \(\mathbb{Z}_4 = \{\overline{0},\overline{1},\overline{2},\overline{3}\}\) dove
              \(\overline{2} \cdot \overline{2} = \overline{0}\), quindi \(\mathbb{Z}_4\) non è un dominio
              di integrità.\\In particolare non è un campo.
              \begin{table}[htbp]
                  \centering
                  \begin{tabular}{ c | c | c | c | c}

                      \(\cdot\)        & \(\overline{0}\) & \(\overline{1}\) & \(\overline{2}\) & \(\overline{3}\) \\ \hline
                      \(\overline{0}\) & \(\overline{0}\) & \(\overline{0}\) & \(\overline{0}\) & \(\overline{0}\) \\ \hline
                      \(\overline{1}\) & \(\overline{0}\) & \(\overline{1}\) & \(\overline{2}\) & \(\overline{3}\) \\ \hline
                      \(\overline{2}\) & \(\overline{0}\) & \(\overline{2}\) & \(\overline{0}\) & \(\overline{2}\) \\ \hline
                      \(\overline{3}\) & \(\overline{0}\) & \(\overline{3}\) & \(\overline{2}\) & \(\overline{1}\) \\
                  \end{tabular}
              \end{table}
    \end{itemize}
\end{example}

Vediamo che \(\mathbb{Z}_n\) è un campo se e solo se \(n \in \mathbb{N} \backslash \{0,1\}\)
è un numero primo (per \(n =0 \) abbiamo \(\mathbb{Z}_0 \simeq \mathbb{Z} \) e per \(n = 1\) abbiamo l'anello nullo).\\
Un ideale di \(\mathbb{Z}_n \) è un sottogruppo di \(\mathbb{Z}_n\).\\
Poiché \(\mathbb{Z}_n\) è ciclico, i suoi sottogruppi sono ciclici e sono \(\{\langle \overline{m} \rangle :
\overline{m} \in \mathbb{Z}_n\}\). Inoltre \(\langle \overline{m} \rangle \subseteq \mathbb{Z}_n\) è un ideale,
\(\forall \overline{m} \in \mathbb{Z}_n\). Infatti, se \(\overline{a} \in \mathbb{Z} \), allora \(
\overline{a} \overline{m} = \overline{am} = \underbrace{\overline{m} + \overline{m} + \cdots + \overline{m}}_{a \text{ volte }}  \in
\langle \overline{m} \rangle\)\\
Quindi \(\{\langle \overline{m} \rangle : \overline{m} \in \mathbb{Z}_n\}\) è l'insieme
degli ideali di \(\mathbb{Z}_n\) (\(\mathbb{Z}_n\) è anello ad ideali principali).\\
Inoltre, se \(n > 1, \{\langle \overline{m} \rangle \overline{m} \in \mathbb{Z}_n\} =
\{\{\overline{0}\}, \mathbb{Z}_n\} \cup \{\langle \overline{m} \rangle : MCD_{m \neq 0} \{m,n\} \neq 1\}\)\\
Quindi \(\mathbb{Z}_n\) è un campo se e solo se \(\{\langle \overline{m} \rangle : \overline{m} \in \mathbb{Z}_n\} =
\{\{\overline{0}\}, \mathbb{Z}_n\}\) se e solo se \(n\) è un numero primo.

\begin{example}
    \(\mathbb{Z}_3\) è un campo, si ha che \(\overline{2}^{-1} = \overline{2}\). Infatti
    \(\overline{2} \cdot \overline{2} = \overline{4} = \overline{1}\).\\Invece \(\mathbb{Z}_4\) non lo è;
    infatti \(\overline{2} \cdot \overline{2} = \overline{0}\) e quindi \(\overline{2}\) non è invertibile.
\end{example}


\subsection{Algoritmo di Euclide e identità di Bézout su \(\mathbb{Z}\)}

Vogliamo calcolare il massimo comun divisore tra 1876 e 365.\\
Usiamo l'algoritmo di Euclide: \\
\begin{center}
    \(1876 = 5 \cdot 365 + 51\)\\
    \(365 = 7 \cdot 51 + 8\)\\
    \(51 = 6 \cdot 8 + 3\)\\
    \(8 = 2 \cdot 3 + 2\)\\
    \(3 = 1 \cdot 2 + 1\)\\
    \(2 = 2 \cdot 1 + 0\)\\
\end{center}

Quindi \(MCD \{1876,365\} = 1\).\\
Adesso vogliamo trovare due numeri \(x,y \in \mathbb{Z} \) tali che \(1876x + 365y = 1\).\\
Un'identità del tipo \(ax + by = MCD \{a,b\}\) si chiama \textbf{identità di Bézout}.\\
Dall'algoritmo di Euclide abbiamo:
\begin{center}
    \(1 = 3 - 2 \cdot 1\)\\
    \(2 = 8 - 3 \cdot 2\)\\
    \(3 = 51 - 6 \cdot 8\)\\
    \(8 = 365 - 7 \cdot 51\)\\
    \(51 = 1876 - 5 \cdot 365\)\\
\end{center}
Quindi\\
\begin{center}
    \(1 = 3 - 2 =\)\\
    \(= 3 - (8 - 3 \cdot 2) = 3 \cdot 3 - 8\)\\
    \(= 3 \cdot (51 - 8 \cdot 6) - 8 = 3 \cdot 51 - 8 \cdot 19\)\\
    \(= 3 \cdot 51 - 19(365 - 51 \cdot 7)\)\\
    \(= 136 \cdot 51 - 19 \cdot 365\)\\
    \(= 136 \cdot (1876 - 365 \cdot 5) - 19 \cdot 365\)\\
    \(= 136 \cdot 1876 - 699 \cdot 365\)\\
\end{center}
Quindi \(x = -699\) e \(y = 136\).\\
In generale possiamo enunciare il seguente teorema:

\begin{teorema}
    siano \(a,b \in \mathbb{N} \setminus {0}\), se \(a \mid b\), allora \(a = MCD\{a,b\}\).\\
    se \(a \nmid b\) e \(r\) è l'ultimo resto non nullo dell'algoritmo di Euclide, allora \(r = MCD \{a,b\}\).\\
    inoltre esistono \(x,y \in \mathbb{Z} \) tali che \(ax + by = MCD \{a,b\}\).
\end{teorema}

\begin{dimostrazione}
    \
    Sia \(I = \{ax + by : x,y \in \mathbb{Z} \}\) l'insieme dei multipli di \(a\) e \(b\).\\
    Poiché \(I\) è un ideale di \(\mathbb{Z} \), allora \(I = n \mathbb{Z} \) per qualche \(n \in \mathbb{N} \).\\
    Poiché \(a \in I\), allora \(n \mid a\).\\
    Poiché \(b \in I\), allora \(n \mid b\).\\
    Quindi \(n = MCD \{a,b\}\).\\
    Inoltre, poiché \(r \in I\), allora \(r = ax + by\) per qualche \(x,y \in \mathbb{Z} \).\\
    Quindi \(r = MCD \{a,b\}\).\\
    \\
    fatta da copilot, controllare a pag 40 di "a concrete introduction to higher algebra" di Lindsay Childs
\end{dimostrazione}

\subsection{Equazioni diofantee lineari}
sono equazioni del tipo \(ax + by = c\), con \(a,b,c \in \mathbb{Z} \).\\

\begin{proposition}
    siano \(a,b,c \in \mathbb{Z} \).\\
    allora esistono \(x,y \in \mathbb{Z} \) tali che \(ax + by = c\) se e solo se \(MCD \{a,b\} \mid c\).
\end{proposition}

\begin{dimostrazione}
    \
    Se \(ax + by = c\), allora \(MCD \{a,b\} \mid c\).\\
    Viceversa, se \(d := MCD \{a,b\} \mid c\), allora abbiamo un'identità di Bézout \(ax + by = d\) \(\forall x,y \in \mathbb{Z}\).\\
    se \(d \mid c\) cioè se \(c = d \cdot k\) per qualche \(k \in \mathbb{Z} \), \(a(kx) + b(ky) = kd = c\)\\
\end{dimostrazione}

\begin{example}
    l'equazione diofantea:\\
    \(365x - 1876y = 24\) ha soluzione perchè \(MCD \{365,1876\} = 1\) e \(1 \mid 24\).\\
    Avevamo l'identità di Bézout \(365(-699) - 1876(-136) = 1\), moltiplicando per 24 otteniamo\\
    \(365(-699 \cdot 24) - 1876(-136 \cdot 24) = 24\).\\
    ossia una soluzione è \(x = -699 \cdot 24\) e \(y = -136 \cdot 24\).
\end{example}

\begin{example}
    in \(\mathbb{Z}_{1876}\) calcolare, se esiste, l'inverso moltiplicativo di \(\overline{365}\).\\
    abbiamo che \(\overline{365} \cdot \overline{a} = \overline{1}\) in \(\mathbb{Z}_{1876}\)\\
    se e solo se esistono \(a,b \in \mathbb{Z}\) t.c. \(365 \cdot a = 1 + b \cdot 1876 \leftrightarrow 365 \cdot a - 1876 \cdot b = 1\).\\
    una soluzione è \(a = -699\) e \(b = 136\), ossia \(\overline{365}^{-1} = \overline{-699} = \overline{1177}\).
\end{example}
\newpage

\subsection{Morfismi di anelli}

\begin{definition}
    se \(p \in \mathbb{N}\) è un numero primo, scriviamo \(\mathbb{F}_p := \mathbb{Z}_p\) ;\\
    il campo \(\mathbb{F}_p\) ha \(p\) elementi.
\end{definition}

\begin{definition}
    Siano \(A,B\) due anelli. Un'applicazione \(f : A \rightarrow B\) è un \textbf{morfismo di anelli} se:
    \
    \begin{itemize}
        \item \(f: (A,+) \rightarrow (B,+)\) è un morfismo di gruppi.
        \item \(f: (A,\cdot) \rightarrow (B,\cdot)\) è un morfismo di monoidi.
    \end{itemize}
\end{definition}

\begin{definition}
    il nucleo di un morfismo di anelli\\
    \(f : A \rightarrow B\) è l'insieme \(Ker(f):=\{a \in A : f(a) = 0\}\).\\
\end{definition}

\begin{osservazione}
    \(Ker(f)\) è un ideale di \(A\), \(A\) anello commutativo.
\end{osservazione}

\begin{example}
    sia \(I \subseteq A\) un ideale di un anello commutativo \(A\).\\
    allora la proiezione canonica \(\pi : A \rightarrow \sfrac{A}{I}\) che mappa \(a \rightarrow [a]\)\\
    è un morfismo di anelli il cui nucleo è \(I\).
\end{example}

\begin{example}
    si consideri l'anello dei numeri complessi \(\mathbb{C}\).\\
    allora il coniugio \(\overline{z} = \overline{a+bi} = a - bi\) è un morfismo di anelli da \(\mathbb{C}\) in \(\mathbb{C}\):\\
    \(\overline{1} = 1, \overline{z_1 + z_2} = \overline{z_1} + \overline{z_2}, \overline{z_1 \cdot z_2} = \overline{z_1} \cdot \overline{z_2}\)
\end{example}

\begin{teorema}[di isomorfismo per anelli commutativi]
    Sia \(f : A \rightarrow B\)\\
    un morfismo di anelli commutativi. Allora esiste un morfismo iniettivo di anelli\\
    \(\Psi : \sfrac{A}{Ker(f)} \rightarrow B\) tale che il seguente diagramma è commutativo:
    \[
        \begin{tikzcd}
            A \arrow{r}{f} \arrow[swap]{d}{\pi} & B \\
            \sfrac{A}{Ker(f)} \arrow[swap]{ur}{\Psi}
        \end{tikzcd}
    \]
    in particolare, se \(f\) è suriettivo, allora \(\Psi\) è un isomorfismo di anelli.
\end{teorema}

\newpage

\textbf{Notazione:} \(\overline{x} \in \mathbb{Z}_n\). La classe di equivalenza \(\overline{x}\) la scriveremo anche  \(x \mod n\).

\begin{teorema} [Teorema cinese dei resti]
    siano \(n_1,n_2,...,n_k \in \mathbb{N} \setminus \{0,1\}\) tali che\\
    \(MCD \{n_i,n_j\} = 1\) per ogni \(1 \leq i,j \leq k, i \neq j\).\\
    sia \(n := n_1 \cdot n_2 \cdot ... \cdot n_k\).\\
    allora la funzione \(\Psi : \mathbb{Z}_n \rightarrow \mathbb{Z}_{n_1} \times \mathbb{Z}_{n_2} \times ... \times \mathbb{Z}_{n_k}\) che mappa\\
    \(x mod n \rightarrow (x mod n_1, x mod n_2, ..., x mod n_k)\) è un isomorfismo di anelli.\\
\end{teorema}

\begin{dimostrazione}
    vediamo prima di tutto che \(\Psi\) è un morfismo di anelli dove\\
    \(f: \mathbb{Z} \rightarrow \mathbb{Z}_{n_1} \times \mathbb{Z}_{n_2} \times ... \times \mathbb{Z}_{n_k} \).
    è definita da \(f(x) = (x mod n_1, x mod n_2, ..., x mod n_k) \forall x \in \mathbb{Z}\).
    \begin{itemize}
        \item \(f(a+b) = ((a+b) mod n_1, ... , (a+b) mod n_k)\) \\
                \(= (a mod n_1 + b mod n_1, ... , a mod n_k + b mod n_k)\)\\
                \(= (a mod n_1, ... , a mod n_k) + (b mod n_1, ... , b mod n_k)\)\\
                \(= f(a) + f(b), \forall a,b \in \mathbb{Z}\)

        \item \(f(1) = (1 mod n_1, ... , 1 mod n_k)\) e \((1 mod n_1, ... , 1 mod n_k)\) è l'unità\\
                del prodotto diretto di anelli \(\mathbb{Z}_{n_1} \times \mathbb{Z}_{n_2} \times ... \times \mathbb{Z}_{n_k}\)

        \item \(f(a \cdot b) = ((a \cdot b) mod n_1, ... , (a \cdot b) mod n_k)\)\\
                \(= (a mod n_1 \cdot b mod n_1, ... , a mod n_k \cdot b mod n_k)\)\\
                \(= (a mod n_1, ... , a mod n_k) \cdot (b mod n_1, ... , b mod n_k)\)\\
                \(= f(a) \cdot f(b), \forall a,b \in \mathbb{Z}\)
    \end{itemize}
    ora mostriamo che \(f\) è suriettivo:\\
    sia \((a_1 mod n_1, ... , a_k mod n_k) \in \mathbb{Z}_{n_1} \times \mathbb{Z}_{n_2} \times ... \times \mathbb{Z}_{n_k}\)\\
    osserviamo che \(MCD\{n_i,n_1 n_2 ... n_{i-1} n_{i+1} ... n_k\} = 1, \forall 1 \leq i \leq k\).\\
    quindi abbiamo le identità di Bézout: \(c_i n_i + b_i \frac{n}{n_i} = 1\) ossia\\
    \(u_i + v_i = 1\) dove \(u_i = c_i n_i \in <n_i>\) e \(v_i = b_i \frac{n}{n_i} \in <\frac{n}{n_i}>\).\\
    diefiniamo \(x := a_1 v_1 + ... + a_k v_k\) e abbiamo che \(f(x) = (a_1 mod n_1, ... , a_k mod n_k)\).\\
    infatti \(v_i mod n_j = 
    \begin{cases}
        0 & \text{se } i \neq j \\
        1 & \text{se } i = j
    \end{cases}\)\\
    dal teorema di isomorfismo abbiamo che \(\sfrac{\mathbb{Z}}{Ker(f)} \simeq \mathbb{Z}_{n_1} \times ... \times \mathbb{Z}_{n_k} \) come anelli.\\
    ma abbiamo che \(Ker(f) = <n_1> \cap <n_2> \cap ... \cap <n_k>\)\\
    \(= <mcm\{n_1 ,..., n_k\}> = <n_1 n_2 ... n_k>\) dato che \(n_i\) e \(n_j\) sono coprimi \(\forall i \neq j\).\\
    quindi \(\sfrac{\mathbb{Z}}{Ker(f)} = \sfrac{\mathbb{Z}}{<n>} = \mathbb{Z}_n \) e l'isomorfismo \(\Psi: \mathbb{Z}_n \rightarrow \mathbb{Z}_{n_1} \times ... \times \mathbb{Z}_{n_k}\)\\
    è quello dell'enunciato del teorema.
\end{dimostrazione}

\newpage

\begin{example}
    siano \(n_1 = 3, n_2 = 7 e n_3 = 10\). Allora \(n := n_1 n_2 n_3 = 210\)\\
    e abbiamo l'isomorfismo di anelli \(\mathbb{Z}_{210} \simeq \mathbb{Z}_3 \times \mathbb{Z}_7 \times \mathbb{Z}_{10}\).\\
    sia \((2 mod 3, 5 mod 7, 4 mod 10) \in \mathbb{Z}_3 \times \mathbb{Z}_7 \times \mathbb{Z}_{10}\), questa terna corrisponde ad un elemento\\
    \(x mod 210 \in \mathbb{Z}_{210}\) che soddisfa il sistema
    \[
        \begin{cases}
            x\mod 3 = 2\mod 3 \\
            x\mod 7 = 5\mod 7 \\
            x\mod 10 = 4\mod 10\\
        \end{cases}
        \]
\end{example}

la dimostrazione del teorema cinese dei resti ci dice come trovare x.\\
\(x = 2 v_1 + 5 v_2 + 4 v_3\) dove se \(3 a + 70 b = 1, 7 a + 30 b = 1\) e \(10 a + 21 b = 1\)\\
sono identità di Bézout, allora \(v_1 = 70 b, v_2 = 30 b  = 30, v_3 = 21 b\)\\
\begin{center}
    \(3 a + 70 b = 1 \rightarrow a = -23, b = 1 \rightarrow v_1 = 70\)\\
    \(7 a + 30 b = 1 \rightarrow 30 = 4 \cdot 7 + 2 , 7 = 3 \cdot 2 + 1\)\\
    \(\rightarrow 1 = 7 - 3 \cdot 2 = 7 - 3(30 - 4 \cdot 7) = \)\\
    \(13 \cdot 7 - 3 \cdot 30 = 91 - 90 = 1 \rightarrow a = 13, b = -3 \rightarrow v_2 = -3 \cdot 30\)\\
    \(10 a + 21 b = 1 \rightarrow a = -2, b = 1 \rightarrow v_3 = 21\)\\
\end{center}

quindi \(x = 2 \cdot 70 - 5 \cdot 3 \cdot 30 + 4 \cdot 21 = 194\mod 210\)\\

\textbf{Corollario:} Sia \(U(\mathbb{Z}_n)\) il gruppo degli elementi invertibili dell'anello \(\mathbb{Z}_n\).\\
sia \(n := n_1 ... n_k\) dove \(MCD\{n_i, n_j\} = 1 \forall 1 \leq i,j \leq k, i \neq j\).\\
e \(n_i \in \mathbb{N} \setminus \{0,1\} \forall 1 \leq i \leq k\).\\
allora come i gruppi \(U(\mathbb{Z}_n) \simeq U(\mathbb{Z}_{n_1}) \times ... \times U(\mathbb{Z}_{n_k})\)

\begin{dimostrazione}
    l'isomorfismo \(\Psi\) del teo. cinese dei restti, ristretto a \(U(\mathbb{Z}_n)\) dà un isomorfismo di gruppi\\
\end{dimostrazione}

Poiché un elemento \(\overline{x} \in \mathbb{Z}_n\) è invertibile s.s.e. esiste un'identità di Bézout \(ax + bn = 1\)\\
abbiamo che \(\overline{x}\) è invertibile s.s.e. \(MCD\{x,n\} = 1\).\\
Quindi \(|U(\mathbb{Z}_n)| = \varphi(n)\), con \(\varphi\) funzione di Eulero.\\

dal precedente Corollario e da questo segue un altro Corollario:

\textbf{Corollario:} Sia \(\varphi : \mathbb{N} \setminus \{0\} \rightarrow \mathbb{N} \setminus \{0\}\) la funzione \(\varphi\) di Eulero.\\
siano \(x,y \in \mathbb{N} \setminus \{0\}\) tali che \(MCD\{x,y\} = 1\), allora \(\varphi(xy) = \varphi(x) \cdot \varphi(y)\).\\

\begin{dimostrazione}
    dal Corollario precedente abbiamo che \(U(\mathbb{Z}_{xy}) \simeq U(\mathbb{Z}_x) \times U(\mathbb{Z}_y)\)\\
    come i gruppi, quindi:
    \begin{center}
        \(\varphi(xy) = |U(\mathbb{Z}_{xy})| = |U(\mathbb{Z}_x) \times U(\mathbb{Z}_y)| = |U(\mathbb{Z}_x)| \cdot |U(\mathbb{Z}_y)| = \varphi(x) \cdot \varphi(y)\)
    \end{center}
\end{dimostrazione}
\newpage

Come conseguenza del corollario precedente otteniamo una formula per calcolare \\
la funzione \(\varphi\) di Eulero.\\
Se \(p\) è un numero primo, allora ci sono \(p^k\) numeri \(1 \leq n \leq p^k\).\\
Di questi numeri \(p, 2p, ... , p^{k-1}p\) hanno fattori comuni con \(p^k\) e quindi
\begin{center}
    \(\varphi(p^k) = p^k - p^{k-1}\).\\
\end{center}
se \(n = p^{k_1}...p^{k_s}\) per il corollario precedente \((n>1)\):\\
\(\varphi(n)= \varphi(p_1^{k_1}...\varphi(p_s^{k_s}) = (p_1^{k_1}-p_1^{k_1-1})...(p_s^{k_s}-p_s^{k_s-1}) = p_1^{k_1}...p_s^{k_s}\prod_{p|n, p primo}(1-\frac{1}{p}) = \)\\
\(= n \prod_{p|n, p primo}(1-\frac{1}{p})\).\\

\begin{teorema}[di Eulero]
    Sia \(n \in \mathbb{N} \setminus \{0,\}\) ed \(a \in \mathbb{N} \setminus \{0\}\) tale che \(MCD\{a,n\} = 1\).\\
    allora \(a^{\overline{\varphi(n)}} = \overline{1} \in \mathbb{Z}_n\). (diciamo che \(a^{\varphi(n)} \equiv 1 \mod n\))\\
\end{teorema}

\begin{dimostrazione}
    sappiamo che la cardinalità del gruppo degli elementi invertibili \\
    di \(\mathbb{Z}_n\) è \(\varphi(n)\).\\
    Sia \(<\overline{a}> \subseteq U(\mathbb{Z}_n)\) il sottogruppo generato da \(\overline{a} in U(\mathbb{Z}_n)\). allora \(|<\overline{a}>|\) divide \(\varphi(n)\),\\
    ossia \(\varphi(n) = k |<\overline{a}>|\), per qualche \(k \in \mathbb{N}\). Sia \(c := |<\overline{a}>|\);\\
    abbiamo che \( = \overline{1} = \overline{a}^c = (\overline{a^c})^k  = \overline{a^{ck}} = \overline{a^{\varphi(n)}} \).\\
\end{dimostrazione}

\textbf{Corollario:}(piccolo teorema di Fermat) Sia \(p\) un numero primo e \(a \in \mathbb{N}\).\\
allora in \(\mathbb{Z}_p\) abbiamo che \(\overline{a} = \overline{a^p} (a^p \equiv a \mod p)\).\\

\begin{dimostrazione}
    se \(p\) è primo si ha che \(\varphi(p) = p-1\). allora dal Teo. di Eulero segue che, se \(a \neq 0, p \nmid a\), \(a^{varphi(p) \equiv 1 \mod p}\implies a ^{p-1} \equiv 1 \mod p \implies a^p \equiv a \mod p\).\\
    se \(a = 0 o p | a\) l' uguaglianza si riduce a \(\overline{0} = \overline{0}\).
\end{dimostrazione}

\newpage

\subsection{Caratteristica di un anello}
sia \(A\) un anello. il sottogruppo \(<1_A> \subseteq (A,+)\)è un gruppo ciclico.\\
quindi esiste un \(n \in \mathbb{N}\) tale che \(<1_A> \simeq \mathbb{Z}_n\). \(n\) è detto la caratteristica dell'anello \(A\).\\

\begin{example}
    la caratteristica di \(\mathbb{Z}\) è 0, infatti \(<1> = \mathbb{Z} \simeq \mathbb{Z}_0\).\\
    la caratteristica degli anelli \(\mathbb{Q},\mathbb{R},\mathbb{C}\) è sempre 0 poiché \(<1> = \mathbb{Z} \simeq \mathbb{Z}_0\) in  \(\mathbb{Q},\mathbb{R},\mathbb{C}\)\\
\end{example}

\begin{example}
    sia \(n \in \mathbb{N}\) allora la caratteristica dell'anello \(\mathbb{Z}_n\) è \(n\).\\
    infatti \(<\overline{1}> = \mathbb{Z}_n\), rispetto all'operazione +\\
\end{example}

indichiamo con \(CHAR(A)\) la caratteristica di un anello \(A\).\\
\begin{definition}
    sia A un anello e sia \(<1_A>\) il sottogruppo di (A,+) generato da \(1_a\).
    l'intersezione di tutti i sottoanelli di A contenenti \(<1_a>\)\\
    si chiama \textbf{sottoanello fondamentale di A}.\\
\end{definition}

\begin{example}
    il sottoanello fondamentale di \(\mathbb{Z},\mathbb{Q},\mathbb{R},\mathbb{C}\) è \(\mathbb{Z}\)
\end{example}

\begin{definition}
    sia K un campo\\
    l'intersezione di tutti i sottocampi di K contenenti il gruppo \(<1_k> \subseteq (K,+)\) si chiama \\
    \textbf{sottocampo fondamentale di K}.\\
\end{definition}

\begin{example}
    il sottocampo fondamentale di \(\mathbb{Q},\mathbb{R},\mathbb{C}\) è \(\mathbb{Q}\).\\
    se \(p \in \mathbb{N}\) è primo, il sottocampo fondamentale di \(\mathbb{F}_p\) è \(\mathbb{F}_p\) perché \(<\overline{1}> = \mathbb{F}_p\).\\
\end{example}

\newpage

\subsection{Anello dei polinomi in una indeterminata a coefficienti in un campo}
Sia \(K\) un campo. una funzione \(f: \mathbb{N} \rightarrow K\) si chiama \textbf{successione a valori in \(K\)}.\\
ad una successione a valori in \(K\) corrisponde una serie formale nella variabile \(x\) su \(K\):\\
\begin{center}
    \(\sum_{n=0}^{\infty} f(n) x^n\)
\end{center}
se l'insieme \(\{ m \ in \mathbb{N} : f(n) \neq 0\}\) è finito diciamo che la serie formale\\
è un polinomio in \(x\) di grado \(deg(P) := MAX \{ n \in \mathbb{N}: f(n) \neq 0\}\).\\
il grado del poliniomio 0 non è definito.\\
l'insieme dei polinomi in \(x\) a coefficienti in \(K\) si indica con \(K[x]\) ed è un anello commutativo con le operazioni:\\
\begin{itemize}
    \item somma: \((\sum_{n=0}^{\infty} a_n X^n) + (\sum_{n=0}^{\infty} b_n X^n) = \sum_{n=0}^{\infty} (a_n + b_n) X^n\)
    \item prodotto: \((\sum_{n=0}^{\infty} a_n X^n) \cdot (\sum_{n=0}^{\infty} b_n X^n) = \sum_{n=0}^{\infty} (\sum_{k=0}^{n} a_k b_{n-k}) X^n\)
\end{itemize}
l'unità di \(K[x]\) è il polinomio \(1_k\).\\

\begin{example}
    in \(\mathbb{F}_2[x]\) siano \(P := 1 + X^2 + X^3\) e \(Q:= X + X^2\).\\
    allora \(P+Q = 1 + X + X^2 + X^3\) e \(P \cdot Q = X + X^2 + X^3 + X^5\)
\end{example}

\begin{proposition}
    siano \(P,Q \in K[x]\) polinomi non nulli. allora il grado del prodotto \(P \cdot Q\) è \(deg(P) + deg(Q)\).\\
    in particolare \(K[x]\) è un dominio di integrità.\\
\end{proposition}

\begin{definition}
    un polinomio si dice \textbf{monico} se il coefficiente del termine di grado massimo è 1.\\
\end{definition}

\begin{definition}
    sia K un campo. un polinomio \(P \in K[x]\) si dice \textbf{irriducibile}\\
    se i suoi unici divisori sono del tipo \(a, aP\) con \(a \in K \setminus \{0\}\).\\
    altrimenti si dice \textbf{riducibile}.\\
\end{definition}

\begin{example}
    in \(\mathbb{F}_2[X]\) il polinomio \(X^2 + 1\) è irriducibile, infatti:\\
    \(X^2 + 1 = (X + 1)^2\), quindi \(X + 1\) divide \(X^2 + 1\) e \(X + 1 \notin K \setminus \{0\}\).
\end{example}

\begin{example}
    in \(K[X]\) ogni polinomio di grado 1 è irriducibile, infatti:\\
    se \(deg(P) = 1\) allora \(P = aX + b\) con \(a,b \in K, a \neq 0\).\\
    i suoi divisori sono \(c\) e \(c^{-1} (aX + b), c \in K \setminus \{0\}\).
\end{example}

\begin{definition}
    sia \(\alpha \in K\). l'elemento \(\alpha\) è detto \textbf{radice} del polinomio\\
    \(P = \sum_{n=0}^{\infty} a_n X^n \in K[X]\) se \(P(\alpha) = \sum_{n=0}^{\infty} a_n \alpha^n = 0\).\\
\end{definition}

\newpage

anche nell'anello \(K[X]\) come in \(\mathbb{Z}\) abiamo un algoritmo di divisione Euclidea.\\
se \(f(X), g(X) \in K[X]\) sono polinomi non nulli allora esistono unici polinomi \(q(X), r(X) \in K[X]\) tali che:\\
\(f(X) = q(X) \cdot g(X) + r(X)\) e \(r(X) = 0\) oppure \(deg(r) < deg(g)\).\\
\(q(X)\) si chiama \textbf{quoziente} e \(r(X)\) si chiama \textbf{resto} della divisione.\\
ne segue il seguente teorema, dimostrato come in \(\mathbb{Z}\):
\begin{teorema}
    l'anello \(K[X]\) è a ideali principali.\\
    se \(I = <p(X)>\) allora esiste un unico generatore monico di \(I\).\\
\end{teorema}

\begin{definition}
    definiamo il \textbf{massimo comune divisore} di due polinomi \(f(X), g(X) \in K[X]\) come l'unico massimo comune divisore monico.
\end{definition}

Come in \(\mathbb{Z}\) possiamo trovarlo con l'algoritmo delle divisioni successive\\
che dà anche un \underline{identità di Bézout}.

\begin{example}
        \(f(X) = X^4 - X^3 -4X^2 + 4X + 1\) e \(g(X) = X^2 - 1\) in \(\mathbb{Q}[X]\), allora:
        \begin{center}
            \(f(X) = g(X)(X^2 - 3) + (X - 2)\)\\
            \(g(X) = (X - 2)(X + 1) +1 \implies MCD(f,g) = 1\)
        \end{center}
        inoltre
        \begin{center}
             \(1 = g(X) - (X - 2)(X + 1) + 1 = g(X) - [f(X) - g(X)(X^2 - 3)](X + 1) =\)\\
            \(= -(X - 1)f(X) + (X^3 + X^2 - 3X -2)g(X)\).\\
        \end{center}
\end{example}

\textbf{proprietà:} sia K un campo e \(P(X) \in K[X]\) un poliniomio irriducibile.\\
allora l'anello quoziente \( \sfrac{K[X]}{<P(X)>}\) è un campo.\\

\begin{dimostrazione}
    sia \([f] \ in \sfrac{K[X]}{<P(X)>}\) tale che \([p] \neq [0]\) ossia \(p(X)\) non divide \(f(X)\).\\
    Dunque \(MCD\{f(X),p(X)\} = 1\) perchè \(p(X)\) è irriducibile.\\
    quindi abbiamo un'identità di Bézout \(a(X)f(X) + b(X)p(X) = 1\).\\
    ossia \([a(X)] = [f(X)]^{-1}\) in \(\sfrac{K[X]}{<P(X)>}\).
\end{dimostrazione}

\begin{example}
    in \(\mathbb{F}_2[X]\) il polinomio \(P(X) = 1 + X + X^2\) è irriducibile.\\
    infatti non ha radici in \(\mathbb{F}_2\).\\
    quindi l'anello \(\sfrac{\mathbb{F}_2[X]}{<1 + X + X^2>}\) è un campo, che chiamiamo \(\mathbb{F}_4\).\\
    un elemento di \(\mathbb{F}_4\) è della forma \(a_0 + a_1X\) con \(a_0,a_1 \in \mathbb{F}_2\).\\
    la tavola moltiplicativa è la seguente:
    \begin{center}
        \begin{tabular}{c|cccc}
            \(\cdot\) & 0 & 1 & X & 1 + X \\
            \hline
            0 & 0 & 0 & 0 & 0 \\
            1 & 0 & 1 & X & 1 + X \\
            X & 0 & X & 1 + X & 1\\
            1 + X & 0 & 1 + X & 1 & X\\
        \end{tabular}
    \end{center}
    l'inverso di \(X\) è \(1 + X\).\\
\end{example}

\newpage

\begin{example}
    in \(\mathbb{F}_3[X]\) il polinomio \(P(X) = 1 + X^2\) è irriducibile.\\
    indichiamo con \(\mathbb{F}_9\) il campo\(\sfrac{\mathbb{F}_3[X]}{<1 + X^2>}\).\\
    un elemento di \(\mathbb{F}_9\) è della forma \(a_0 + a_1X\) con \(a_0,a_1 \in \mathbb{F}_3\) quindi sono 9.\\
    la tavola moltiplicativa è la seguente:
    \begin{center}
        \begin{tabular}{c|ccccccccc}
            \(\cdot\) & 0 & 1 & 2 & X & 1 + X & 2 + X & 2X & 1 + 2X & 2 + 2X\\
            \hline
            0 & 0 & 0 & 0 & 0 & 0 & 0 & 0 &0 & 0 \\
            1 & 0 & 1 & 2 & X & 1 + X & 2 + X & 2X & 1 + 2X & 2 + 2X \\
            2 & 0 & 2 & 1 & 2X & 2 + 2X & 1 + 2X & X & 2 + X & 1 + X \\
            X & 0 & X & 2X & 2 & 2 + X & 2 + 2X & 1 & 1 + X & 1 + 2X \\
            1 + X & 0 & 1 + X & 2 + 2X & 2 + X & 2X & 1 & 1 + 2X & 2 & X \\
            2 + X & 0 & 2 + X & 1 + 2X & 2 + 2X & 1 & X & 1 + X & 2X & 2 \\
            2X & 0 & 2X & X & 1 & 1 + 2X & 1 + X & 2 & 2 + 2X & 2 + X \\
            1 + 2X & 0 & 1 + 2X & 2 + X & 1 + X & 2 & 2X & 2 + 2X & X & 1 \\
            2 + 2X & 0 & 2 + 2X & 1 + X & 1 + 2X & X & 2 & 2 + X & 1 & 2X \\
        \end{tabular}
    \end{center}
    l'inverso di \(X\) è 2.\\
\end{example}

\begin{teorema}[di Ruffini]
    sia \(f(X) \in K[X]\) un polinomio non nullo.\\
    se \(\alpha \in K\), il resto della divisione di \(f(X)\) per \(X - \alpha\) è \(f(\alpha)\),\\
    in particolare \(\alpha\) è una radice di \(f(X)\) s.s.e. \(X - \alpha\) divide \(f(X)\) in \(K[X]\).\\
\end{teorema}

\begin{dimostrazione}
    \(f(X) = (X - \alpha) q(X) + r(X)\) con \(r(X) = 0\) oppure \(deg(r(X)) < 1\).\\
    quindi \(r(X)\) è un polinomio costante, \(r(X) = x \in K\).\\
    calcolando in \(\alpha\) otteniamo \(f(\alpha) = c\).\\
\end{dimostrazione}

\begin{example}
    il polinomio \(X^2 + 1 \in \mathbb{R}[X]\) non ha radici in \(\mathbb{R}\)\\
    quindi è irriducibile e \(\sfrac{\mathbb{R}[X]}{<X^2 + 1>}\) è un campo isomorfo a \(\mathbb{C}\),\\
    dove l'isomorfismo è dato dall'assegnazione \(1 \rightarrow 1\) e \(x \rightarrow i\)\\
\end{example}

\newpage

enunciamo il seguente importante risultato, senza fornire la dimostrazione.\\
(vedi proposizione 4.3.5 di "Teoria delle equazioni e teoria di Galois" - S.Gabelli).\\

\begin{proposition}
    se K è un campo, ogni sottogruppo finito del gruppo moltiplicativo\\
    \(K \setminus \{0\}\) è ciclico. in particolare, se K è un campo finito, \(K \setminus \{0\}\) è un gruppo ciclico.
\end{proposition}

\begin{example}
    \begin{itemize}
        \item in \(\mathbb{F}_4 = \sfrac{\mathbb{F}_2}{<1 + X + X^2>}\) si ha che \(\{ X, X^2, X^3\} = \{ X, 1 + X, 1\} = \mathbb{F}_4 \setminus \{0\}\)\\
                quindi \(X\) è un generatore del gruppo moltiplicativo \(\mathbb{F}_4 \setminus \{0\}\), l'altro è \(1 + X\)\\
        \item in \(\mathbb{F}_9 = \sfrac{\mathbb{F}_3}{<1 + X^2>}\) abbiamo:\\ 
                \(<X> = \{X, X^2, X^3, X^4\} = \{X, 2, 2X,1\}\)\\
                \(<1 + X> = \{1 + X, (1 + X)^2, (1 + X)^3, (1 + X)^4, (1 + X)^5, (1 + X)^6, (1 + X)^7, (1 + X)^8\} =\)\\
                \(= \{1 + X, 2X, 1 + 2X, 2, 2 + 2X, X, 2 + X, 1 \}\)\\
                \(= \mathbb{F}_9 \setminus \{0\}\) quindi \(1 + X\) genera il gruppo moltiplicativo.
    \end{itemize}
\end{example}

Sia \(p \in \mathbb{N}\) un numero prima e sia \(n \in \mathbb{N} \setminus \{0\}\).\\
sia \(Q(X), \mathbb{F}_p[X]\) un qualsiasi polinomio irriducibile di grado n.\\
definiamo il campo
\begin{center}
    \(\mathbb{F}_{p^n} = \sfrac{\mathbb{F}_p[X]}{<Q(X)>}\)
\end{center}

vogliamo ora mostrare che se \(Q(X), Q'(X) e \mathbb{F}_p[X]\) sono polinomi irriducibili di grado n,\\
allora 
\begin{center}
    \(\sfrac{\mathbb{F}_p[X]}{<Q(X)>} \simeq \sfrac{\mathbb{F}_p[X]}{<Q'(X)>}\), isomorfismo tra campi
\end{center}

quindi la definizione di \(\mathbb{F}_p\) è ben posta, a meno di isomorfismi.\\

\begin{definition}
    siano \(F \subseteq K\) due campi (ampliamento di campi).\\
    un elemento \(\alpha \in K\) si dice \underline{algebrico} su F se è radice di qualche polinomio non nullo\\
    su \(f(X) \in F(X)\), altrimenti si dice \underline{trascendente} su F.
\end{definition}

dato un ampliamento di campi \(F \subseteq K\) e  \(\alpha \in K\), si consideri il morfismo di anelli
\begin{center}
    \(v_\alpha : F[X] \rightarrow K\)\\
    \(f(X) \rightarrow f(\alpha)\).
\end{center}

\(Ker(v_\alpha)\) è l'ideale di \(F[X]\) costituito dai polinomi che si annullano in \(\alpha\).\\
quindi \(\alpha\) è algebrico su F s.s.e. \(Ker(v_\alpha)\) è un ideale non nullo di \(F[X]\).\\
poiche \(F[X]\) è ad ideali principali, \(ker(v_\alpha) = <m(X)>\)\\
dove \(m(X)\) è l'unico polinomio monico di grado minimo in \(Ker(v_\alpha)\).

\begin{definition}
    se \(\alpha \in K\) è algebrico su F, il polinomio \(m(X)\) definito sopra si chiama\\
    \textbf{polinomio minimmo di \(\alpha\) su F}, se \(deg(m(X)) = n\), \(\alpha \) si dice algebrico di grado n
\end{definition}

\textbf{Nota:} sia \(\alpha \in K\) e \(P(X) \in F[X] \setminus \{0\})\) tale che \(p(\alpha) = 0\),\\
allora \(p(X)\) è il polinomio minimo di \(\alpha\) su F s.s.e. \(p(X)\) è monico e irriducibile.\\

\begin{example}
    si consideri l'ampliamento \(\mathbb{R} \subseteq \mathbb{C}\). allora \(1 + X^2 \in \mathbb{R}[X]\)\\
    è il polinomio minimo di \(i \in \mathbb{C}\) su \(\mathbb{R}\).
\end{example}

\newpage

\textbf{Proprietà:} sia \(F \in K\) un ampliamento di campi e \(\alpha \in K\).\\
si consideri il morfismo di anelli \(v_\alpha: F[X] \rightarrow K\).\\
allora \(Im(v_\alpha)\) è il più piccolo sottoanello di K contenente sia F che \(\alpha\)

\begin{dimostrazione}
    si osservi che l'immagine di un morfismo di anelli è un sottoanello.\\
    di conseguenza \(Im(v_\alpha)\) è un sottoanello di K.\\
    sia \(c \in F\) e si consideri il polinomio costante \(c \in F[X]\). allora \(v_\alpha(c)=c\).\\
    quindi \(F \subseteq Im(v_\alpha)\) e \(v_\alpha(X) = \alpha\ \implies \alpha \in Im(v_\alpha)\)\\
    d'altra parte per chiusura aditiva e moltiplicativa,\\
    ogni sottoanello di K contenete sia F che \(\alpha\) contiene anche \(Im(v_\alpha)\).
\end{dimostrazione}

\begin{proposition}
    sia \(F \subseteq K\) un ampliamento di campi e sia \(\alpha \in K\).\\
    il più piccolo sottocampo di K contenente sia F che \(\alpha\) si chiama\\
    \textbf{ampliamento di F in K generato da \(\alpha\)} e si indica con \textbf{F(\(\alpha\))}
    tale ampliamento si dice \textbf{semplice} (poichè generato da un solo elemento)
\end{proposition}

\vspace{\baselineskip}

da questa proposizione segue questo Corollario:\\

\textbf{Corollario:} sia \(F \subseteq K\) un ampliamento di campi e sia \(\alpha \in K\).\\
allora \(F(\alpha) = \{f(\alpha)g(\alpha)^{-1} : f(X),g(X) \in F[X], g(\alpha) \neq 0\}\).

\begin{dimostrazione}
    per la proposizione precedente\\
    il più piccolo sottoanello di K contenente sia F che \(\alpha\) è \(Im(v_\alpha = \{f(\alpha) : f(X) \in F[X]\})\).\\
    prendendo gli inversi in K si ottiene la tesi.
\end{dimostrazione}

\vspace{\baselineskip}

se \(\alpha \in K\) è algebrico su F si ha che \(Im(v_\alpha \simeq \sfrac{F[X]}{<m(X)>})\),\\
dove \(m(X)\) è il polinomio minimo di \(\alpha\). quindi \(Im(v_\alpha)\) è un campo e \(F(\alpha) = Im(v_\alpha)\).\\
se n è il grado di \(\alpha\) si ha quindi:
\begin{center}
    \(F(\alpha) = \{c_0 + c_1\alpha + ... + c_{n-1}\alpha^{n-1} : c_i \in F\}\)
\end{center}

\begin{example}
    si consideri l'ampliamento \(\mathbb{Q} \subseteq \mathbb{R}\). l'elemento \(\sqrt{2}\in \mathbb{R} \setminus \mathbb{Q}\) è akgebrico su \(\mathbb{Q}\)\\
    con polinomio minimo \(X^2 -2\). quindi \(\sqrt{2}\) ha grado 2 su \(\mathbb{Q}\) e
    \begin{center}
        \(\mathbb{Q}(\sqrt{2}) = \{ c_0 + c_1 \sqrt{2} : c_0,c_1 \in \mathbb{Q}\})\).
    \end{center}
\end{example}

adesso mostriamo che il campo \(\mathbb{F}_{p^n}\) è un ampliamento semplice di \(\mathbb{F}_p\)
\begin{proposition}
    sia \(\alpha \in \mathbb{F}_{p^n}\) un generatore del campo moltiplicativo \(\mathbb{F}_{p^n} \setminus \{0\}\).\\
    allora \(\mathbb{F}_{p^n} = \mathbb{F}_p(\alpha)\).
\end{proposition}

\begin{dimostrazione}
    \(\mathbb{F}_p(\alpha)\) è il più piccolo sottocampo di \(\mathbb{F}_{p^n}\) contenente sia \(\mathbb{F}_p\) che \(\alpha\)\\
    quindi \(\mathbb{F}_p(\alpha) \subseteq \mathbb{F}_{p^n}\). Poiché \(\alpha\) genera il gruppo moltiploicativo \(\mathbb{F}_{p^n} \setminus \{0\}\) anche \(\mathbb{F}_{p^n} \subseteq \mathbb{F}_p(\alpha)\)
\end{dimostrazione}

\newpage

Ora, se \(P(X), Q(X) \in \mathbb{F}_p[X]\) sono due polinomi irriducibili di grado n,\\
vogliamo costruire un isomorfismo 
\begin{center}
    \(f: \sfrac{\mathbb{F}_p[X]}{<P(X)>} \rightarrow \sfrac{\mathbb{F}_p[X]}{<Q(X)>}\)
\end{center}
ci serve il seguente risultato:

\begin{proposition}
    siano \(F \subseteq K\) e \(F \subseteq K'\) due ampliamenti di campi.\\
    se \(\alpha \in K\) è algebrico di grado n su F, con polinomio minimo \(m(x)\),\\
    esiste un morfismo di campi \(\varphi : F(\alpha) \rightarrow K'\) che fissa F in K'.\\
    in questo caso i morfismi \(\varphi\) sono tanti quante le radici distinte \(\beta_1, ... ,\beta_s\) di \(m(X)\) in K'.\\
    sono tutti e soli quelli definiti da:
    \begin{center}
        \(c_0 + c_1\alpha + ... + c_{n-1}\alpha^{n-1} \rightarrow c_0 + c_1\beta_i + ... + c_{n-1}\beta_i^{n-1} \)
    \end{center}
\end{proposition}

\begin{dimostrazione}
    se \(\alpha\) è algebrico di grado n su F con polinomio minimo \(m(X)\)\\
    e \(\varphi : F(\alpha) \rightarrow K'\) è isomorfismo, allora \(0 = \varphi(0) = \varphi(m(\alpha)) = m(\varphi(\alpha))\)\\
    quindi \(\varphi(\alpha)\) deve essere radice di \(m(X)\) in K'.\\
    viceversa, sia \(\beta\) una radice di \(m(X)\) in K' e consideriamo il morfismo di anelli\\
    \begin{center}
        \(v_\beta : F[X] \rightarrow K'\)\\
        \(f(X) \rightarrow f(\beta)\)
    \end{center}
    poiché \(m(X) \in Ker(v_\beta)\), dal Teorema di isomorfismo per anelli\\
    abbiamo che il seguente diagramma è commutativo:\\
    \begin{center}
        \begin{tikzcd}
            F[X] \arrow{r}{v_\beta} \arrow{d}{\pi} & K'\\
            F(\alpha) \simeq \sfrac{F[X]}{<m(X)>} \arrow{ur}{\varphi}
        \end{tikzcd}
    \end{center}
    infatti \(Ker(v_\beta) = <m(X)>\), essedo \(m(X)\) irriducibile.\\
    quindi abbiamo trovato un morfismo iniettivo \(\varphi : F(\alpha) \rightarrow K'\) che soddisfa le proprietà dell'enunciato.\\
\end{dimostrazione}

sia F un campo e \(f(X) \in F[X]\) un polinomio di grado \(n \geq 1\).\\
un campo K, ampliamento di F, si dice \textbf{campo di spezzamento di f(X) su F} se:\\  
\begin{itemize}
    \item \(f(X)\) fattorizza in polinomi di grado 1 su \(K[X]\)
    \item non ci sono campi intermedi \(F \subseteq L \subsetneq K\) con la stessa proprietà.
\end{itemize}

\begin{example}
    \(\mathbb{Q}(\sqrt{2})\) è un campo di spezzamenro di \(X^2 - 2 \in \mathbb{Q}[X]\).\\
    \(\mathbb{C}\) è un campo di spezzamenro di \(X^2 + 1 \in \mathbb{R}[X]\).\\
\end{example}

\newpage

Ora vogliamo mostrare che un campo che ha cardinalità \(p^n\)\\
è un campo di spezzamento del polinomio \(X^{p^n} - X \in \mathbb{F}_p[X]\).\\
infatti se K è un campo e \(|K| = p^n\), allora il suo gruppo moltiplicativo \(K \setminus \{0\}\) ha cardinalità \(p^n - 1\)\\
e quindi oer ogni \(\alpha \in K \setminus \{0\}\) si ha \(\alpha^{p^n - 1} = 1\).\\
quindi ogni elemento di K è radice del polinomio \(X^{p^n} - X\).\\
per il teorema di Ruffini, K è un campo di spezzamento di \(X^{p^n} - X\).\\
Adesso mostriamo che ogni poliniomio di grado n irriducibile in \(\mathbb{F}_p[X]\)\\
divide \(X^{p^n} - X\ \in \mathbb{F}_p[X]\).\\

\begin{proposition}
    tutti e soli i polinomi irriducibili su \(\mathbb{F}_p\) di grado n\\
    dividono \(X^{p^n} - X \in \mathbb{F}_p[X]\).
\end{proposition}

\begin{dimostrazione}
    sia \(P(X) \in \mathbb{F}_p[X]\) irriducibile di grado n e sia \(K := \sfrac{\mathbb{F}_p[Y]}{<P(Y)>}\).\\
    allora K ha \(p^n\) elementi che sono le radici di \(X^{p^n} - X \in K[X]\).\\
    poichè \(Y \in K\) è una radice \(P(X) \in K[X]\), \(P(X) e X^{p^n} - X\) hanno una radice in comune in K,\\
    allora per il teorema di Ruffini hanno un fattore comune \(X - Y in K[X]\).\\
    quindi, poiché \(\mathbb{F}_p \subseteq K\) e \(MCD\) in \(\mathbb{F}_p = MCD\) in \(K[X]\)\\
    \(\implies P(X), X^{p^n} - X\) hanno \(MCD \neq 1\) in \(\mathbb{F}_p[X]\).\\
    poiché \(P(X)\) è irriducibile in \(\mathbb{F}_p[X]\), \(P(X)\) divide \(X^{p^n} - X\).\\
\end{dimostrazione}

\vspace{\baselineskip}

adesso vogliamo costruire un isomorfismo di campi
\begin{center}
    \(f: \sfrac{\mathbb{F}_p[X]}{<P(X)>} \rightarrow \sfrac{\mathbb{F}_p[X]}{<Q(X)>}\)
\end{center}
dove \(P(X), Q(X) \in \mathbb{F}_p[X]\) sono monici irriducibili di grado n.\\
basta costruire un isomorfismo di anelli.\\
Infatti un morfismo di anelli che sono campi è iniettivo. Inoltre:
\begin{center}
    \(|\sfrac{\mathbb{F}_p[X]}{<P(X)>}| = |\sfrac{\mathbb{F}_p[X]}{<Q(X)>}| = p^n\)
\end{center}

quindi tale morfismo è biunivoco, ossia è isomorfismo.

\vspace{\baselineskip}

Si ha che, se \(y \in \sfrac{\mathbb{F}_p[Y]}{<P(Y)>}\) allora \(P(X) \in \mathbb{F}_p[X]\) è il polinomio minimo di \(y\) su \(\mathbb{F}_p\).\\
quindi, se \(P(X)\) ha una radice in \(\sfrac{\mathbb{F}_p[Y]}{<Q(Y)>}\),\\
possiamo usare la proposizione sull'estensione di morfismi di campi per definire\\
il morfismo f, che sarà un isomorfismo. Infatti \(\mathbb{F}_p \subseteq \sfrac{\mathbb{F}_p[X]}{<Q(X)>}\). \\
Inoltre \(\sfrac{\mathbb{F}_p[X]}{<P(X)>} = \mathbb{F}_p([X])\), dove \([X]\) è la classe di \(X\) in \(\sfrac{\mathbb{F}_p[X]}{<P(X)>}\).\\
poiché \(\sfrac{\mathbb{F}_p[Y]}{<Q(Y)>}\) è un campo di spezzamento di \(X^{p^n} - X\) e \(P(X)\) divide \(X^{p^n} - X\),\\
allora \(P(X)\) si fattorizza in fattori di grado 1 in \(\sfrac{\mathbb{F}_p[Y]}{<Q(Y)>}\).

\vspace{\baselineskip}

sia \(\beta \in \sfrac{\mathbb{F}_p[Y]}{<Q(Y)>}\) tale che \(p(\beta)\) = 0.\\
allora l'assegnazione
\begin{center}
    \(c_0 + c_1 x + ... + c_{n-1} x^{n-1} \rightarrow c_0 + c_1 \beta + ... + c_{n-1} \beta^{n-1}\)
\end{center}
definisce un morfismo di anelli
\begin{center}
    \(f: \sfrac{\mathbb{F}_p[X]}{<P(X)>} \rightarrow \sfrac{\mathbb{F}_p[X]}{<Q(X)>}\)
\end{center}

\newpage

\begin{example}
    in \(\mathbb{F}_3[X]\) si considerino i polinomi irriducibili 
    \begin{center}
        \(1 + X^2\) e \(2 + X + X^2\).
    \end{center}
    il polinomio minimo di \(X\) in \(\sfrac{\mathbb{F}_3[X]}{<1 + X^2 >} := K\) su \(\mathbb{F}_3\) è \(1 + X^2\).\\
    in \(K' := \sfrac{\mathbb{F}_3[Y]}{<1 + Y + Y^2>}\) si ha che 
    \begin{center}
        \(1 + X^2 = (X + Y + 2)(X + 2Y + 1)\)
    \end{center}
    quindi in \(K'[X], 1 + X^2\) ha due radici:
    \begin{center}
        \(-Y-2 = 2Y+1\) e \(-2Y - 1 = y + 2\).
    \end{center}
    abbiamo quindi due isomorfismi
    \begin{center}
        \(f: K \rightarrow K'\)\\
        \(a_0 + a_1x \rightarrow a_0 + a_1(2Y + 1)\)\\
        \(g: K \rightarrow K'\)\\
        \(a_0 + a_1x \rightarrow a_0 + a_1(Y + 2)\)\\
    \end{center}
    \(f(0) = 0\)\\
    \(f(1) = 1\)\\
    \(f(2) = 2\)\\
    \(f(X) = 2Y + 1\)\\
    \(f(1 + X) = f(1) + f(X) = 2Y + 2\)\\
    \(f(2 + X) = f(2) + f(X) = 2Y\)\\
    \(f(2X) = f(2)f(X) = 2f(X) = y + 2\)\\
    \(f(1 + 2X) = f(1) + f(2X) = Y\)\\
    \(f(2 + 2X) = f(2) + f(2X) = y + 1\) \vspace{\baselineskip}\\
    \(g(0) = 0\)\\
    \(g(1) = 1\)\\
    \(g(2) = 2\)\\
    \(g(X) = Y + 2\)\\
    \(g(1 + X) = g(1) + g(X) = Y\)\\
    \(g(2 + X) = g(2) + g(X) = Y + 1\)\\
    \(g(2X) = g(2)g(X) = 2g(X) = 2Y + 1\)\\
    \(g(1 + 2X) = g(1) + g(2X) = 2Y + 2\)\\
    \(g(2 + 2X) = g(2) + g(2X) = 2Y\)\\
\end{example}

\begin{osservazione}
    \(X \in K\) non è un generatore di \(K \setminus \{0\}\).\\
    infatti il sottogruppo del gruppo moltiplicativo \(K \setminus \{0\}\) generato da \(X\) è\\
    \(<X> = \{X, 2, 2X, 1\} \subsetneq K \setminus \{0\}\)\\
\end{osservazione}

\newpage

\textbf{Lemma:} se K è un anello commutativo di caratteristica prima p, allora
\begin{center}
    \((X + Y)^{p^h} = X^{p^h} + Y^{p^h}\)
\end{center}
per ogni \(x, y \in K, h \geq 1\).

\begin{dimostrazione}
    sia \(h = 1\). se \(p > k > 0\), p divide tutti i coefficienti binomiali\\
    \(\binom{p}{k} := \frac{p!}{k!(p-k)!}\) perché non divide \(k!(p-k)!\).\\
    allora \((X + Y)^p = \sum_{k=0}^{p} \binom{p}{k} X^k Y^{p-k} = X^p + Y^p\).\\
    la tesi segue per induzione.\\
\end{dimostrazione}

\textbf{Automorfismo di Frobenius:}\\
Dal lemma precedente segue che se K è un campo di caratteristica p, allora la funzione 
\begin{center}
    \(\Phi : K \rightarrow K\)\\
    \(x \rightarrow x^p\)
\end{center}
è un morfismo di campi. infatti\\
\(\Phi(x + y) = (x + y)^p = x^p + y^p = \Phi(x) + \Phi(y)\)\\
\(\Phi(xy) = (xy)^p = x^p y^p = \Phi(x) \Phi(y)\)\\
\(\forall x,y \in K\).\\
se \(K = \mathbb{F}_{p^n}, \Phi\) è un automorfismo\\
(essendo morfismo initettivo da un campo di cardinalità finita in se stesso)\\
detto \textbf{automorfismo di Frobenius}.

\begin{teorema}
    il gruppo degli automorfismi di \(\mathbb{F}_{p^n}, AUT(\mathbb{F}_p^n)\) è ciclico di cardinalità n,\\
    generato dall'automorfismo di Frobenius.
\end{teorema}

\begin{dimostrazione}
    vedi teorema 4.3.17 del libro di Stefania Gabelli. 
\end{dimostrazione}

\textbf{Lemma:} sia F un campo. Il polinomio \(X^d - 1\) divide il polinomio \(X^n - 1\) s.s.e. d divide n.

\begin{dimostrazione}
    se \(n = qd + r, 0 \leq r \leq d\), in \(\mathbb{F}[X]\) si ha:
    \begin{center}
        \((x^n - 1) = (X^d - 1)(X^{n-d} + X^{n-2d} + ... + x^{n-(p-1)d} + X^r) + (X^r -1).\)\\
    \end{center}
    quindi \(X^d - 1\) divide \(X^n - 1\) s.s.e. \(X^r - 1\) è il polinomio nullo, cioè s.s.e. \(r = 0\)
\end{dimostrazione}

\vspace{\baselineskip}

dalla fattorizzazione nella dimostrazione del lemma otteniamo che, calcolazndo in p,\\
se \(p^d - 1\) divide \(p^n - 1\) allora d divide n.\\

\textbf{Corollario:} \(d\) divide \(n \iff (X^{p^d} - X)\) divide \((X^{p^n} - X)\) in \(\mathbb{F}_p[X]\).

\begin{dimostrazione}
    \(\implies\)\\
    per il lemma precedente, \(X^d - 1\) divide \(X^n - 1\).\\
    calcolando in p si ottiene che \(p^d - 1\) divide \(p^n - 1\).\\
    quindi sempre per il lemma, \(X^{p^{d - 1}} - 1\) divide \(X^{p^n - 1} - 1\).\\
    \(\impliedby\)\\
    viceversa se \(X^{p^{d - 1}} - 1\) divide \(X^{p^n -1 } - 1\), allora \(p^d - 1\) divide \(p^n - 1 \implies d | n\).\\
\end{dimostrazione}

\begin{proposition}
    tutti e soli i sottocampi di \(\mathbb{F}_{p^n}\) sono i campi \(\mathbb{F}_{p^d}\) con \(d | n\).
\end{proposition}

\begin{dimostrazione}
    abbiamo che, se \(\mathbb{F}_{p^d} \subseteq \mathbb{F}_{p^n}\),\\
    allora tutte le radici di \(X^{p^d} - X\) in \(\mathbb{F}_{p^d}\) sono radici di \(X^{p^n} - X\) in \(\mathbb{F}_{p^n}\),\\
    ossia \(X^{p^d} - X\) divide \(X^{p^n} - X \implies d | n\).\\
    se d divide n, \(X^{p^d} - X \) divide \(X^{p^n} - X\)\\
    e l'insieme delle radici di \(X^{p^d} - X\) (è un campo) sta in \(\mathbb{F}_{p^n}\)\\
\end{dimostrazione}

\vspace{\baselineskip}

Finora, dato un numero primo \(p\) e un numero naturale \(n \neq 0\), abbiamo costruito il campo \(\mathbb{F}_{p^n}\)\\
di cardinalità \(p^n\) prendendo un polinomio irriducibile \(Q \in \mathbb{F}_p\) e facendo il quoziente:
\begin{center}
    \(\mathbb{F}_{p^n} = \sfrac{\mathbb{F}_p[X]}{<Q(X)>}\)
\end{center}
Abbiamo visto che due campi costruiti in questo modo sono isomorfi.\\
Facciamo alcune osservazioni e un discorso più generale.\\

\begin{enumerate}
    \item sia \(K\) un campo finito. qual'è la caratteristica di \(K\)?\\
        prendiamo il sottogruppo \(<1_K> \subseteq K\). poiché \(<1_K>\) è finito,\\
        \(<1_K> \simeq \mathbb{Z}_n\) per qualche \(n > 1\).\\
        dato che gli elementi di \(<1_K>\) sono di un campo, non sono divisori dello zero,\\
        quindi \(n\) è primo, ossia \underline{un campo finito ha caratteristica prima \(p\)}\\
        e il suo sottocampo fondamentale è \(\mathbb{F}_p\)\\
    \item sia \(K\) un campo finito. Abbiamo detto nel punto 1. che \(\mathbb{F}_p \subseteq K\) per qualche primo p.\\
        inoltre il gruppo moltiplicativo \(K \setminus \{0\}\) è ciclico e quindi,\\
        come precedentemente dimostrato, \(se K \setminus \{0\} = \alpha, K = \mathbb{F}_p(\alpha)\).\\
        Quindi, se il grado di \(\alpha\) su \(\mathbb{F}_p\) è n, abbiamo che:\\
        \(|K| = p^n\), ossia \underline{ogni campo finito ha cardinalità \(p^n\)}, per qualche \(p\) primo e \(n \neq 0\).\\
    \item siano \(K_1\) e \(K_2\) due campi finiti di cardinalità \(p^n\).\\
        sia \(K_1 = \mathbb{F}_p(\alpha)\) dove \(\alpha\) è un generatore del gruppo \(K_1 \setminus \{0\}\) e ha grado \(n\) su \(K_1\).\\
        sia \(Q \in \mathbb{F}_p[X]\) il suo polinomio minimo. Quindi \(deg(Q) = n\), e \(Q\) è irriducibile.\\
        \begin{enumerate}
            \item \(K_1\) e \(K_2\) sono campi di spezzamento di \(X^{p^n} - X \in \mathbb{F}_p[X]\).
            \item ogni polinomio irriducibile di grado n in \(\mathbb{F}_p[X]\) è fattore di \(X^{p^n} - X\).
            \item da (b) segue che \(Q\) ha una radice in \(K_2\), la chiamiamo \(\beta\).
            \item l'assegnazione \(\alpha \rightarrow \beta\) definisce un mofismo di campi da \(K_1\) in \(K_2\).\\
                poiché un morfismo tra campi è sempre iniettivo, ed essendo anche suriettivo,
                perché \(K_1\) e \(K_2\) hanno la stessa cardinalità, è un isomorfismo:
                \begin{center}
                    \(K_1 \simeq K_2\)
                \end{center}
        \end{enumerate}
\end{enumerate}

\newpage

\subsection{Algoritmo di Berlekamp}

\begin{teorema}
    sia \(f(x) \in \mathbb{F}_p[x]\) di grado \(d > 1\), sia \(h(x) \in \mathbb{F}_p[x]\) di grado \(1 < deg(h) < d\)\\
    tale che \(f(x)\) divide \(h(x)^p - h(x)\).\\
    allora
    \begin{center}
        \(f(x) = MCD\{f(x), h(x)\} \cdot MCD\{f(x), h(x) - 1\} \cdot ... \cdot MCD\{f(x), h(x) - (p - 1)\}\)
    \end{center}
    è una fattorizzazione non banale di \(f(x)\) in \(\mathbb{F}_p[x]\).
\end{teorema}

\begin{dimostrazione}
    supponiamo che \(f(x)\) divida \(h(x)^p - h(x)\). il polinomio \(X^p - X \in \mathbb{F}_p[X]\) si fattorizza come:
    \begin{center}
        \(X^p - X = X(X - 1)(X - 2)...(X - (p - 1))\)
    \end{center}
    mettendo \(h(x)\) al posto di \(X\) si ha:
    \begin{center}
        \(h(x)^p - h(x) = h(x)(h(x) - 1)(h(x) - 2)...(h(x) - (p - 1))\)
    \end{center}
    abbiamo che \(MCD\{ h(x) - i, h(x) - j\} = 1 \forall i,j \in \mathbb{F}_p, i \neq j\).\\
    infatti, se \(MCD\{ h(x) - i, h(x) - j\} = D(x)\) allora
    \begin{center}
        \[
        \begin{cases}
            h(x) - i = D(x) \cdot H_i(x)\\
            h(x) - j = D(x) \cdot H_j(x)
        \end{cases}
        \]
        \(\implies D(x)[H_i(x) - H_j(x)] = j - i \in \mathbb{F}_p\)\\
        \(\implies deg(D) = 0, i \neq j\)
    \end{center}
    inoltre, se \( MCD\{ a, b\} = 1\) si ha che \(MCD\{ f, ab\} = MCD\{ f, a\} = MCD\{ f, b\} \).\\
    per induzione si ha che 
    \begin{center}
        \(MCD\{ f, a_1 \cdot ... \cdot a_k\} = MCD\{ f, a_1\} \cdot ... \cdot MCD\{ f, a_k\}\)
    \end{center}
    dato che \(f(x)\) divide \(h(x)^p - h(x)\), abbiamo che
    \begin{center}
        \(f(x) = MCD\{ f(x), h(x)^p - h(x)\}\)
    \end{center}
    poiché, se \(i \neq j\), \(MCD\{ h(x) - i, h(x) - j\} = 1\), si ha\\
    \begin{center}
        \(f(x) = MCD\{ f(x), h(x)^p - h(x)\} = MCD\{ f(x), h(x)[h(x) - 1] \cdot ... \cdot [h(x) - p + 1]\} =\)\\
        \(= MCD\{ f, h\} \cdot MCD\{ f, h - 1\} \cdot ... \cdot MCD\{ f, h - p + 1\}\).
    \end{center}
    poiché \(deg(h - i) < deg(f), MCD\{f, h - i\} \neq f(x), \forall i \in \mathbb{F}_p\).\\
    quindi nella fattoriazzazione precedente appaiono solo polinomi di grado \(< d\),\\
    perciò è non banale.
\end{dimostrazione}

\newpage

\textbf{Proposizione:} Un polinomio \(h(x) \in \mathbb{F}_p[x]\) che soddisfa le condizioni del teorema esiste sempre.\\
\begin{dimostrazione}
    Sia
    \begin{center}
        \(h(x) = b_0 + b_1 x + ... + b_{d - 1} x^{d - 1} \in \mathbb{F}_p[X]\)
    \end{center}
    allora
    \begin{center}
        \(h(x)^p = b_0^p + b_1^p x + ... + b_{d - 1}^p x^{p(d - 1)}\)\\
        (avendo dimostrato che \((X + Y)^p = x^p + Y^p\) e induttivamente che \((\sum_{i=1}^{k} x_i)^p = \sum_{i=1}^{k} x_i^p\))
    \end{center}
    ma
    \begin{center}
        \(b_i^p = b_i \forall 0 \leq i \leq d - 1\) quindi \(h(x)^p = b_0 + b_1 x^p + ... + b_{d - 1} x^{p(d - 1)}\)
    \end{center}
    si ha che
    \begin{center}
        \(h(x)^p \mod f(x) = b_0 (\mod f) + b_1 (x^p \mod f) + ... + b_{d - 1} (x^{p(d - 1)} \mod f)\)
    \end{center}
    sia \(x^{ip} = f(x) q_i(x) + r_i(x)\) con \(deg(r_i) < d, 0 \leq i \leq d - 1\).\\
    abbiamo che
    \begin{center}
        \([h(x)^p - h(x)] \mod f = 0 \mod f \iff h(x)^p \mod f = h(x) \mod f\)\\
        \(\iff b_0 r_0(x) + b_1 r_1(x) + ... + b_{d - 1} r_{d - 1}(x) = b_0 + b_1 x + ... + b_{d - 1} x^{d - 1}\).
    \end{center}
    otteniamo così un sistema lineare di \(d\) equazioni nelle incognite \(b_i\).\\
    dobbiamo mostrare che esistono soluzioni non nulle.\\
    sia \(f(x) = p_1(x) ... p_k(x)\) una fattoriazzazione di \(f(x) \in \mathbb{F}_p[x]\) in fattori irriducibili.\\
    supponiamo che \(f\) non habbia fattori multipli (verificabile con Teorema seguente).\\
\end{dimostrazione}

\begin{teorema}
    sia K un campo.
    \begin{enumerate}
        \item se \(f(x) \in K[x]\) è ha un fattore multiplo, allora \(MCD\{f, f'\} \neq 1\)\\
            dove \(f'\) è la derivata di \(f\) rispetto a \(x\).
        \item se K ha caratteristica 0 o \(p\), e \(MCD\{f, f'\} \neq 1\), allora \(f(x)\) ha un fattore multiplo.
    \end{enumerate}
\end{teorema}

abbiamo una versione in \(\mathbb{F}_p[x]\) del teorema cinese dei resti.\\
\begin{center}
    \( MCD\{ p_i(x), p_j(x)\} = 1, \forall \ \leq i \leq k, 1 \leq j \leq k, i \neq j\)\\
    \(\sfrac{\mathbb{F}_p[x]}{<f>} \simeq \sfrac{\mathbb{F}_p[x]}{<p_1(x)} \times ... \times \sfrac{\mathbb{F}_p[x]}{<p_k(x)>}\)
\end{center}
dato \((s_1, ..., s_k) \in \mathbb{F}_p^k\), esisite un unica classe \([h(x)] \in \sfrac{\mathbb{F}_p[x]}{f}\) tale che
\begin{center}
    \[
    \begin{cases}
        [h(x)] = s_1 in  \sfrac{\mathbb{F}_p[x]}{<p_1(x)>}\\
        ...\\
        [h(x)] = s_k in \sfrac{\mathbb{F}_p[x]}{<p_k(x)>}
    \end{cases}
    \]
\end{center}
ossia \(h(x) - s_i\) è divisibile per \(p_i(x), \forall 1 \leq i \leq k\)\\
quindi \(p_i(x)\) divide \(h(x)[h(x) - 1] ... [h(x) - (p - 1)] = h(x)^p - h(x), \forall 1 \leq i \leq k\)\\

\newpage

\begin{example}
    fattorizziamo \(f = x^5 + x^2 + 2x + 1 \in \mathbb{F}_3[x]\).\\
    ferifichiamo che \(MCD\{f, f'\} = MCD\{ x^5 + x^2 + 2x + 1, 2x^4 + 2x + 2\} = 1\)\\
    poi calcoliamo i resti:
    \begin{center}
        \(x^{3(5-1)} = x^{12} \equiv (x^2 + 2) \mod f\)
        \(x^{3\cdot3} = x^9 \equiv (2x^4 + x^3 + x^2 + 2x + 2) \mod f\)
        \(x^{3 \cdot 2} = x^6 \equiv (2x^3 + x^2 + 2x) \mod f\)
        \(x^3 \equiv x^3 \mod f\)
        \(1 \equiv 1 \mod f\)
        \(\implies b_0 + b_1 x^3 + b_2(2x^3 + x^2 + 2x) + b_3 (2x^4 + x^3 + x^2 + 2x + 2) + b_4(x^2 + 2) = \)\\
        \(= b_0 + b_1 x + b_2x^2 + b_3 x^3 + b_4 x^4\)\\
        \(\iff\) \[
        \begin{cases}
            2 b_3 + 2 b_4 = 0\\
            2 b_2 + 2 b_3 - b_1 = 0\\
            b_2 + b_3 + b_4 - b_2 = 0\\
            b_1 + 2 b_2 = 0\\
            2 b_3 - b_4 = 0
        \end{cases}
        \iff
        \begin{cases}
            b_3 = 2 b_4\\
            b_1 + b_2 + b_3 = 0\\
            b_1 = b2
        \end{cases}
        \iff
        b_1 = b_2 = b_3 = 2 b_4
        \]
    \end{center}
    una soluzione è dunque \((0, 1, 1, 1, 2)\), ossia \(h(x) =x + x^2 + x^3 +2 x^4\).\\
    quindi
    \begin{center}
        \(f(x) = MCD\{f, x + x^2 + x^3 +2 x^4\} \cdot MCD\{f, 1 + x^2 + x^3 +2 x^4\} \cdot MCD\{f, 2 + x^2 + x^3 + 2x^4\} = \)\\
        \(= (1 + x^2)(x^3 + 2x + 1)\)
    \end{center}
\end{example}

Sia \(f(x) \mathbb{F}_p[x], deg(f) = d\).\\
sia \(f(x) = p_1(x)\cdot ... \cdot p_k(x)\) una fattoriazzazione di \(f(x)\) in fattori irriducibili,\\
non banali e aventi molteplicità 1.\\
siano 
\begin{center}
    \(r_0 = 1 \mod f(x)\)\\
    \(r_1 = x^p \mod f(x)\)\\
    ...\\
    \(r_{d - 1} = x^{p(d - 1)} \mod f(x)\)
\end{center}
con \(deg(r_i) < d \forall 0 \leq i \leq d - 1\)\\
definiamo la matrice \(A \in Mat_{d \times d}(\mathbb{F}_p)\) nel segueente modo:\\
\(A_{ij} = \) coefficiente del termine di grado i del polinomio \(r_j(x)\)\\
\begin{example}
    considerando l'esempio precedente, si ha:\\
    \(A \in Mat_{5 \times 5}(\mathbb{F}_3) = 
    \begin{bmatrix}
    1 & 0 & 0 & 2 & 2 \\
    0 & 0 & 2 & 2 & 0 \\
    0 & 0 & 1 & 1 & 1 \\
    0 & 1 & 2 & 1 & 0 \\
    0 & 0 & 0 & 2 & 0
    \end{bmatrix}\)\\
    la matrice \(A - I\) è la matrice del sistema che abbiamo risolto, ossia \((A - I)\overline{b} = \overline{0}\)
\end{example} 

\newpage

\begin{teorema}
    Il numero di fattori irriducibili k nella fattorizzazione di f \\
    è uguale alla dimensione del nucleo di A - I.\\
    ossia \(k = d - rk(A - I)\), rango calcolato sul campo \(\mathbb{F}_p\).
\end{teorema}

\begin{dimostrazione}
    osserviamo innanzitutto che \(dim(ker(A - I)) \geq 1\).\\
    infatti la d-tupla \((b_0, 0 , ..., 0)\) è sempre soluzione del sistema \(\forall b_0 \in \mathbb{F}_p\).\\
    abbiamo visto che l'Insieme
    \begin{center}
        \(H = \{ h \in \mathbb{F}_p[x] : deg (h) < d, f | h^p - h\}\)
    \end{center}
    è uno spazio vettoriale su \(\mathbb{F}_p\) isomorfo a \(ker(A - I)\).\\
    sia k il numero di fattori irriducibili non banali di \(f\), aventi tutti molteplicità 1.\\
    dimostriamo che \(\mathbb{F}_p^k\) è isomorfo a H.\\
    abbiamo già dimostrato che per ogni \((s_1, ... s_k) \in \mathbb{F}_p^k\) troviamo un unico elemento di H,\\
    usando il Teorema cinese dei resti per l'anello \(\mathbb{F}_p[X]\).\\
    quindi abbiamo definito una funzione \(\varphi : \mathbb{F}_p^k \rightarrow H\)
    \begin{enumerate}
        \item \(\varphi\) è un morfismo si spazi vettoriali.
        \item \(\varphi\) è iniettiva:\\
                \(ker(\varphi) = \{(s_1, ... s_k) \in \mathbb{F}_p^k : s_i \mod p_i = 0, \forall 1 \leq i \leq k\}\)\\
                \(=\{(0, ... , 0)\}\)
        \item \(\varphi\) è suriettiva:\\
                se \(h \in H\), abbiamo visto che \(h^p - h = h (h - 1)(h - 2)...(h - (p - 1))\).\\
                questi fattori sono coprimi a coppie, quindi se \(f | h^p - h\), allora \(p_i(x) | (h - s_i)\)\\
                per un unico \(s_i \in \mathbb{F}_p, \forall 1 \leq i \leq k\).\\
                quindi \(h\) è soluzione del sistema
                \[
                \begin{cases}
                    h \equiv s_1 \mod p_1\\
                    ...\\
                    h \equiv s_k \mod p_k
                \end{cases}
                \]
    \end{enumerate}
    abbiamo dimostrato che \(\varphi : \mathbb{F}_p^k \rightarrow H \) è un isomorfismo do spazi vettoriali, quindi
    \begin{center}
        \(\mathbb{F}_p^k \simeq H \simeq ker(A - I)\)
    \end{center}
    ossia \(dim(ker(A - I)) = k = d - rk(A - I)\).
\end{dimostrazione}

\begin{example}
    sempre considerando l'esempio precedente, si ha che \(2 = 5 - rk(A - I)\)\\
\end{example}

\newpage

se \(f \in \mathbb{F}_p[x]\) ha fattori irriducibili di molteplicità > 1, procediamo come segue:\\
abbiamo che \(D = MCD\{f, f'\} \neq 1\).\\
osserviamo che il polinomio \(\frac{f}{D}\) ha fattori irriducibili tutti di molteplicità 1.\\
infatti se \(p_1, ... , p_k\) sono tutti distinti,\\
\begin{center}
    \(f' = (p_1^{e_1}(x) ... p_k^{e_k}(x))' =\)\\
    \(e_1 p_1 ^ {e_1 - 1} p_1'p_2^{e_2} ... p_k^{e_k} + ... + e_k p_1^{e_1}p_2^{e_2} ... p_k^{e_k - 1}p_k'\)\\
\end{center}
e \(D = p_1^{e_1 - 1} ... p_k^{e_k - 1}\) quindi \(\frac{f}{D} = p_1 ... p_k\)\\
allora fattorizziamo \(\frac{f}{D}\) poi fattorizziamo \(D\), eventualmente ripetendo con \(D, D'\).\\
fincé non otteniamo \(MCD\{ D_i, D_i'\} = 1\).

\begin{example}
    in \(\mathbb{F}_3[x]\) consideriamo il polinomio \(f = 1 + 2x + 2x^2 + x^5 + x^6 + x^7\).\\
    si ha che 
    \begin{center}
        \(f' = 2 + 4x + 5x^4 + 7x^6 = 2 + x + 2x^4 + x^6\).\\
    \end{center}
    e
    \begin{center}
        \(MCD\{f, f'\} = 1 + 2x + x^3 =: D\).\\
        \(\frac{f}{D} = 1 + 2x^2 + x^3 + x^4\), fattorizzando otteniamo \((x + 1)(1 + 2x+x^3)\).\\
    \end{center}
    dato che \(D\) non ha radici in \(\mathbb{F}_3\), \(D\) è irriducibile.\\
    allora
    \begin{center}
        \(f = \frac{f}{D} \cdot D = (x + 1)(1 + 2x + x^3)^2\)
    \end{center}
\end{example}

\newpage

\section{Tensori}
\subsection{Prodotto tra matrici}
\begin{definition}
    prodotto righe per colonne di matrici 2 x 2\\
    sia \(Mat_{2 \times 2}(K) = \{\begin{pmatrix} x_1 & x_2 \\ x_3 & x_4 \end{pmatrix} : x_i \in K \}\) l'insieme delle matrici 2 x 2 a coefficienti in un campo K.\\
\end{definition}
diamo all'insieme una struttura di anello:\\
\begin{itemize}
    \item somma: \(\begin{pmatrix} x_1 & x_2 \\ x_3 & x_4 \end{pmatrix} + \begin{pmatrix} y_1 & y_2 \\ y_3 & y_4 \end{pmatrix} = \begin{pmatrix} x_1 + y_1 & x_2 + y_2 \\ x_3 + y_3 & x_4 + y_4 \end{pmatrix}\)
    \item prodotto: \(\begin{pmatrix} x_1 & x_2 \\ x_3 & x_4 \end{pmatrix} \cdot \begin{pmatrix} y_1 & y_2 \\ y_3 & y_4 \end{pmatrix} = \begin{pmatrix} x_1 y_1 + x_2 y_3 & x_1 y_2 + x_2 y_4 \\ x_3 y_1 + x_4 y_3 & x_3 y_2 + x_4 y_4 \end{pmatrix}\)
\end{itemize}
con queste operazioni \(Mat_{2 \times 2}(K)\) è un anello con unità 
\(\begin{pmatrix} 1_k & 0\\ 0 & 1_k 
\end{pmatrix}\).\\

il prodotto così definito richiede di eseguire 8 moltiplicazioni.\\
abalogamente possiamo dotare \(Mat_{n \times n}(K)\) di una struttura di anello.\\
la moltiplicazione righe per colonne richiede l'esecuzione di \(n^3\) moltiplicazioni.\\

\begin{example}
    in \(Mat_{3 \times 3}(\mathbb{F}_2)\) abbiamo
    \begin{center}
        \(\begin{pmatrix}
            1 & 1 & 1\\
            0 & 1 & 1\\
            1 & 0 & 1
        \end{pmatrix} 
        \cdot
        \begin{pmatrix}
            1 & 0 & 0\\
            1 & 1 & 0\\
            1 & 1 & 1
        \end{pmatrix}
        = 
        \begin{pmatrix}
            1 & 0 & 1\\
            0 & 0 & 1\\
            1 & 1 & 1
        \end{pmatrix}\)
    \end{center}
\end{example}

\begin{definition}
    Algorimo di Strassen per il prodotto di matrici 2 x 2:\\
    sia \(A = \begin{pmatrix} x_1 & x_2 \\ x_3 & x_4 \end{pmatrix}\) e \(B = \begin{pmatrix} y_1 & y_2 \\ y_3 & y_4 \end{pmatrix}\)\\
    e \(AB = \begin{pmatrix} z_1 & z_2 \\ z_3 & z_4 \end{pmatrix}\)\\
    definiamo
    \begin{enumerate}
        \item \((x_1 + x_4)(y_1 + y_4)\)
        \item \((x_3 + x_4)y_1\)
        \item \(x_1(y_2 + y_4)\)
        \item \(x_4(-y_1 + y_3)\)
        \item \((x_1 + x_2)y_4\)
        \item \((-x_1 + x_3)(y_1 + y_2)\)
        \item \((x_2 - x_4)(y_3 + y_4)\)
    \end{enumerate}
    allora \(z_1 = 1 + 4 - 5 + 7, z_2 = 3 + 5, z_3 = 2 + 4, z_4 = 1 + 3 - 2 + 6\)\\
\end{definition}

\begin{example}
    in \(Mat_{2 \times 2}(\mathbb{F}_2)\) siano
    \begin{center}
        \(A = \begin{pmatrix} 1 & 1 \\ 1 & 1 \end{pmatrix}\) e \(B = \begin{pmatrix} 1 & 0 \\ 1 & 1 \end{pmatrix}\)\\
    \end{center}
        allora \(.1 = 0, .2 = 0, .3 = 1, .4 = 0, .5 = 0, .6 = 0, .7 = 0 \)\\
        quindi \(AB = \begin{pmatrix} .1 + .4 - .5 + .7 & .3 + .5 \\ .2 + .5 & .1 + .3 - .2 + .6 \end{pmatrix} = 
        \begin{pmatrix} 0 & 0 \\ 1 & 1 \end{pmatrix}\)
\end{example}

Nell'algoritmo di Strassen per matrici 2 x 2 si eseguono 7 moltiplicazioni.\\
L'algoritmo può anche essere usato ricorsivamente per motiplicare matici più grandi.\\
ad esempio, se \(M, N \in Mat_{4 \times 4}(K)\), possiamo scrivere:
\begin{center}
    \(M = \begin{pmatrix} A & B \\ C & D \end{pmatrix}\) e \(N = \begin{pmatrix} A' & B' \\ C' & D' \end{pmatrix}\)\\
    dove \(A, B, C, D, A', B', C', D' \in Mat_{2 \times 2}(K)\)\\
    poiche \(MN = \begin{pmatrix} AA' + BC' & AB' + BD' \\ CA' + DC' & CB' + DD' \end{pmatrix}\)\\
\end{center}
possiamo usare l'algoritmo in due passi.\\
primo passo:
\begin{enumerate}
    \item \((A + D)(A' + D')\)
    \item \((C + D) A'\)
    \item \(A(B' + D')\)
    \item \(D(-A' + C')\)
    \item \((A + B)D'\)
    \item \((-A + C)(A' + B')\)
    \item \((B - D)(C' + D')\)
\end{enumerate}
e quindi \(MN = \begin{pmatrix}
    .1 + .4 - .5 + .7 & .2 + .5\\
    .3 + .4 & .1 + .3 - .2 + .6
\end{pmatrix}\)\\
\vspace{\baselineskip}secondo passo: calcoliamo il prodotto in (.1, .2, 3, ... , .7) con l'algoritmo di Strassen. 
Quindi l'algoritmo di Strassen per il prodotto tra due matrici 4 x 4 richiede\\
\(7^2\) moltiplicazioni.\\
se vogliamo moltiplicare matrici 3 x 3\\
\vspace{\baselineskip}possiamo aggiungerre una riga e una colonna di zeri e considerarle 4 x 4.\\

In generale la moltiplicazione di due matrici n x n usando l'algoritmo di Strassen richiede\\
\(7^k\) moltiplicazioni se \(n = 2^k\)\\
abbiamo che
\begin{center}
    \(7^k = 2^{\log_2 7^k}=2^{k \log_2 7} \approx 2^{2.81}\)
\end{center}

\begin{definition}
    \textbf{L'esponente w} della moltiplicazione di matrici è\\
    \(w := INF\{ h \in \mathbb{R} : Mat_{n \times n}(K)\) può essere moltiplicato con \(O(n^h)\) operazioni aritmrtiche \(\}\)\\
\end{definition}

\vspace{\baselineskip} L'algoritmo di Strassen mostra che \(w \leq 2.81\).\\
Se n = 2 l'algoritmo di Strassen è ottimale (dal Teorema di Brockett- Dobkin).\\
Non è noto un algoritmo ottimale per la moltiplicazione matrici 3 x 3.\\
Nel 2022 è stato pubblicato un algoritmo trovato da Alphatensor per matrici 4 x 4 su \(\mathbb{F}_2\)\\
che richiede l'esecuzione di 47 moltiplicazioni, l'algoritmo di Strassen ne richiederebbe 49.\\
\end{document}