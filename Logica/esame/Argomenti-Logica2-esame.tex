\documentclass[10pt,a4paper,twoside]{book}
\usepackage[italian]{babel}

% Modified version of https://github.dev/polimi-cheatsheet/MIDA2/style.tex

% Fonts and language
\usepackage[T1]{fontenc}
\usepackage[utf8]{inputenc}

\usepackage{amsmath, amssymb, amsthm} % amssymb also loads amsfonts
\usepackage{latexsym}

\usepackage{booktabs}
\usepackage{pgfplots}
\usepackage{tikz}
\usepackage{mathdots}
\usepackage{cancel}
\usepackage{color}
\usepackage{siunitx}
\usepackage{array}
\usepackage{multirow}
\usepackage{makecell}
\usepackage{tabularx}
\usepackage{caption}
\captionsetup{belowskip=12pt,aboveskip=4pt}
\usepackage{subcaption}
\usepackage{placeins} % for  \FloatBarrier
\usepackage{flafter}  % The flafter package ensures that floats don't appear until after they appear in the code.
\usepackage[shortlabels]{enumitem}
\usepackage[english]{varioref}
\renewcommand{\ref}{\vref}

\usepackage{import}
\usepackage{pdfpages}
\pgfplotsset{compat=1.18} % To avoid the warning: "Package pgfplots Warning: running in backwards compatibility mode (unsuitable tick labels; missing features). Consider writing \pgfplotsset{compat=1.18} into your preamble."
\usepackage{transparent}
\usepackage{xcolor}
\usepackage{graphicx}
\graphicspath{ {./images/} } % Path relative to the main .tex file
\usepackage{float}

\newcommand{\fg}[3][\relax]{%
    \begin{figure}[H]%[htp]%
        \centering
        \captionsetup{width=0.7\textwidth}
        \includegraphics[width = #2\textwidth]{#3}%
        \ifx\relax#1\else\caption{#1}\fi
        \label{#3}
    \end{figure}%
    \FloatBarrier%
}

\usepackage[none]{hyphenat} % no hyphenation

\emergencystretch 3em % to prevent the text from going beyond margins

\usepackage[skip=0.2\baselineskip+2pt]{parskip}

% Headers and footers
\usepackage{fancyhdr}

\fancypagestyle{toc}{
    \fancyhf{}
    \fancyfoot[C]{\thepage}
    \renewcommand{\headrulewidth}{0pt}%
    \renewcommand{\footrulewidth}{0pt}
}

\fancypagestyle{fancy}{%
    \fancyhf{}%
    \fancyhead[RE]{\nouppercase{\leftmark}}%
    \fancyhead[LO]{\nouppercase{\rightmark}}%
    \fancyhead[LE,RO]{\thepage}%
    \renewcommand{\footrulewidth}{0pt}%
    \renewcommand{\headrulewidth}{0.4pt}
}

% Removes the header from odd empty pages at the end of chapters
\makeatletter
\renewcommand{\cleardoublepage}{
    \clearpage\ifodd\c@page\else
        \hbox{}
        \vspace*{\fill}
        \thispagestyle{empty}
        \newpage
    \fi}


% Custom Stuff
\usepackage{xspace}
\newcommand{\latex}{\LaTeX\xspace}

\newcommand{\notimplies}{\mathrel{{\ooalign{\hidewidth$\not\phantom{=}$\hidewidth\cr$\implies$}}}}


\renewcommand{\emptyset}{\varnothing}

\usepackage{mathtools}
\DeclarePairedDelimiter{\abs}{\lvert}{\rvert} % Absolute value

% Space after \exists and \forall: https://tex.stackexchange.com/questions/438612/space-between-exists-and-forall
\let\oldforall\forall
\renewcommand{\forall}{\oldforall \, }
\let\oldexist\exists
\renewcommand{\exists}{\oldexist \: }
% Exists unique
\newcommand\existu{\oldexist! \: }

% Appendice
\usepackage[title,titletoc]{appendix}

% Theorems
\definecolor{grey245}{RGB}{245,245,245}

\newtheoremstyle{blacknumbox} % Theorem style name
{0pt}% Space above
{0pt}% Space below
{\normalfont}% Body font
{}% Indent amount
{\bf\scshape}% Theorem head font --- {\small\bf}
{.\;}% Punctuation after theorem head
{0.25em}% Space after theorem head
{\small\thmname{#1}\nobreakspace\thmnumber{\@ifnotempty{#1}{}\@upn{#2}}% Theorem text (e.g. Theorem 2.1)
    %{\small\thmname{#1}% Theorem text (e.g. Theorem)
    \thmnote{\nobreakspace\the\thm@notefont\normalfont\bfseries---\nobreakspace#3}}% Optional theorem note

\newtheoremstyle{unnumbered} % Theorem style name
{0pt}% Space above
{0pt}% Space below
{\normalfont}% Body font
{}% Indent amount
{\bf\scshape}% Theorem head font --- {\small\bf}
{.\;}% Punctuation after theorem head
{0.25em}% Space after theorem head
{\small\thmname{#1}\thmnumber{\@ifnotempty{#1}{}\@upn{#2}}% Theorem text (e.g. Theorem 2.1)
    %{\small\thmname{#1}% Theorem text (e.g. Theorem)
    \thmnote{\nobreakspace\the\thm@notefont\normalfont\bfseries---\nobreakspace#3}}% Optional theorem note

\newcounter{dummy}
\numberwithin{dummy}{chapter}

\theoremstyle{blacknumbox}
\iflanguage{italian}{
    \newtheorem{definitionT}[dummy]{Definizione}
    \newtheorem{theoremT}[dummy]{Teorema}
    \newtheorem{corollaryT}[dummy]{Corollario}
    \newtheorem{lemmaT}[dummy]{Lemma}
    \newtheorem{propositionT}[dummy]{Proposizione}
}{
    \newtheorem{definitionT}[dummy]{Definition}
    \newtheorem{theoremT}[dummy]{Theorem}
    \newtheorem{corollaryT}[dummy]{Corollary}
    \newtheorem{lemmaT}[dummy]{Lemma}
    \newtheorem{propositionT}[dummy]{Proposition}
}

% Per gli unnumbered tolgo il \nobreakspace subito dopo {\small\thmname{#1} perché altrimenti c'è uno spazio tra Teorema e il ".", lo spazio lo voglio solo se sono numerati per distanziare Teorema e "(2.1)"
\theoremstyle{unnumbered}
\iflanguage{italian}{
    \newtheorem*{remarkT}{Osservazione}
    \newtheorem*{noteT}{Nota}
    \newtheorem*{proofT}{Dimostrazione}
    \newtheorem*{exampleT}{Esempio}
}{
    \newtheorem*{remarkT}{Remark}
    \newtheorem*{noteT}{Note}
    \newtheorem*{proofT}{Proof}
    \newtheorem*{exampleT}{Example}
}

\RequirePackage[framemethod=default]{mdframed} % Required for creating the theorem, definition, exercise and corollary boxes

% orange box
\newmdenv[skipabove=7pt,
    skipbelow=7pt,
    rightline=false,
    leftline=true,
    topline=false,
    bottomline=false,
    linecolor=orange,
    innerleftmargin=5pt,
    innerrightmargin=5pt,
    innertopmargin=5pt,
    leftmargin=0cm,
    rightmargin=0cm,
    linewidth=2pt,
    innerbottommargin=5pt]{oBox}

% green box
\newmdenv[skipabove=7pt,
    skipbelow=7pt,
    rightline=false,
    leftline=true,
    topline=false,
    bottomline=false,
    linecolor=green,
    innerleftmargin=5pt,
    innerrightmargin=5pt,
    innertopmargin=5pt,
    leftmargin=0cm,
    rightmargin=0cm,
    linewidth=2pt,
    innerbottommargin=5pt]{gBox}

% blue box
\newmdenv[skipabove=7pt,
    skipbelow=7pt,
    rightline=false,
    leftline=true,
    topline=false,
    bottomline=false,
    linecolor=blue,
    innerleftmargin=5pt,
    innerrightmargin=5pt,
    innertopmargin=5pt,
    leftmargin=0cm,
    rightmargin=0cm,
    linewidth=2pt,
    innerbottommargin=5pt]{bBox}

% red box
\newmdenv[skipabove=7pt,
    skipbelow=7pt,
    rightline=false,
    leftline=true,
    topline=false,
    bottomline=false,
    linecolor=red,
    innerleftmargin=5pt,
    innerrightmargin=5pt,
    innertopmargin=5pt,
    leftmargin=0cm,
    rightmargin=0cm,
    linewidth=2pt,
    innerbottommargin=5pt]{rBox}

% dim box
\newmdenv[skipabove=7pt,
    skipbelow=7pt,
    rightline=false,
    leftline=true,
    topline=false,
    bottomline=false,
    linecolor=black,
    innerleftmargin=5pt,
    innerrightmargin=5pt,
    innertopmargin=5pt,
    leftmargin=0cm,
    rightmargin=0cm,
    linewidth=2pt,
    innerbottommargin=5pt]{blackBox}

\newenvironment{definition}{\begin{bBox}\begin{definitionT}}{\end{definitionT}\end{bBox}}
\newenvironment{theorem}{\begin{rBox}\begin{theoremT}}{\end{theoremT}\end{rBox}}
\newenvironment{corollary}{\begin{oBox}\begin{corollaryT}}{\end{corollaryT}\end{oBox}}
\newenvironment{lemma}{\begin{oBox}\begin{lemmaT}}{\end{lemmaT}\end{oBox}}
\newenvironment{proposition}{\begin{oBox}\begin{propositionT}}{\end{propositionT}\end{oBox}}
\newenvironment{remark}{\begin{oBox}\begin{remarkT}}{\end{remarkT}\end{oBox}}
\newenvironment{note}{\begin{noteT}}{\end{noteT}}
\newenvironment{example}{\begin{gBox}\begin{exampleT}}{\end{exampleT}\end{gBox}}

\renewcommand{\qedsymbol}{$\blacksquare$}
\renewenvironment{proof}{\begin{blackBox}\begin{proofT}}{\hfill\qed\end{proofT}\end{blackBox}}

% Contents
\setcounter{secnumdepth}{3} % \subsubsection is level 3
\setcounter{tocdepth}{2}

\usepackage{bookmark}% loads hyperref too
\hypersetup{
    %pdftitle={Fundamentos de C\'alculo},
    %pdfsubject={C\'alculo diferencial},
    bookmarksnumbered=true,
    bookmarksopen=true,
    bookmarksopenlevel=1,
    hidelinks,% remove border and color
    pdfstartview=Fit, % Fits the page to the window.
    pdfpagemode=UseOutlines, %Determines how the file is opening in Acrobat; the possibilities are UseNone, UseThumbs (show thumbnails), UseOutlines (show bookmarks), FullScreen, UseOC (PDF 1.5), and UseAttachments (PDF 1.6). If no mode if explicitly chosen, but the bookmarks option is set, UseOutlines is used.
}

\usepackage{glossaries} % certain packages that must be loaded before glossaries, if they are required: hyperref, babel, polyglossia, inputenc and fontenc
\setacronymstyle{long-short}

% hide section from the ToC \tocless\section{hide}
\newcommand{\nocontentsline}[3]{}
\newcommand{\tocless}[2]{\bgroup\let\addcontentsline=\nocontentsline#1{#2}\egroup}

\usepackage[textsize=tiny, textwidth=1.5cm]{todonotes} % add disable to options to not show in pdf


\usepackage[
    left=2.5cm,
    right=2.5cm,
    top=2.5cm,
    bottom=3cm,
]{geometry}

\usepackage{subfiles}
\usepackage{nicefrac}
\usepackage{tikz-cd}
\usepackage{multicol}

\newcommand{\RomanNumber}[1]{
  \textup{\uppercase\expandafter{\romannumeral#1}}
}

\begin{document}

\frontmatter

\pagestyle{empty}

% Title page
\hypertarget{mytitlepage}{}
\bookmark[dest=mytitlepage,level=chapter]{Title Page}

\vspace*{\fill}
\begin{center}

    {\Huge
        \textsc{Argomenti d'esame}}\\
    \vspace*{0.3cm}
    {\Huge
        \textsc{Logica e Algebra 2 }}\\

    \vspace*{1cm}


    {\large Dagli appunti del professor Paolo Sentinelli}\\

    \vspace*{1cm}

    Autori\\
    \vspace*{0.1cm}
    \textsc{Pietro Pizzoccheri}\\
    \textsc{Lorenzo Bardelli}\\

    \vspace*{0.4cm}

    Document formatting by\\
    \vspace*{0.1cm}
    \textsc{Luca Zani}\\

    \vspace*{1cm}

    Politecnico di Milano\\A.Y. 2024/2025
\end{center}
\vspace*{\fill}
\clearpage

% Copyright
\hypertarget{mycopyright}{}
\bookmark[dest=mycopyright,level=chapter]{Copyright Page}
\vspace*{\stretch{12}}

\textcopyright \ The authors. Some rights reserved.

This work is licensed under CC BY-NC-SA 4.0.\\
\url{http://creativecommons.org/licenses/by-nc-sa/4.0/}

In particular, without the authors' permission, it is forbidden to make digital or printed copies to sell them.

\vspace*{\stretch{2}}

\textsc{Document created on \today}

\vspace*{\stretch{2}}

\textsc{Developed by:}\\
\textsc{Luca Zani}\\
\textsc{Pietro Pizzoccheri}\\
\textsc{Lorenzo Bardelli}

\vspace*{\stretch{5}}

\clearpage

% Preface

% \hypertarget{mypreface}{}
% \bookmark[dest=mypreface,level=chapter]{Preface}
% \input{firstpages/preface}
% \clearpage

% Contents
\cleardoublepage
\pagestyle{toc}
\hypertarget{mytoc}{}
\bookmark[dest=mytoc,level=chapter]{\contentsname}
\tableofcontents
\cleardoublepage

\pagestyle{fancy}
\mainmatter



\chapter{Campi Finiti e Tensori}

\section{I sottogruppi di un gruppo ciclico sono ciclici}

\begin{theorem}[di struttura per i gruppi ciclici]
    Sia $G$ un gruppo ciclico. Allora ogni sottogruppo di $G$ è ciclico.
\end{theorem}

\begin{proof}
    Sia $g \in G$ tale che $G = \langle g \rangle$. La funzione $\varphi: (\mathbb{Z} , +) \rightarrow G$ definita da $\varphi(g) = g^n, \forall n \in \mathbb{Z}$ è un morfismo suriettivo di gruppi.
    \begin{enumerate}[label=\alph*)]
        \item G è infinito: allora $Ker(f) = \{0\}$ e quindi $\varphi$ è iniettivo. Dunque $\varphi$ è un
              isomorfismo di gruppi. Tutti i sottogruppi di $\mathbb{Z} $ sono ciclici.
        \item G è finito: sia $H \subseteq G$ un sottogruppo. Allora $\varphi^{-1}(H) := \{n \in \mathbb{Z} : \varphi(n) \in H\} \subseteq \mathbb{Z} $ è un sottogruppo di $\mathbb{Z} $, quindi esiste $\varphi^{-1}(H)= \langle k \rangle$ con $k \in \mathbb{N} $.

              La restrizione $\varphi: k \mathbb{Z} \rightarrow H$ è un morfismo suriettivo di gruppi e
              \begin{equation*}
                  \varphi(hk) = \varphi(\underbrace{k+k+\ldots+k}_{h \text{ volte}}) = \varphi(k) \varphi(k) \ldots \varphi(k) = [\varphi(k)]^h \qquad \forall h \in \mathbb{Z}
              \end{equation*}
              Quindi $H = \langle \varphi(k) \rangle$.
    \end{enumerate}
\end{proof}

\newpage

\section{Un anello è un campo se e solo se i suoi ideali sono banali}
\begin{proposition}
    Sia $A$ un anello commutativo e $I \subseteq A$ un ideale. Allora:
    \begin{itemize}
        \item $I = A$ se e solo se $I$ contiene un elemento invertibile
        \item $A$ è un campo sse i suoi unici ideali sono $\langle 0 \rangle$ e $A = \langle 1_A \rangle$
    \end{itemize}
\end{proposition}

\begin{proof}
    \
    \begin{itemize}
        \item se $I = A$ allora $1_A \in I$ e $1_A$ è invertibile.

              Sia $u \in I$ un elemento invertibile.

              Allora $u^{-1} \in A$ e quindi $1_A u u^{-1} \in I$.

              Ne segue che $A = \langle 1_A \rangle \subseteq I$. e quindi $I = A$.
        \item Sia $A$ un campo e sia $I \neq \langle 0 \rangle$. se $ n \in I$ e $x \neq 0$ allora $x$ è invertibile e quindi $I = A$
              per il punto sopra.

              Viceversa, se $\langle 0 \rangle$ e $A$ sono gli unici ideali di $A$,
              e se $x \in A  \backslash \{0\}$, allora $\langle x \rangle = \langle 1_A \rangle$, ossia $ax = 1_A$ per qualche $a \in A$. Quindi $x$ è invertibile.
    \end{itemize}
\end{proof}
\newpage


\section{Teorema di isomorfismo per anelli commutativi}

\begin{theorem}[di isomorfismo per gruppi abeliani]
    Sia $f: G_1 \rightarrow G_2$ un morfismo di gruppi abeliani. Allora esiste un morfismo iniettivo $\varphi : \nicefrac{G_1}{Ker (f)} \rightarrow G_2$ tale che il seguente diagramma è commutativo:
    \begin{equation*}
        \begin{tikzcd}
            G_1 \arrow[r, "f"] \arrow[d, "\pi"'] & G_2 \\
            \nicefrac{G_1}{Ker(f)} \arrow[ru, "\varphi"']
        \end{tikzcd}
    \end{equation*}
    In particolare, $\nicefrac{G_1}{Ker(f)} \simeq Im(f)$.
\end{theorem}

\begin{proof}
    L'assegnazione $[g] \mapsto f(g), \forall g \in  G$, definisce una funzione $\varphi : \nicefrac{G_1}{Ker(\varphi)}
        \rightarrow G_2$.\\
    Infatti, se $g' \sim g$, ossia $[g]  = [g']$, allora $g = g' + h , h \in Ker(f)$.

    Dunque $f(g) = f(g' + h) = f(g') + f(h) = f(g')$. Poiché $f$ è morfismo di gruppi, anche $\varphi$ lo è.

    Inoltre $Ker(f) = \{[g] \in \nicefrac{G}{Ker(f)} : \varphi([g]) = e_2\}=\{[g] \in  \nicefrac{G}{Ker(f)} : f(g) = e_2\} = {[e_1]}$. Quindi $\varphi$ è iniettiva.

    Infine, $ \varphi : \nicefrac{G_1}{Ker(f)} \rightarrow Im(f)$ è un morfismo di gruppi, iniettivo e suriettivo, quindi un isomorfismo.
\end{proof}

\begin{theorem}[di isomorfismo per anelli commutativi]
    Sia $f : A \rightarrow B$ un morfismo di anelli commutativi. Allora esiste un morfismo iniettivo di anelli $\Psi : \nicefrac{A}{Ker(f)} \rightarrow B$ tale che il seguente diagramma è commutativo:
    \begin{equation*}
        \begin{tikzcd}
            A \arrow{r}{f} \arrow[swap]{d}{\pi} & B \\
            \nicefrac{A}{Ker(f)} \arrow[swap]{ur}{\Psi}
        \end{tikzcd}
    \end{equation*}
    in particolare, se $f$ è suriettivo, allora $\Psi$ è un isomorfismo di anelli.
\end{theorem}
\newpage

\section{Insieme degli ideali dell’anello \( \mathbb{Z} \) e dell'anello \( \mathbb{Z}_n \)}

\begin{proposition}
    L'insieme dei sottogruppi di $(\mathbb{Z} ,+)$ è $\{n \mathbb{Z} : n \in \mathbb{N} \} $.
\end{proposition}

\begin{proof}
    Sia $H \subseteq \mathbb{Z} $ un sottogruppo non banale. Sia $k := min (H_{>0})$ dove $H_{>0} := \{h \in H : h > 0\}$. Sia $h \in H_{>0}, h \neq k$.

    Allora $h > k$ e $h = nk + r$, $n \in \mathbb{N} , 0 \leq r <k$.

    Dunque $r = h - nk \in H \rightarrow r =0$ per la minimalità di $k$.
\end{proof}

\begin{corollary}
    L'insieme dei sottogruppi di $\mathbb{Z}_n , n \in \mathbb{N} $ è:
    \begin{equation*}
        \{\langle \overline{m} \rangle : \overline{m} \in \mathbb{Z}_n \}
    \end{equation*}
\end{corollary}

\begin{example}
    Abbiamo già visto che ogni sottogruppo di $(\mathbb{Z} , +)$ è del tipo $n \mathbb{Z} = \{kn : k \in \mathbb{Z} \}$, dove $n \in \mathbb{N}$.

    Inoltre, se $a \in \mathbb{Z} $ e $x \in n\mathbb{Z} $, ossia $x=kn$ per qualche $k \in \mathbb{Z} $, si ha che $ax=akn \in n \mathbb{Z} $.

    Quindi $n \mathbb{Z} $ è un ideale di $\mathbb{Z}, \forall n \in \mathbb{N}  $, e tutti gli ideali di $\mathbb{Z} $ sono di questo tipo.
\end{example}

\newpage

\section{Teorema cinese dei resti}
\begin{theorem}[Teorema cinese dei resti]
    siano $n_1,n_2,\ldots,n_k \in \mathbb{N} \setminus \{0,1\}$ tali che $MCD \{n_i,n_j\} = 1$ per ogni $1 \leq i,j \leq k, i \neq j$.

    Sia $n := n_1 \cdot n_2 \cdot \ldots \cdot n_k$.

    Allora la funzione
    \begin{equation*}
        \Psi : \mathbb{Z}_n \rightarrow \mathbb{Z}_{n_1} \times \mathbb{Z}_{n_2} \times \ldots \times \mathbb{Z}_{n_k}
    \end{equation*}
    che mappa
    \begin{equation*}
        x \bmod n \mapsto (x \bmod n_1, x \bmod n_2, \ldots, x \bmod n_k)
    \end{equation*}
    è un isomorfismo di anelli.
\end{theorem}

\begin{proof}
    vediamo prima di tutto che $\Psi$ è un morfismo di anelli dove $f: \mathbb{Z} \rightarrow \mathbb{Z}_{n_1} \times \mathbb{Z}_{n_2} \times \ldots \times \mathbb{Z}_{n_k} $ è definita da $f(x) = (x \bmod n_1, x \bmod n_2, \ldots, x \bmod n_k) \forall x \in \mathbb{Z}$.
    \begin{itemize}
        \item \begin{flalign*}
                  f(a+b) & = ((a+b) \bmod n_1, \ldots , (a+b) \bmod n_k)                               &  & \\
                         & = (a \bmod n_1 + b \bmod n_1, \ldots , a \bmod n_k + b \bmod n_k)           &  & \\
                         & = (a \bmod n_1, \ldots , a \bmod n_k) + (b \bmod n_1, \ldots , b \bmod n_k) &  & \\
                         & = f(a) + f(b), \forall a,b \in \mathbb{Z}
              \end{flalign*}
        \item $f(1) = (1 \bmod n_1, \ldots , 1 \bmod n_k)$ e $(1 \bmod n_1, \ldots , 1 \bmod n_k)$ è l'unità del prodotto diretto di anelli $\mathbb{Z}_{n_1} \times \mathbb{Z}_{n_2} \times \ldots \times \mathbb{Z}_{n_k}$
        \item \begin{flalign*}
                  f(a \cdot b) & = ((a \cdot b) \bmod n_1, \ldots , (a \cdot b) \bmod n_k)                       &  & \\
                               & = (a \bmod n_1 \cdot b \bmod n_1, \ldots , a \bmod n_k \cdot b \bmod n_k)       &  & \\
                               & = (a \bmod n_1, \ldots , a \bmod n_k) \cdot (b \bmod n_1, \ldots , b \bmod n_k) &  & \\
                               & = f(a) \cdot f(b), \forall a,b \in \mathbb{Z}
              \end{flalign*}
    \end{itemize}
    Ora mostriamo che $f$ è suriettivo:

    sia $(a_1 \bmod n_1, \ldots , a_k \bmod n_k) \in \mathbb{Z}_{n_1} \times \mathbb{Z}_{n_2} \times \ldots \times \mathbb{Z}_{n_k}$.

    Osserviamo che $MCD\{n_i,n_1 n_2 \ldots n_{i-1} n_{i+1} \ldots n_k\} = 1, \forall 1 \leq i \leq k$.

    Quindi abbiamo le identità di Bézout: $c_i n_i + b_i \frac{n}{n_i} = 1$ ossia\\
    $u_i + v_i = 1$ dove $u_i = c_i n_i \in <n_i>$ e $v_i = b_i \frac{n}{n_i} \in <\frac{n}{n_i}>$.

    Definiamo $x := a_1 v_1 + \ldots + a_k v_k$ e abbiamo che $f(x) = (a_1 \bmod n_1, \ldots , a_k \bmod n_k)$. infatti:
    \begin{equation*}
        v_i \bmod n_j =
        \begin{cases}
            0 & \text{se } i \neq j \\
            1 & \text{se } i = j
        \end{cases}\\
    \end{equation*}
    dal teorema di isomorfismo abbiamo che $\nicefrac{\mathbb{Z}}{Ker(f)} \simeq \mathbb{Z}_{n_1} \times \ldots \times \mathbb{Z}_{n_k} $ come anelli. ma abbiamo che $Ker(f) = <n_1> \cap <n_2> \cap \ldots \cap <n_k> = <mcm\{n_1 ,\ldots, n_k\}> = <n_1 n_2 \ldots n_k>$ dato che $n_i$ e $n_j$ sono coprimi $\forall i \neq j$.

    Quindi $\nicefrac{\mathbb{Z}}{Ker(f)} = \nicefrac{\mathbb{Z}}{<n>} = \mathbb{Z}_n $ e l'isomorfismo $\Psi: \mathbb{Z}_n \rightarrow \mathbb{Z}_{n_1} \times \ldots \times \mathbb{Z}_{n_k}$ è quello dell'enunciato del teorema.
\end{proof}
\newpage

\section{Costruzione di un campo finito di cardinalità \(p^n\)}
\begin{proposition}
    sia $K$ un campo e $P(X) \in K[X]$ un poliniomio irriducibile. Allora l'anello quoziente $\nicefrac{K[X]}{\langle P(X)\rangle}$ è un campo.
\end{proposition}
\begin{proof}
    Sia $[f] \in \nicefrac{K[X]}{\langle P(X)\rangle}$ tale che $[p] \neq [0]$ ossia $p(X)$ non divide $f(X)$.

    Dunque $MCD\{f(X),p(X)\} = 1$ perchè $p(X)$ è irriducibile.

    Quindi abbiamo un'identità di Bézout $a(X)f(X) + b(X)p(X) = 1$.

    Ossia $[a(X)] = [f(X)]^{-1}$ in $\nicefrac{K[X]}{\langle P(X)\rangle}$.
\end{proof}

\begin{proposition}
    Tutti e soli i polinomi irriducibili su $\mathbb{F}_p$ di grado $n$ dividono $X^{p^n} - X \in \mathbb{F}_p[X]$.
\end{proposition}

\begin{proof}
    Sia $P(X) \in \mathbb{F}_p[X]$ irriducibile di grado n e sia $K := \nicefrac{\mathbb{F}_p[Y]}{\langle P(Y)\rangle}$.

    Allora $K$ ha $p^n$ elementi che sono le radici di $X^{p^n} - X \in K[X]$.

    Poichè $Y \in K$ è una radice $P(X) \in K[X]$, $P(X)$ e $X^{p^n} - X$ hanno una radice in comune in K, allora per il teorema di Ruffini hanno un fattore comune $X - Y \in K[X]$.

    Quindi, poiché $\mathbb{F}_p \subseteq K$ e $MCD$ in $\mathbb{F}_p = MCD$ in $K[X] \implies P(X), X^{p^n} - X$ hanno $MCD \neq 1$ in $\mathbb{F}_p[X]$.

    Poiché $P(X)$ è irriducibile in $\mathbb{F}_p[X]$, $P(X)$ divide $X^{p^n} - X$.
\end{proof}

Adesso vogliamo costruire un isomorfismo di campi
\begin{equation*}
    f: \nicefrac{\mathbb{F}_p[X]}{\langle P(X)\rangle} \rightarrow \nicefrac{\mathbb{F}_p[X]}{\langle Q(X)\rangle}
\end{equation*}
Dove $P(X), Q(X) \in \mathbb{F}_p[X]$ sono monici irriducibili di grado $n$.

Basta costruire un isomorfismo di anelli.

Infatti un morfismo di anelli che sono campi è iniettivo. Inoltre:
\begin{equation*}
    |\nicefrac{\mathbb{F}_p[X]}{\langle P(X)\rangle}| = |\nicefrac{\mathbb{F}_p[X]}{\langle Q(X)\rangle}| = p^n
\end{equation*}

Quindi tale morfismo è biunivoco, ossia è isomorfismo.

Si ha che, se $y \in \nicefrac{\mathbb{F}_p[Y]}{\langle P(Y)\rangle}$ allora $P(X) \in \mathbb{F}_p[X]$ è il polinomio minimo di $y$ su $\mathbb{F}_p$.

Quindi, se $P(X)$ ha una radice in $\nicefrac{\mathbb{F}_p[Y]}{\langle Q(Y)\rangle}$, possiamo usare la proposizione sull'estensione di morfismi di campi per definire il morfismo f, che sarà un isomorfismo. Infatti $\mathbb{F}_p \subseteq \nicefrac{\mathbb{F}_p[X]}{\langle Q(X)\rangle}$.

Inoltre $\nicefrac{\mathbb{F}_p[X]}{\langle P(X)\rangle} = \mathbb{F}_p([X])$, dove $[X]$ è la classe di $X$ in $\nicefrac{\mathbb{F}_p[X]}{\langle P(X)\rangle}$.

Poiché $\nicefrac{\mathbb{F}_p[Y]}{\langle Q(Y)\rangle}$ è un campo di spezzamento di $X^{p^n} - X$ e $P(X)$ divide $X^{p^n} - X$, allora $P(X)$ si fattorizza in fattori di grado 1 in $\nicefrac{\mathbb{F}_p[Y]}{\langle Q(Y)\rangle}$.

Sia $\beta \in \nicefrac{\mathbb{F}_p[Y]}{\langle Q(Y)\rangle}$ tale che $p(\beta)$ = 0.

Allora l'assegnazione
\begin{equation*}
    c_0 + c_1 x + \ldots + c_{n-1} x^{n-1} \mapsto c_0 + c_1 \beta + \ldots + c_{n-1} \beta^{n-1}
\end{equation*}
definisce un morfismo di anelli
\begin{equation*}
    f: \nicefrac{\mathbb{F}_p[X]}{\langle P(X)\rangle} \rightarrow \nicefrac{\mathbb{F}_p[X]}{\langle Q(X)\rangle}
\end{equation*}

\newpage

\section{Tutti i polinomi irriducibili di grado \( n \) a
coefficienti in \(\mathbb{F}_p \) sono fattori di
\( X^{p^n} - X  \in \mathbb{F}_p [X]\)}

\begin{proposition}
    Tutti e soli i polinomi irriducibili su $\mathbb{F}_p$ di grado $n$ dividono $X^{p^n} - X \in \mathbb{F}_p[X]$.
\end{proposition}

\begin{proof}
    Sia $P(X) \in \mathbb{F}_p[X]$ irriducibile di grado n e sia $K := \nicefrac{\mathbb{F}_p[Y]}{\langle P(Y)\rangle}$.

    Allora $K$ ha $p^n$ elementi che sono le radici di $X^{p^n} - X \in K[X]$.

    Poichè $Y \in K$ è una radice $P(X) \in K[X]$, $P(X)$ e $X^{p^n} - X$ hanno una radice in comune in K, allora per il teorema di Ruffini hanno un fattore comune $X - Y \in K[X]$.

    Quindi, poiché $\mathbb{F}_p \subseteq K$ e $MCD$ in $\mathbb{F}_p = MCD$ in $K[X] \implies P(X), X^{p^n} - X$ hanno $MCD \neq 1$ in $\mathbb{F}_p[X]$.

    Poiché $P(X)$ è irriducibile in $\mathbb{F}_p[X]$, $P(X)$ divide $X^{p^n} - X$.
\end{proof}

\newpage

\section{Sottocampi di un campo finito}
\begin{lemma}
    sia $F$ un campo. Il polinomio $X^d - 1$ divide il polinomio $X^n - 1$ s.s.e. $d$ divide $n$.
\end{lemma}
\begin{proof}
    Se $n = qd + r, 0 \leq r \leq d$, in $\mathbb{F}[X]$ si ha:
    \begin{equation*}
        (x^n - 1) = (X^d - 1)(X^{n-d} + X^{n-2d} + \ldots + x^{n-(p-1)d} + X^r) + (X^r -1)
    \end{equation*}
    quindi $X^d - 1$ divide $X^n - 1$ s.s.e. $X^r - 1$ è il polinomio nullo, cioè s.s.e. $r = 0$
\end{proof}


\begin{corollary}
    $d$ divide $n \iff (X^{p^d} - X)$ divide $(X^{p^n} - X)$ in $\mathbb{F}_p[X]$.
\end{corollary}
\begin{proof}
    Per il lemma precedente, $X^d - 1$ divide $X^n - 1$.

    Calcolando in p si ottiene che $p^d - 1$ divide $p^n - 1$.

    Quindi sempre per il lemma, $X^{p^{d - 1}} - 1$ divide $X^{p^n - 1} - 1$.

    Viceversa se $X^{p^{d - 1}} - 1$ divide $X^{p^n -1 } - 1$, allora $p^d - 1$ divide $p^n - 1 \implies$ $d$ divide $n$.
\end{proof}

\begin{proposition}
    Tutti e soli i sottocampi di $\mathbb{F}_{p^n}$ sono i campi $\mathbb{F}_{p^d}$ dove $d$ divide $n$.
\end{proposition}

\begin{proof}
    Abbiamo che, se $\mathbb{F}_{p^d} \subseteq \mathbb{F}_{p^n}$, allora tutte le radici di $X^{p^d} - X$ in $\mathbb{F}_{p^d}$ sono radici di $X^{p^n} - X$ in $\mathbb{F}_{p^n}$, ossia $X^{p^d} - X$ divide $X^{p^n} - X \underbrace{\implies}_{\text{corollario}} d$ divide $n$.

    Se $d$ divide $n$, $X^{p^d} - X $ divide $X^{p^n} - X$ e l'insieme delle radici di $X^{p^d} - X$ (è un campo) sta in $\mathbb{F}_{p^n}$.
\end{proof}

\newpage

\section{Algoritmo di Berlekamp}
\begin{theorem}
    Sia $f(x) \in \mathbb{F}_p[x]$ di grado $d > 1$, sia $h(x) \in \mathbb{F}_p[x]$ di grado $1 < deg(h) < d$ tale che $f(x)$ divide $h(x)^p - h(x)$. allora:
    \begin{equation*}
        f(x) = MCD\{f(x), h(x)\} \cdot MCD\{f(x), h(x) - 1\} \cdot \ldots \cdot MCD\{f(x), h(x) - (p - 1)\}
    \end{equation*}
    è una fattorizzazione non banale di $f(x)$ in $\mathbb{F}_p[x]$.
\end{theorem}
\begin{proof}
    Supponiamo che $f(x)$ divida $h(x)^p - h(x)$. il polinomio $X^p - X \in \mathbb{F}_p[X]$ si fattorizza come:
    \begin{equation*}
        X^p - X = X(X - 1)(X - 2)\ldots(X - (p - 1))
    \end{equation*}
    mettendo $h(x)$ al posto di $X$ si ha:
    \begin{equation*}
        h(x)^p - h(x) = h(x)[h(x) - 1][h(x) - 2]\ldots[h(x) - (p - 1)]
    \end{equation*}
    Abbiamo che $MCD\{ h(x) - i, h(x) - j\} = 1 \forall i,j \in \mathbb{F}_p, i \neq j$.

    Infatti, se $MCD\{ h(x) - i, h(x) - j\} = D(x)$ allora
    \begin{equation*}
        \begin{cases}
            h(x) - i = D(x) \cdot H_i(x) \\
            h(x) - j = D(x) \cdot H_j(x)
        \end{cases}
        \implies D(x)[H_i(x) - H_j(x)] = j - i \in \mathbb{F}_p
        \implies deg(D) = 0, i \neq j
    \end{equation*}
    inoltre, se $ MCD\{ a, b\} = 1$ si ha che $MCD\{ f, ab\} = MCD\{ f, a\} = MCD\{ f, b\} $. Per induzione si ha che
    \begin{equation*}
        MCD\{ f, a_1 \cdot \ldots \cdot a_k\} = MCD\{ f, a_1\} \cdot \ldots \cdot MCD\{ f, a_k\}
    \end{equation*}
    dato che $f(x)$ divide $h(x)^p - h(x)$, abbiamo che
    \begin{equation*}
        f(x) = MCD\{ f(x), h(x)^p - h(x)\}
    \end{equation*}
    poiché, se $i \neq j$, $MCD\{ h(x) - i, h(x) - j\} = 1$, si ha
    \begin{align*}
        f(x) & = MCD\{ f(x), h(x)^p - h(x)\} =                                               \\
             & = MCD\{ f(x), h(x)[h(x) - 1] \cdot \ldots \cdot [h(x) - p + 1]\} =            \\
             & = MCD\{ f, h\} \cdot MCD\{ f, h - 1\} \cdot \ldots \cdot MCD\{ f, h - p + 1\}
    \end{align*}
    Poiché $deg(h - i) < deg(f), MCD\{f, h - i\} \neq f(x), \forall i \in \mathbb{F}_p$.

    Quindi nella fattoriazzazione precedente appaiono solo polinomi di grado $< d$, perciò è non banale.
\end{proof}

\begin{proposition}
    Un polinomio $h(x) \in \mathbb{F}_p[x]$ che soddisfa le condizioni del teorema esiste sempre.
\end{proposition}
\begin{proof}
    Sia
    \begin{equation*}
        h(x) = b_0 + b_1 x + \ldots + b_{d - 1} x^{d - 1} \in \mathbb{F}_p[X]
    \end{equation*}
    allora
    \begin{equation*}
        h(x)^p = b_0^p + b_1^p x + \ldots + b_{d - 1}^p x^{p(d - 1)}
    \end{equation*}
    (avendo dimostrato che $(X + Y)^p = x^p + Y^p$ e induttivamente che $(\sum_{i=1}^{k} x_i)^p = \sum_{i=1}^{k} x_i^p$), ma
    \begin{center}
        $b_i^p = b_i \forall 0 \leq i \leq d - 1$ quindi $h(x)^p = b_0 + b_1 x^p + \ldots + b_{d - 1} x^{p(d - 1)}$
    \end{center}
    si ha che
    \begin{equation*}
        h(x)^p \bmod f(x) = b_0 (\bmod f) + b_1 (x^p \bmod f) + \ldots + b_{d - 1} (x^{p(d - 1)} \bmod f)
    \end{equation*}
    Sia $x^{ip} = f(x) q_i(x) + r_i(x)$ con $deg(r_i) < d, 0 \leq i \leq d - 1$. Abbiamo che
    \begin{align*}
        [h(x)^p - h(x)] & \bmod f = 0 \bmod f                                                                                         \\
                        & \iff h(x)^p \bmod f = h(x) \bmod f                                                                          \\
                        & \iff b_0 r_0(x) + b_1 r_1(x) + \ldots + b_{d - 1} r_{d - 1}(x) = b_0 + b_1 x + \ldots + b_{d - 1} x^{d - 1}
    \end{align*}
    Otteniamo così un sistema lineare di $d$ equazioni nelle incognite $b_0, b_1, \ldots, b_{d-1}$.

    Dobbiamo mostrare che esistono soluzioni non nulle.

    Sia $f(x) = p_1(x) \ldots p_k(x)$ una fattoriazzazione di $f(x) \in \mathbb{F}_p[x]$ in fattori irriducibili.

    Supponiamo che $f$ non habbia fattori multipli (verificabile con Teorema seguente).
\end{proof}

\begin{theorem}
    sia $K$ un campo.
    \begin{enumerate}[label=\alph*)]
        \item se $f(x) \in K[x]$ è ha un fattore multiplo, allora $MCD\{f, f'\} \neq 1$, dove $f'$ è la derivata di $f$ rispetto a $x$.
        \item se $K$ ha caratteristica 0 o $p$, e $MCD\{f, f'\} \neq 1$, allora $f(x)$ ha un fattore multiplo.
    \end{enumerate}
\end{theorem}
Abbiamo una versione in $\mathbb{F}_p[x]$ del teorema cinese dei resti.
\begin{equation*}
    MCD\{ p_i(x), p_j(x)\} = 1, \forall \ \leq i \leq k, 1 \leq j \leq k, i \neq j
\end{equation*}
\begin{align*}
    \implies \nicefrac{\mathbb{F}_p[x]}{\langle f\rangle} \underbrace{\simeq}_{\text{Isomorfismo di anelli}} \nicefrac{\mathbb{F}_p[x]}{\langle p_1(x)\rangle} \times \ldots \times \nicefrac{\mathbb{F}_p[x]}{\langle p_k(x)\rangle}
\end{align*}

Dato $(s_1, \ldots, s_k) \in \mathbb{F}_p^k$, esiste un'unica classe $[h(x)] \in \nicefrac{\mathbb{F}_p[x]}{\langle f \rangle}$ tale che
\begin{equation*}
    \begin{cases}
        [h(x)] = s_1 \quad \text{in} \quad \nicefrac{\mathbb{F}_p[x]}{\langle p_1(x)\rangle} \\
        \vdots                                                                               \\
        [h(x)] = s_k \quad \text{in} \quad \nicefrac{\mathbb{F}_p[x]}{\langle p_k(x)\rangle}
    \end{cases}
\end{equation*}
ossia $h(x) - s_i$ è divisibile per $p_i(x), \forall 1 \leq i \leq k$.

Quindi $p_i(x)$ divide $h(x)[h(x) - 1] \ldots [h(x) - (p - 1)] = h(x)^p - h(x), \forall 1 \leq i \leq k$.

Ossia $f(x)$ divide $h(x)^p - h(x)$.


Sia $f(x) \mathbb{F}_p[x], deg(f) = d$.

Sia $f(x) = p_1(x)\cdot \ldots \cdot p_k(x)$ una fattoriazzazione di $f(x)$ in fattori irriducibili, non banali (cioè di grado $\geq 1$) e aventi molteplicità $1$. siano
\begin{align*}
    r_0       & = 1 \bmod f(x)            \\
    r_1       & = x^p \bmod f(x)          \\
    \vdots    &                           \\
    r_{d - 1} & = x^{p(d - 1)} \bmod f(x)
\end{align*}
Con $deg(r_i) < d \forall 0 \leq i \leq d - 1$.

Definiamo la matrice $A \in Mat_{d \times d}(\mathbb{F}_p)$ nel segueente modo:

$A_{ij} = $ coefficiente del termine di grado i del polinomio $r_j(x)$
\begin{example}
    Considerando l'esempio precedente, si ha:
    \begin{equation*}
        A \in Mat_{5 \times 5}(\mathbb{F}_3) =
        \begin{bmatrix}
            1 & 0 & 0 & 2 & 2 \\
            0 & 0 & 2 & 2 & 0 \\
            0 & 0 & 1 & 1 & 1 \\
            0 & 1 & 2 & 1 & 0 \\
            0 & 0 & 0 & 2 & 0
        \end{bmatrix}
    \end{equation*}
    la matrice $A - I$ è la matrice del sistema che abbiamo risolto, ossia $(A - I)\overrightarrow{b} = \overrightarrow{0}$
\end{example}

\begin{theorem}
    Il numero di fattori irriducibili $k$ nella fattorizzazione di $f$ è uguale alla dimensione del nucleo di $A - I$. Ossia:
    \begin{equation*}
        k = d - rk(A - I)
    \end{equation*}
    (dove il rango è calcolato sul campo $\mathbb{F}_p$).
\end{theorem}

\begin{proof}
    Osserviamo innanzitutto che $dim(Ker(A - I)) \geq 1$.

    Infatti la d-tupla $(b_0, 0 , \ldots, 0)$ è sempre soluzione del sistema $\forall b_0 \in \mathbb{F}_p$.

    Abbiamo visto che l'Insieme
    \begin{equation*}
        H = \{ h \in \mathbb{F}_p[x] : deg (h) < d, f \mid h^p - h\}
    \end{equation*}
    è uno spazio vettoriale su $\mathbb{F}_p$ isomorfo a $Ker(A - I)$.

    Sia $k$ il numero di fattori irriducibili non banali di $f$, aventi tutti molteplicità 1.

    Dimostriamo che $\mathbb{F}_p^k$ è isomorfo a $H$.

    Abbiamo già dimostrato che per ogni $(s_1, \ldots s_k) \in \mathbb{F}_p^k$ troviamo un unico elemento di $H$, usando il Teorema cinese dei resti per l'anello $\mathbb{F}_p[X]$.

    Quindi abbiamo definito una funzione $\varphi : \mathbb{F}_p^k \rightarrow H$
    \begin{enumerate}[label=\alph*)]
        \item $\varphi$ è un morfismo si spazi vettoriali.
        \item $\varphi$ è iniettiva:
              \begin{align*}
                  Ker(\varphi) & = \{(s_1, \ldots s_k) \in \mathbb{F}_p^k : s_i \bmod p_i = 0, \forall 1 \leq i \leq k\} \\
                               & =\{(0, \ldots , 0)\}
              \end{align*}

        \item $\varphi$ è suriettiva:

              Se $h \in H$, abbiamo visto che $h^p - h = h (h - 1)(h - 2)\ldots(h - (p - 1))$.

              Questi fattori sono coprimi a coppie, quindi se $f | h^p - h$, allora $p_i(x) | (h - s_i)$ per un unico $s_i \in \mathbb{F}_p, \forall 1 \leq i \leq k$.

              Quindi $h$ è soluzione del sistema
              \begin{equation*}
                  \begin{cases}
                      h \equiv s_1 \bmod p_1 \\
                      \vdots                 \\
                      h \equiv s_k \bmod p_k
                  \end{cases}
              \end{equation*}
    \end{enumerate}
    Abbiamo dimostrato che $\varphi : \mathbb{F}_p^k \rightarrow H $ è un isomorfismo do spazi vettoriali, quindi
    \begin{center}
        $\mathbb{F}_p^k \simeq H \simeq Ker(A - I)$
    \end{center}
    ossia $dim(Ker(A - I)) = k = d - rk(A - I)$.
\end{proof}

\begin{example}
    Sempre considerando l'esempio precedente, si ha che
    \begin{equation*}
        \underbrace{2}_{\text{fattori irriducibili di $f(x)$}} = \underbrace{5}_{\text{grado di $f(x)$}} - rk(A - I)
    \end{equation*}
\end{example}

Se $f \in \mathbb{F}_p[x]$ ha fattori irriducibili di molteplicità > 1, procediamo come segue:

Abbiamo che $D = MCD\{f, f'\} \neq 1$.

Osserviamo che il polinomio $\frac{f}{D}$ ha fattori irriducibili tutti di molteplicità 1. Infatti se $p_1, \ldots , p_k$ sono tutti distinti
\begin{align*}
    f' & = (p_1^{e_1}(x) \ldots p_k^{e_k}(x))' =                                                                     \\
       & e_1 p_1^{e_1 - 1} p_1'p_2^{e_2} \ldots p_k^{e_k} + \ldots + e_k p_1^{e_1}p_2^{e_2} \ldots p_k^{e_k - 1}p_k'
\end{align*}
e $D = p_1^{e_1 - 1} \ldots p_k^{e_k - 1}$ quindi $\frac{f}{D} = p_1 \ldots p_k$.

Allora fattorizziamo $\frac{f}{D}$ poi fattorizziamo $D$, eventualmente ripetendo con $D, D'$.

Finché non otteniamo $MCD\{ D_i, D_i'\} = 1$.

\newpage

\section{Rango di un tensore}

Ogni elemento di $V_1 \otimes V_2 \otimes \ldots \otimes V_h$ si scrive come combinazione lineare di tensori di rango 1.

Infatti la base $\{v_{j_1}^1 \otimes v_{j_2}^2 \otimes \ldots \otimes v_{j_h}^h\}$ è costituita da tensori di rango 1.
\begin{definition}[Rango di un tensore]
    Sia $T \in V_1 \otimes V_2 \otimes \ldots \otimes V_k$.\\
    definiamo \textbf{rango di T} e lo indichiamo \textbf{rk(T)} il minimo $r \in \mathbb{N}$ tale che:
    \begin{equation*}
        T = \sum_{i=1}^{r}T_i
    \end{equation*}
    dove $T_i \in V_1 \otimes V_2 \otimes \ldots \otimes V_k$ sono di rango 1 $\forall 1 \leq i \leq r$.
\end{definition}

\begin{example}
    Sia $U$ con base $\{u_1, u_2\}$, $V$ con base $\{v_1, v_2\}$ e $W$ con base $\{w_1, w_2\}$.
    \begin{itemize}
        \item $T: u_1 \otimes v_1 \otimes w_1 + u_1 \otimes v_2 \otimes w_1  \in U \otimes V  \otimes W$\\
              ha rango 1. infatti $T = u_1 \otimes (v_1 + v_2 )\otimes w_1$.
        \item $T: u_1 \otimes v_1 \otimes w_1 + u_2 \otimes v_2 \otimes w_1$ ha rango 2.
              Infatti l'unica fattorizzazione possibile è $T = (u_1 \otimes v_1  + u_2 \otimes v_2) \otimes w_1$ che non è un tensore di rango 1.
        \item $T = u_1 \otimes v_1 \otimes w_1 + u_2 \otimes v_2 \otimes w_2 \in U \otimes V \otimes W $ ha rango 2.
    \end{itemize}
\end{example}

Poiché $dim(\otimes_{i=1}^{h} V_i) = \prod_{i=1}^{h} dim(V_i)$, abbiamo che, se $T \in \otimes_{i=1}^{h} V_i$ allora $rk(T) \leq \prod_{i=1}^{h} dim(V_i)$, poiché $\otimes_{i=1}^{h} V_i$ ha una base fatta di tensori di rango 1.

Ora verifichiamo che la nozione di rango di un Tensore è coerente con quella di rango di una matrice interpretando una matrice come forma bilineare, e quindi come un tensore.

Vediamo subito che una matrice di rango 1 corrisponde ad un tensore di rango 1.

Una matrice $m \times n$ di rango 1 h come colonne multipli di un vettore $v \in K^m \setminus \{0\}$.

La prima colonna sia $a_1v$, la seconda $a_2v, \ldots , a_nv, a_i \in K$.

Quindi tale matrice di rango 1 si scrive come
\begin{equation*}
    A = \begin{pmatrix}
        v_1    \\
        v_2    \\
        \vdots \\
        v_m
    \end{pmatrix}
    \begin{pmatrix}
        a_1 & a_2 & \ldots & a_n
    \end{pmatrix} = \overrightarrow{v} \overrightarrow{a}^T
\end{equation*}
Come forma bilineare è il seguente elemento di $(K^m)^* \otimes (K^n)^*$:
\begin{align*}
     & v_1 a_1 e_1^* \otimes e_1^* + v_2 a_1 e_2^* \otimes e_1^* + \ldots + v_1 a_2 e_1^* \otimes e_2^* + v_2 a_2 e_2^* \otimes e_2^* + \ldots + v_1 a_n e_1^* \otimes e_n^* + \ldots + v_m a_n e_m^* \otimes e_n^* = \\
     & = (v_1 e_1^* + \ldots + v_m e_m^*) \otimes a_1 e_1^* + (v_1 e_1^* + \ldots +v_m e_m^*) \otimes a_2 e_2^* + \ldots +(v_1 e_1^* + \ldots + v_m e_m^*) \otimes a_n e_n^* =                                        \\
     & = (v_1 e_1^* + \ldots + v_m e_m^*) \otimes (a_1 e_1^* + \ldots + a_n e_n^*)
\end{align*}
Dunque una matrice $A \in Mat_{m \times n} (K)$ tale che $rk(A) = 1$ corrisponde ad un tensore $T_A \in (K^m)^* \otimes (K^n)^*$ tale che $rk(T_A) = 1$.

\begin{example}
    La matrice
    \begin{equation*}
        A = \begin{pmatrix}
            1 & 0 & 2 \\
            2 & 0 & 1
        \end{pmatrix} \in Mat_{2 \times 3}(\mathbb{F}_3)
    \end{equation*}
    Ha rango 1 perchè
    $\begin{pmatrix}
            2 \\
            1
        \end{pmatrix} = 2
        \begin{pmatrix}
            1 \\
            2
        \end{pmatrix}$ in $(\mathbb{F}_3)^2$.

    Ad $A$ corrisponde la forma bilineare $T_A: (\mathbb{F}_3)^2 \times (\mathbb{F}_3)^3 \rightarrow \mathbb{F}_3$ definita da
    \begin{equation*}
        T_A(u, v) = u^T A v \qquad \forall u \in (\mathbb{F}_3)^2, v \in (\mathbb{F}_3)^3 \qquad \text{($u^T$ è il trasposto del vettore colonna $u$)}
    \end{equation*}
    come elemento di $(\mathbb{F}_3^2)^* \otimes (\mathbb{F}_3^3)^*$ si scrive
    \begin{align*}
        T_A & = e_1^* \otimes e_1^* + 2 e_2^* \otimes e_1^* + 2 e_1^* \otimes e_3^* + e_2^* \otimes e_3^* \\
            & = (e_1^* + 2 e_2^*) \otimes e_1^* + (2 e_1^* + e_2^*) \otimes e_3^*                         \\
            & = (e_1^* + 2 e_2^*) \otimes (e_1^* + e_3^*)
    \end{align*}
    D'altra parte avevamo che $A = \begin{pmatrix}
            1 & 0 & 2 \\
            2 & 0 & 1
        \end{pmatrix} = \begin{pmatrix}
            1 \\
            2
        \end{pmatrix} \begin{pmatrix}
            1 & 0 & 2
        \end{pmatrix}$ sul campo $\mathbb{F}_3$
\end{example}

Ovviamente ad un Tensore di rango 1 $v_1 \otimes v_2 \in (K^m)^* \otimes (K^n)^*$ corrisponde una matrice di rango 1 $v_1 v_2^T \in Mat_{m \times n}(K)$ dove $v_i$ sono i vettori colonna delle coordinate nella base duale.

\begin{example}
    Sia $(2 e_1^* + 3 e_2^*) \otimes (e_2^* + 4 e_3^*) \in (\mathbb{F}_5^2)^* \otimes (\mathbb{F}_5^3)^*$. La matrice corrispondente è
    \begin{equation*}
        \begin{pmatrix}
            2 \\
            3
        \end{pmatrix} \begin{pmatrix}
            0 & 1 & 4 \\
        \end{pmatrix} = \begin{pmatrix}
            0 & 2 & 3 \\
            0 & 3 & 2
        \end{pmatrix} \in Mat_{2 \times 3}(\mathbb{F}_5)
    \end{equation*}
\end{example}

Quindi abbiamo dato una corrispondenza biunivoca tra matrici di rango 1 $\in Mat_{m \times n}(K)$ e tensori di rango 1 $\in (K^m)^* \otimes (K^n)^*$.

Dalla caratterizzazione del rango di una matrice in termini di combinazioni lineari di matrici di rango 1, e dalla definizione di rango di un tensore, segue che le matrici di rango $r$ in $Mat_{m \times n}(K)$ stanno in corrispondenza con i tensori di rango $r$ in $(K^m)^* \otimes (K^n)^*$.

\newpage


\chapter{Logica modale}
\section{Sintassi della logica modale e semantica di Kripke}
Introduciamo il linguaggio della logica proposizionale. L'alfabeto è costituito da:
\begin{enumerate}
    \item Insieme numerabile di variabili
    \item Connettivi logici: $\neg, \land, \lor$
\end{enumerate}
le parole del linguaggio possono essere:
\begin{enumerate}
    \item \textbf{Un Letterale}: variabile $x$ o la sua negazione $\neg x$
    \item \textbf{Una Clausola}: disgiunzione finita di letterali $l_1 \lor \ldots \lor l_n$
    \item \textbf{Una Formula CNF (forma normale congiuntiva)}: congiunzione finita di clausole $C_1 \land \ldots \land C_n$
\end{enumerate}
Per ogni letterale $L$ definiamo
\begin{equation*}
    \overline{L} = \begin{cases}
        \neg x & \text{se } L \text{ è } x      \\
        x      & \text{se } L \text{ è } \neg x
    \end{cases}
\end{equation*}
Indichiamo con $VAR(F)$ l'insieme delle variabili che appaiono in una formula CNF $F$.

Useremo anche le parentesi $(,)$ come simboli ausiliari per rendere chiara la lettura delle formule CNF.
\begin{example}
    $(\neg x_1 \lor x_2) \land x_1 \land (x_2 \lor x_3)$ è una formula CNF. Chiamiamola F. Allora $VAR(F)=\{x_1,x_2,x_3\}$.

    Le clausole di $F$ sono $\{\neg x_1 \lor x_2, x_1, x_2 \lor x_3\}$ e i letterali di $F$ sono $\{x_1, x_2, x_3, \neg x_1 \}$
\end{example}

\subsection{Semantica}
\begin{definition}[Assegnazione appropriata]
    Sia $F$ una formula CNF. Una \textbf{assegnazione appropriata a F} è una funzione
    \begin{equation*}
        V: X \rightarrow \{0,1\}
    \end{equation*}
    dove $X \supseteq VAR(F)$.
\end{definition}

L'insieme $\{0,1\}$ è l'insieme dei valori di verità di F e può essere interpretato come \{falso, vero\}.

\begin{definition}[Soddisfacibilità]
    Sia $F$ una formula e $V$ un'assegnazione appropriata a $F$.

    Diciamo che \textbf{V soddisfa F}, scritto $V \vDash F$, se
    \begin{enumerate}
        \item \underline{F è una variabile X}: $V \vDash X$, significa che $V(X) = 1$
        \item \underline{F è il letterale $\neg X$}: $V \vDash \neg X$, significa che $V(X) = 0$
        \item \underline{F è una clausola $L_1 \lor \ldots \lor L_n$}: $V \vDash F$, significa che $V$ soddisfa almeno uno dei letterali $L_i$
        \item \underline{F è una congiunzione di clausole} $C_1 \land \ldots \land C_n$: $V \vDash F$, significa che $V$ soddisfa tutte le clausole $C_i$
    \end{enumerate}
    In questo modo abbiamo dato un significato al linguaggio definito precedentemente.

    Se $V$ non soddisfa $F$, scriveremo $V \nvDash F$.
\end{definition}
\begin{example}
    Sia $F$ la formula CNF $(\neg x_1 \lor x_2) \land x_1 \land (x_2 \lor x_3)$ e sia $U: \{x_1,x_2,x_3\} \rightarrow \{0,1\}$ tale che $U(x_1) = U(x_2) = 1, U(x_3) = 0$.

    Dunque
    \begin{align*}
        U(\neg x_1)          & = 0 \\
        U(\neg x_1 \lor x_2) & = 1 \\
        U(x_2 \lor x_3)      & = 1 \\
        U(F)                 & = 1
    \end{align*}
    Quindi $U \vDash F$.

    Sia $V: \{x_1,x_2,x_3\} \rightarrow \{0,1\}$ tale che $V(x_1) = 1, V(x_2) = V(x_3) = 0$.

    Allora
    \begin{align*}
        V(\neg x_1)          & = 0 \\
        V(\neg x_1 \lor x_2) & = 0 \\
        V(x_2 \lor x_3)      & = 0 \\
        V(F)                 & = 0
    \end{align*}
    quindi $V \nvDash F$.
\end{example}

\begin{definition}[Formula soddisfacibile]
    Diciamo che una formula è \textbf{soddisfacibile} se esiste un'assegnazione $V: VAR(F) \rightarrow \{0,1\}$ che la soddisfa ($V \vdash F$).

    Altrimenti è \textbf{insoddisfacibile}.
\end{definition}
La formula dell' esempio precedente è soddisfacibile, $x \land \neg x$ è insoddisfacibile.

\begin{definition}[Tautologia]
    Una formula $F$ è una \textbf{tautologia} se per ogni assegnazione $V$ si ha $V \vDash F$.
\end{definition}
La formula $x \lor \neg x$ è una tautologia, la formula dell' esempio precedente no

\begin{definition}[Conseguenza logica]
    Date due formule $F,G$ diciamo che $G$ è \textbf{conseguenza logica} di $F$, se per ogni assegnazione $V$ appropriata ad entrambe si ha che $V \vDash F \implies V \vDash G$.
\end{definition}
\begin{example}
    La formula $y$ è conseguenza logica della formula $F := (\neg x \lor y) \land x$.

    Infatti l'unica assegnazione che soffisfa $F$ è $x \rightarrow 1, y \rightarrow 1$.

    Tale assegnazione soddisfa anche $y$.
\end{example}
\begin{definition}[Implicazione logica]
    Definiamo l'\textbf{implicazione logica} tra due formule $F,G$ come
    \begin{gather*}
        x \implies y := \neg x \lor y \\
        \begin{tabular}{c|c|c}
            \hline
            $x$ & $y$ & $x \implies y$ \\
            \hline
            0   & 0   & 1              \\
            0   & 1   & 1              \\
            1   & 0   & 0              \\
            1   & 1   & 1              \\
        \end{tabular}
    \end{gather*}
\end{definition}
\begin{definition}[Equivalenza logica]
    Due formule $F,G$ sono logicamente equivalenti se $F$ è conseguenza logica di $G$ e $G$ è conseguenza logica di $F$. In tal caso scriviamo $F \equiv G$.
\end{definition}
\begin{example}
    Nell'esempio precedente abbiamo visto che $y$ è conseguenza logica di $F:= (\neg x \lor y) \land x $, che possiamo ricrivere come $(x \implies y) \land x$.

    Poiché $x \mapsto 0, y \mapsto 1$ soddisfa $y$ ma non $F$, abbiamo che $F$ non è conseguenza logica di $y$.
\end{example}
\begin{example}\
    \begin{itemize}
        \item $l_1 \lor l_2 \equiv l_2 \lor l_1$ (la disgiunzione è commutativa)
        \item $c_1 \land c_2 \equiv c_2 \land c_1$ (la congiunzione è commutativa)
        \item $c \land c \equiv c$ e $l \lor l \equiv l$ (congiunzione e disgiunzione sono idempotenti)
    \end{itemize}
\end{example}
Diamo altre definizioni:
\begin{definition}[Doppia implicazione]
    \begin{equation*}
        x \iff y := (x \implies y)\land(y \implies x)
    \end{equation*}
\end{definition}

\begin{definition}[Negazione di formule CNF]
    \begin{enumerate}
        \item $l$ letterale: $\neg l = \overline{l}$
        \item $\neg (l_1 \lor \ldots \lor l_n) := \neg l_1 \land \ldots \land \neg l_n$
        \item $\neg (c_1 \land \ldots \land c_n) := \neg c_1 \lor \ldots \lor \neg c_n$
    \end{enumerate}
\end{definition}
L'interpretazione di questa definizione di negazione è quella corretta grazie alle leggi di De Morgan e alla proprietà distributiva di $\lor$ e $\land$.
\begin{example}
    Consideriamo il problema di colorare i vertici di un quadrato con due colori in modo che i vertici su uno stesso lato abbiano colori diversi.

    Tale problema ha ovviamente una soluzione:
    \begin{center}
        \begin{tikzpicture}
            \draw (0,0) -- (2,0) -- (2,2) -- (0,2) -- (0,0);
            \draw (0,0) node[below left] {2};
            \draw (2,0) node[below right] {1};
            \draw (2,2) node[above right] {2};
            \draw (0,2) node[above left] {1};
        \end{tikzpicture}
    \end{center}
    Formuliamo il problema come una formula CNF.

    Potremo così dire che il problema è soddisfacibile se e solo se tale formula CNF è soddisfacibile. $x_{ij}$ indica "il vertice $i$ ha colore $j$" $\forall 1 \leq i \leq 4, 1 \leq j \leq 2$
    \begin{align*}
        F & := (x_{11} \lor x_{12}) \land (x_{21} \lor x_{22}) \land (x_{31} \lor x_{32}) \land (x_{41} \lor x_{42}) \land                                                                              \\
          & \land (\neg x_{11} \lor \neg x_{12}) \land ( \neg x_{21} \lor \neg x_{22}) \land ( \neg x_{31} \lor \neg x_{32}) \land ( \neg x_{41} \lor \neg x_{42}) \land                                \\
          & \land (\neg x_{11} \lor \neg x_{21}) \land ( \neg x_{12} \lor \neg x_{22}) \land ( \neg x_{21} \lor \neg x_{31}) \land ( \neg x_{22} \lor \neg x_{32}) \land (\neg x_{31} \lor \neg x_{41})
    \end{align*}
    l'assegnazione $x_{11} \mapsto 1, x_{12} \mapsto 0, x_{21} \mapsto 0, x_{22} \mapsto 1, x_{31} \mapsto 1, x_{32} \mapsto 0, x_{41} \mapsto 0, x_{42} \mapsto 1$ soddisfa $F$
\end{example}

\subsection{Sintassi}
La logica modale è una estensione della logica proposizionale.

L'alfabeto è quello della logica proposizionale a cui si aggiungono i \textbf{connettivi modali}:
\begin{enumerate}
    \item Un insieme numerabili di variabili (o formule atomiche)
    \item I connettivi logici $\neg, \land, \lor, \implies, \iff$
    \item I simboli ausiliari $(,)$
    \item I connettivi modali $\Box$ (\textbf{Scatola o Box}) e $\Diamond$ (\textbf{Diamante o Diamond})
\end{enumerate}
Le parole del lingiaggio sono le formule ben formate (FBF) definite in modo ricorsivo:
\begin{enumerate}
    \item Ogni variabile è una FBF
    \item Se $A$ è una FBF, allora $\neg A, \Box A, \Diamond A$ sono FBF
    \item Se $A,B$ sono FBF, allora $(A \land B), (A \lor B), (A \implies B), (A \iff B)$ sono FBF
\end{enumerate}
Alcune letture dei simboli $\Box$ e $\Diamond$:
\begin{itemize}
    \item La lettura piu comune è: $\Box A$:"è necessario che A", $\Diamond A$: "è possibile che A".

          Secondo questa lettura i connettivi modali possono essere definiti uno in termini dell'altro:
          \begin{gather*}
              \Box A \equiv \neg \Diamond \neg A\\
              \Diamond A \equiv \neg \Box \neg A
          \end{gather*}
    \item Logiche modali epistemiche: $\Box A$: "si sa che A"
    \item Logiche modali deonetiche: $\Box A$: "è obbligatorio che A"
    \item Logiche modali doxastiche: $\Box A$: "si crede che A"
    \item Logica modale dimostrativa: $\Box A$: "è dimostrabile che A"
\end{itemize}

Come abbiamo visto, la logica proposizionale è una logica vero-funzionale: assegnando valori "0" e "1" alle variabili possiamo assegnare un valore "0" o "1" ad una formula in modo univoco, che corrisponde alla nostra inutizione di negazione, disgiunzione, congiunzione.

Per la logica modale la situazione è più complicata.

Interpretando il simbolo "$\Box$" come operatore di necessità, ossia "$\Diamond$" come operatore di possibilità, possiamo essere, ad esempio, d'accordo che le formule
\begin{equation*}
    \Box A \implies \Diamond A, \quad A \implies \Diamond A
\end{equation*}
siano vere, ma è vera la formula
\begin{equation*}
    A \implies \Box \Diamond A \quad \text{?}
\end{equation*}
Non è chiaro se sia vera o falsa. Nel caso della logica epistemica, l'operatore "$\Box$" si indica di solito con "K" (da "knowledge").

In questo contesto, la formula $KA \implies A$ (se si sa che A allora A vale) sembra dover essere vera.

Invece la formula $A \implies KA$ (se vale A allora si sa che A) sembra essere falsa perchè non si è onniscenti.

\subsection{Semantica dei mondi possibili (semantica di Kripke)}
\begin{definition}[Frame]
    Un \textbf{Frame} è una coppia $(S,R)$, dove $S$ è un insieme non vuoto detto \textbf{insieme dei mondi} e $R \subseteq S \times S$ è una relazione su $S$, detta \textbf{relazione di accessibilità} (se $(x,y) \in R$ si dice che $y$ è accessibile da $x$).
\end{definition}
Un Frame può essere rappresentato con un grafo diretto con cappi (loop) i cui vertici sono gli elementi dell'insieme $S$ e ho una freccia da $x$ a $y$ se $(x,y) \in R$.

\begin{example}
    Il frame $(\mathbb{N},R)$, dove $R = \{(n,n+1) : n \in \mathbb{N}\} \subseteq \mathbb{N} \times \mathbb{N}$ è rappresentato dal seguente grafo diretto:
    \begin{center}
        \begin{tikzpicture}[->]
            \node (0) {0};
            \node (1) [right of=0] {1};
            \node (2) [right of=1] {2};
            \node (3) [right of=2] {3};
            \node (4) [right of=3] {...};

            \path
            (0) edge node {} (1)
            (1) edge node {} (2)
            (2) edge node {} (3)
            (3) edge node {} (4);
        \end{tikzpicture}
    \end{center}
\end{example}
\begin{example}
    Il frame $(S,R)$ dove $S = \{2,3,4,5,6\}$ e $R=\{(x,y) \in S \times S : x \text{ divide } y \}$ è rappresentato dal seguente grafo diretto:
    \begin{center}
        \begin{tikzpicture}[->]
            \node (2) {2};
            \node (3) [right of=2] {3};
            \node (4) [right of=3] {4};
            \node (5) [right of=4] {5};
            \node (6) [right of=5] {6};

            \path
            (2) edge[loop] node {} (2)
            (2) edge[bend right] node {} (4)
            (2) edge[bend left] node {} (6)
            (3) edge[loop] node {} (3)
            (3) edge[bend right] node {} (6)
            (4) edge[loop] node {} (4)
            (4) edge[bend left] node {} (6)
            (5) edge[loop] node {} (5)
            (5) edge node {} (6)
            (6) edge[loop] node {} (6);
        \end{tikzpicture}
    \end{center}
\end{example}
\begin{example}
    Il frame $(\{1,2,3\}, R)$ dove $R = \{(x,y) \in \{1,2,3\} \times \{1,2,3\} : y=f(x)\}$ essedo $f : \{1,2,3\} \rightarrow \{1,2,3\}$ la funzione definita da $f(1) = 2, f(2) = 3, f(3) = 1$ è
    \begin{center}
        \begin{tikzpicture}[->]
            \node (1) {1};
            \node (2) [right of=1] {2};
            \node (3) [right of=2] {3};

            \path
            (1) edge[bend left] node {} (2)
            (2) edge[bend left] node {} (3)
            (3) edge[bend left] node {} (1);
        \end{tikzpicture}
    \end{center}
\end{example}
\begin{definition}[Modello]
    Un \textbf{modello} su un frame $(S,R)$ è una terna $(S,R,V)$ dove $V : Var \rightarrow \mathcal{P}(S) $, ci dice in quali mondi le variabili valgono 1 è detta \textbf{funzione di valutazione}.
\end{definition}
\begin{definition}
    Una formula $F$ si dice \textbf{Vera in un mondo x del modello M} e scriviamo $M \vDash_x F$ se e solo se:
    \begin{enumerate}
        \item \underline{F è una variabile}: $M \vDash_x F$ significa che $x \in V(F)$
        \item \underline{ F è $\neg y$ e $y$ è una variabile}: $M \vDash_x F$ significa che $x \notin V(y)$
        \item \underline{F è del tipo $\neg G$}, dove $G$ è una formula: $M \vDash_x F$ significa che $M \nvDash_x G$
        \item \underline{F è del tipo $G_1 \land G_2$}: $M \vdash_x F$ significa che $M \vdash_x G_1$ e $M \vDash_x G_2$
        \item \underline{F è del tipo $G_1 \lor G_2$}: $M \vdash_x F$ significa che $M \vdash_x G_1$ o $M \vDash_x G_2$
        \item \underline{F è del tipo $\Box G$}: $M \vDash_x F$ significa che $M \vDash_y G$, per ogni $y \in S : (x,y) \in R$, ossia per ogni mondo y raggiungibile da $x$.
        \item \underline{F è del tipo $\Diamond G$}: $M \vDash_x F$ significa che $M \vDash_y G$ per qualche $y \in S : (x,y) \in R$, ossia per almeno un mondo raggiungibile da $x$.
    \end{enumerate}
\end{definition}
\begin{definition}[Soddisfacibilità]
    Una formula $F$ è \textbf{soddisfacibile} se esiste un modello $M = (S, R, V)$ e un mondo $x \in S$, tali che $M \vDash_x F$.
\end{definition}
\begin{theorem}
    Se una formula modale $F$ è soddisfacibile, allora è soddisfacibile in una struttura di Kripke $(S,R)$ tale che $|S| \leq 2^{|F|}$, $|F| = $"lunghezza di F".

    Quindi il problema di soddisfacibilità di una formula modale è decidibile.
\end{theorem}
\newpage

\section{Esprimibilità della proprietà riflessiva}
\begin{theorem}
    Lo schema $\Box A \implies A$ è valido in un frame $(S,R)$ se e solo se $R$ è riflessiva.
\end{theorem}
\begin{proof}
    Perchè sia sempre vera la formula $\Box x \implies x$, la relazione $R$ del frame deve essere riflessiva, ossia $(y,y) \in R \forall y \in S$.

    Infatti, se $R$ non fosse riflessiva ci sarebbe un mondo $y \in S$ tale che $(y,y) \notin R$.

    TODO
    %TODO
\end{proof}
\newpage

\section{Esprimibilità della proprietà simmetrica}
\begin{theorem}
    Lo schema $ A \implies \Box \Diamond A$ è valido in un frame $(S,R)$ se e solo se $R$ è simmetrica.
\end{theorem}
\begin{proof}
    Sia $R$ simmetrica, ossia $(x,y) \in R \implies (y,x) \in R$.

    Sia $M \vDash_w A$ e $(w,v) \in R$. Dunque $(v,w) \in R$ e $M \vDash_v \Diamond A \forall v \in S$ t.c. $(w,v) \in R$, ossia $M \vDash_w \Box \Diamond A$.

    Adesso assumiamo che lo schema $A \implies \Box \Diamond A $ sia valido in $(S,R)$.

    Sia x una variabile e $V(x) = \{s\}$;

    sia $t \in S$ t.c. $(s,t) \in R$. Quindi $M \vDash_s x$.

    Dalla validità dello schema segue allora che $M \vDash_s \Box \Diamond x$, da cui $M \vDash_t \Diamond x$.

    Quindi esiste $r \in S$ t.c. $(t,r) \in R e M \vDash_r x$, ossia $r = s$
\end{proof}
\newpage

\section{Esprimibilità della proprietà transitiva}
\begin{theorem}
    Lo schema $\Box A \implies \Box \Box A$ è valido in un frame $(S,R)$ se e solo se R è transitiva.
\end{theorem}
\begin{proof}
    Sia $R$ transitiva, ossia $(x,y) \in R, (y,z) \in R \implies (x,z) \in R$.

    Sia $M \vDash_w \Box A$, ossia $M \vDash_v A \forall v \in S$ t.c. $(w,v) \in R$.

    Sia $u \in S$ t.c. $(v,u) \in R$, con $(w,v) \in R$.

    Allora $(w,u) \in R$ e quindi $M \vDash_v \Box A, \forall v \in S$ t.c. $(w,v) \in R$, ossia $M \vDash_w \Box \Box A$.

    Assumiamo adesso che sia valido lo schema $\Box A \implies \Box \Box A$ su un frame $(S,R)$.

    Sia $x$ una variabile, $s \in S$ e $V(x) = \{w \in S : (s,w) \in R\}$.

    Allora $M \vDash_s \Box x$ e quindi per la validità dello schema, $M \vDash_s \Box \Box x $, da cui $M \vDash_t \Box x \forall t \in S$ t.c. $(s,t) \in R$, ossia $M \vDash_r x \forall r \in S$ t.c. $(t,r) \in R$, $(s,t) \in R$.

    Da ciò segue che $r \in V(x)$ ossia $(s,t) \in R$ e $(t,r) \in R \implies (s,r) \in R$\\
\end{proof}
\newpage

\section{Morfismi di modelli + Lemma 1}
\begin{definition}[Morfismo di Frame]
    Siano $(S_1, R_1)$ e $(S_2, R_2)$ due frame. una funzione $f : S_1 \rightarrow S_2$ è un \textbf{morfismo di frame} se:
    \begin{equation*}
        (x,y)\in R_1 \implies (f(x), f(y)) \in R_2 \qquad \forall x,y \in S_1
    \end{equation*}
\end{definition}
\begin{example}
    Siano $(\mathbb{N}, R_1)$ e $(\mathbb{N}, R_2)$ i frame tali che
    \begin{equation*}
        R_1 = R_2 = \{(x,y) \in \mathbb{N} \times \mathbb{N} : x < y\}
    \end{equation*}
    allora la funzione
    \begin{align*}
        f : \; \mathbb{N} & \rightarrow \mathbb{N} \\
        n                 & \mapsto n^2
    \end{align*}
    è un morfismo di frame.
\end{example}

\begin{definition}[Morfismo di modelli]
    siano $M_1 = (S_1, R_1, V_1)$ e $M_2 = (S_2, R_2, V_2)$ due modelli.

    Un morfismo di frame $f: (S_1, R_1) \rightarrow (S_2, R_2)$ è un \textbf{morfismo di modelli} se:
    \begin{enumerate}
        \item $w \in V_1(x) \iff f(w) \in V_2(x)$ $\forall w \in S_1, x \in Var$
        \item $(f(w),y) \in R_2 \implies \exists v \in S_1 t.c. (w,v) \in R_1, f(v) = y$ $\forall w \in S_1, y \in S_2$
    \end{enumerate}
\end{definition}
\begin{note}
    I morfismi di modelli sono solitamente detti \textbf{p-morfismi}.
\end{note}
\begin{example}
    Siano $M_1 = (\mathbb{N}, R_1, V_1)$ e $M_2 = (\{0,1\}, \{0,1\} \times \{0,1\}, V_2)$ dove $R_1 = \{(x,y) \in \mathbb{N} \times \mathbb{N} : x \leq y\}$, $Var = \{x\}$ e $V_1(x)= \{2n : n \in \mathbb{N}\}, V_2(x) = \{0\}$. Sia
    \begin{align*}
        f : \; \mathbb{N} & \rightarrow \{0,1\} \\
        n                 & \mapsto n \bmod 2
    \end{align*}
    Allora $f$ è un morfismo di modelli, infatti:
    \begin{enumerate}
        \item $x \leq y \implies (x \bmod 2, y \bmod 2) \in \{0,1\} \times \{0,1\}$
        \item $w \in V_1(x) \iff w \in \{2n: n \in \mathbb{N}\}$; allora $w \in V_1(x) \implies f(w) = 0$.

              $f(w) \in V_2(x) \iff f(w) = 0 $; allora $f(w) \in V_2(x) \implies w \in V_1(x)$
        \item $(f(w),y) \in R_2 = \{0,1\} \times \{0,1\}$:
              \begin{enumerate}
                  \item $f(w) = 0$: se $y=0$ allora $w \leq w$ e $f(w) = 0 = y$.

                        Se $y=1$ allora $w \leq w+1$ e $f(w+1) = 1 = y$.
                  \item $f(w) = 1$: se $y=0$ allora $w \leq w + 1$ e $f(w + 1) = 0 = y$.

                        Se $y=1$ allora $w \leq w$ e $f(w) = 1 = y$.
              \end{enumerate}
    \end{enumerate}
\end{example}
\begin{lemma}[Lemma 1]
    Sia $f: (S_1, R_1, V_1) \rightarrow (S_2, R_2, V_2)$ un morfismo dal modello $M_1$ al modello $M_2$. Allora
    \begin{equation*}
        M_1 \vDash_w F \iff M_2 \vDash_{f(x)} F
    \end{equation*}
    $\forall w \in S_1$ e ogni formula $F$
\end{lemma}
\begin{proof}
    Se $F$ è una variabile allora $M_1 \vDash_w F \text{ se e solo se } w \in V_1(F) \text{se e solo se } f(w) \in V_2(F)$ (per il punto 1 nella definizone di morfismo di modelli) se e solo se $M_2 \vDash_{f(w)} F$.

    Per tutti gli altri tipi di formule, si dimostra induttivamente sulla costruzione della formula.

    Vediamo dsolo il caso in cui $F = \Diamond G$.

    Sia $M_1 \vDash_w \Diamond G$, allora esiste $v \in S_1$ t.c. $(w,v) \in R_1$ e $M_1 \vDash_v G$.

    Poiché $(f(w),f(v)) \in R_2$ perchè $f$ è un morfismo di modelli e induttivamente $M_2 \vDash_{f(v)} G$, allora $M_2 \vDash_{f(w)} \Diamond G$.

    Sia ora $M_2 \vDash_{f(w)} \Diamond G$, allora esiste $u \in R_2$ t.c. $(f(w),u) \in R_2$ e $M_2 \vDash_u G$.

    Per la condizione 2 nella definizione di morfismo di modelli, esiste $v \in S_1$ t.c. $(w,v) \in R_1$ e $f(v) = u$.

    Per ipotesi induttiva $M_1 \vDash_v G$, e quindi $M_1 \vDash_w \Diamond G$.
\end{proof}

\newpage
\section{Lemma 2 + Lemma 3 + Lemma 4 + non esprimibilità della proprietà antisimmetria}
\begin{lemma}[Lemma 2]
    Sia $f: (S_1,R_1,V_1) \rightarrow (S_2,R_2,V_2)$ un morfismo dal modello $M_1$ al modello $M_2$. se $f$ è suriettiva, allora
    \begin{equation*}
        M_1 \vDash F \text{ se e solo se } M_2 \vDash F
    \end{equation*}
    per ogni formula $F$.
\end{lemma}
\begin{proof}
    $M_1 \vDash F$ se e solo se $M_1 \vDash_w F, \forall w \in S_1$.

    Se e solo se $M_2 \vDash_{f(w)} F, \forall w \in S_1$ (per il lemma 1).

    Se e solo se $M_2 \vDash F$, perchè $f$ è suriettivo.
\end{proof}
\begin{lemma}[Lemma 3]
    Sia $M_2$ un modello su $S_2, R_2$ e $f: (s_1,R_1) \rightarrow (S_2, R_2)$ un morfismo di frame tale che valga la condizione 2 della definizione di morfismo di modelli.

    Allora esiste un modello $M_1$ su $S_1, R_1$ tale che $f: M_1 \rightarrow M_2$ è un morfismo di modelli.
\end{lemma}
\begin{proof}
    Basta definire $M_1 = (S_1, R_1, V_1)$ con $V_1(x) = \{ w \in S_1 : M_2 \vDash_{f(w)} x\} \forall x \in Var$.
\end{proof}
\begin{lemma}[Lemma 4]
    Sia $f: (S_1, R_1) \rightarrow (S_2, R_2)$ un morfismo di frame tale che valga la condizione 2 della definizione di morfismo di modelli.

    Se $f$ è suriettivo, si ha $(S_1, R_1) \vDash F \implies (S_2, R_2) \vDash F$, per ogni formula $F$.
\end{lemma}
\begin{proof}
    Sia $S_2, R_2 \nvDash F$. Allora esiste un modello $M_2$ su $(S_2, R_2)$ tale che $M_2 \nvDash F$. Per il lemma 3 esiste un modello $M_1$ su $(S_1, R_1)$ tale che $f: M_1 \rightarrow M_2$ è un morfismo di modelli.

    Dato che $f$ è suriettivo, per il lemma 2 si ha $M_1 \nvDash F$, ossia $(S_1, R_1) \nvDash F$.
\end{proof}
\begin{definition}[Relazione antisimmetria]
    Una relazione $R$ su un insieme $X$ si dice \textbf{antisimmetrica} se
    \begin{equation*}
        (x,y) \in R, (y,x) \in R \implies x = y \qquad \forall x,y \in X
    \end{equation*}
\end{definition}
\begin{example}
    L'ordinamento $\leq$ dei numeri naturali è una relazione antisimmetrica su $\mathbb{N}$.
\end{example}
\begin{example}
    La relazione $x \mid y$ su $\mathbb{N}$ è antisimmetrica.
\end{example}
\begin{example}
    La relazione $A \subseteq B$ su $\mathcal{P}(X)$ di un insieme $X$ è antisimmetrica.
\end{example}
\begin{theorem}
    L'antisimmetria non è esprimibile, ossia non esiste una formula $F$ tale che $(S,R) \vDash F$ se e solo se $R$ è antisimmetrica.
\end{theorem}
\begin{proof}
    Sia $(S_1,R_1) = (\mathbb{N},\leq)$ e $(S_2, R_2) = (\{0,1\}, \{0,1\} \times \{0,1\})$.

    Nell'esempio di morfismo di modelli abbiamo visto che la funzione
    \begin{align*}
        f: \; \mathbb{N} & \rightarrow \{0,1\} \\
        n                & \mapsto n \bmod 2
    \end{align*}
    è un morfismo dal frame $(\mathbb{N},\leq)$ al frame $(\{0,1\}, \{0,1\} \times \{0,1\})$ che soffisfa la condizione 2 della definizione di morfismo di modelli.

    La relazione $\leq$ su $\mathbb{N}$ è antisimmetrica.

    Supponiamo per assurdo che esista una formula $F$ come nell'enunciato del teorema. Allora:
    \begin{equation*}
        (\mathbb{N},\leq) \vDash F
    \end{equation*}
    Per il lemma 4 si ha che $(\{0,1\}, \{0,1\} \times \{0,1\}) \vDash F$.

    Da cui seguirebbe che $R_2$ è antisimmetrica, il che è falso.
\end{proof}

\newpage
\section{Logiche modali normali, dimostrazioni, teoremi e validità della logica K}
Abbiamo già mostrato che lo schema di formule
\begin{equation*}
    K: \Box (A \implies B) \implies (\Box A \implies \Box B)
\end{equation*}
è valido, $\vDash K$.

Adesso viediamo che lo schema di formule
\begin{equation*}
    def_\Diamond : \Diamond A \iff \neg \Box \neg A
\end{equation*}
è valido, $\vDash def_\Diamond$.

\begin{proof}
    Sia $(S,R)$ un frame, $M$ un modello su $(S,R)$ e $w \in S$.

    Allora $M \vDash_w \Diamond A$ se e solo se esiste $v \in S t.c. (w,v) \in R$ e $M \vDash_v A$.

    $M \vDash_w \neg \Box \neg A$ se e solo se $M \nvDash_w \Box \neg A$, se e solo se esiste $v \in S t.c. (w,v) \in R$ e $M \nvDash_v \neg A$, se e solo se esite $v \in S t.c. (w,v) \in R$ e $M \vDash_v A$.

    Abbiamo quindi dimostrato che $\vDash def_\Diamond$
\end{proof}
\begin{definition}[Sostituzione uniforme]
    Sia $x$ una variabile e $F$ una formula. Definiamo l'operzione di sostituzione uniforme di $F$ al posto di $x$ in una formula $G$, indicato come
    \begin{equation*}
        G[F/x]
    \end{equation*}
    La formula ottenuta da $G$ dove ogni occorrenza di $x$ è stata sostituita con $F$.
\end{definition}
\begin{example}
    Sia $G$ la formula $\Box x \implies x \land y$ e $F$ la formula $\Diamond y \iff \neg \Box \neg y$.

    Allora $G[F/x] = \Box (\Diamond y \iff \neg \Box \neg y) \implies (\Diamond y \iff \neg \Box \neg y) \land y$
\end{example}
\begin{definition}[Logica Modale Normale]
    Una \textbf{logica modale normale} è un insieme $\Gamma$ di formule tale che:
    \begin{enumerate}
        \item $\Gamma$ contiene tutte le tautologie della logica proposizionale
        \item $\Gamma$ contiene tutte le istanze dello schema $K: \Box (A \implies B) \implies (\Box A \implies \Box B)$
        \item $\Gamma$ contiene tutte le istanze dello schema $def_\Diamond : \Diamond A \iff \neg \Box \neg A$
        \item $\Gamma$ è chiuso sotto \textbf{modus ponens} se $A \in \Gamma$ e $(A \implies B) \in \Gamma$, allora $B \in \Gamma$
        \item $\Gamma$ è chiuso sotto \textbf{necessitazione} se $A \in \Gamma$, allora $\Box A \in \Gamma$
        \item $\Gamma$ è chiuso sotto \textbf{sostituzione uniforme} se $A \in \Gamma$, allora $A[B/x] \in \Gamma$
    \end{enumerate}
\end{definition}
\begin{example}
    Se $(S,R)$ è un frame, $\{F: (S,R) \vDash F\}$ è una logica normale.
\end{example}
\begin{example}
    $\{F: \; \vDash F\}$ è una logica normale.
\end{example}
\begin{example}
    Se $M$ è un modello su un frame $(S,R)$, $\{F: \; M \vDash F\}$ NON è una logica normale.
\end{example}
\begin{definition}[Logica Modale K]
    La \textbf{logica modale K} è definita dai seguenti schemi di assioni e regole:
    \begin{enumerate}
        \item Schemi di assiomi:
              \begin{enumerate}
                  \item Tutte le tautologie della logica proposizionale
                  \item $K: \Box (A \implies B) \implies (\Box A \implies \Box B)$
                  \item $def_\Diamond : \Diamond A \iff \neg \Box \neg A$
              \end{enumerate}
        \item Regole di inferenza:
              \begin{enumerate}
                  \item Modus ponens
                  \item Necessitazione
                  \item Sostituzione uniforme
              \end{enumerate}
    \end{enumerate}
\end{definition}
\begin{definition}[Dimostrazione in una logica modale]
    Data una logica modale L una \textbf{dimostrazione in L} è una successione finita di formule tali che ognuna di esse o è un assioma o è ottenuta da formule precedenti tramite una regola di inferenza
\end{definition}
\begin{definition}[Teorema di una logica modale]
    Una formula $F$ si dice \textbf{teorema di L}, scritto $\vdash_L F$ se e solo se esite una dimostrazione in $L$ in cui la ultima formula è $F$
\end{definition}
\begin{example}
    $\Box (A \land B) \implies \Box A$ è un teorema della logica K:
    \begin{enumerate}
        \item $\vdash_K A \land B \implies A$ (Tautologia)
        \item $\vdash_K \Box (A \land B \implies A)$ (Necessitazione)
        \item $\vdash_K \Box (A \land B \implies A ) \implies (\Box (A \land B ) \implies \Box A)$ (K)
        \item $\vdash_K \Box (A \land B ) \implies \Box A$ (Modus Ponens)
    \end{enumerate}
\end{example}
Quindi la logica basata su principi episistemici che abbiamo considerato ragionevoli si supporta su frame che sono relazioni di equivalenza.
\begin{definition}[Logica Valida]
    Una logica L è \textbf{valida (sound)} se
    \begin{equation*}
        \vdash_L A \implies \vDash A
    \end{equation*}
\end{definition}
\begin{theorem}
    La logica K è valida.
\end{theorem}
\begin{proof}
    Sia $B_1,B_2,\ldots,B_n$ una dimostrazione di A in K, con $B_n \equiv A$.

    $B_1$ è valida perchè è un assioma.

    Se $i > 1$, allora $B_i$ è un assioma o è ottenuta da formule precedenti tramite necessitazione o modus ponens.
    \begin{enumerate}
        \item $\vdash_K B_j$, per induzione $\vDash B_j$ e allora $\vDash \Box B_j$, quindi $B_i = \Box B_j$ è valida.
        \item $\vdash_K B_j, \vdash_K B_h$ dove $B_h \equiv B_i \implies B_i$. Allora per induzione $\vDash B_j, \vDash B_j \implies B_i$\\
              e quindi $\vDash B_i$
    \end{enumerate}
\end{proof}
\begin{definition}[Logica Completa]
    Una logica L è \textbf{completa} se
    \begin{equation*}
        \vDash A \implies \vdash_L A
    \end{equation*}
\end{definition}
\begin{theorem}
    La logica K è completa.
\end{theorem}
\begin{proof}
    vedee cap.4 "Corso di Logica modale proposizionale"
\end{proof}



\end{document}
